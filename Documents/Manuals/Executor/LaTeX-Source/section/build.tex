
\section{Build the Executor Project Structure with Autotools}\label{section:build}
This section provides a tutorial on  how the user can build the structure of the executor project for distribution.
\subsection{Generate the Default Structure}\label{subsection:default}

The execution of the script \texttt{autotools.sh} generates the structure for a simple ``Hello World'' project. To create the executor project using autotools files, the following instructions can be followed:
\begin{enumerate}
 \item Create a directory and name it \textbf{executor}
 \item Move the script \texttt{autotools.sh} in the directory \textbf{executor}
 \item Run \texttt{autotools.sh} with \texttt{./ autotools.sh}
\end{enumerate}


The structure of the new directory is depicted in Figure~\ref{fig:default-executor}.
%\begin{itemize}
%\item[\IconFolder] \directory{executor}
%  \begin{itemize}
%    \item[\IconFolder] \directory{bin}
%      \begin{itemize}
%	    \item[\IconFile] Makefile.am
%      \end{itemize}
%    \item[\IconFolder] \directory{config}
%      \begin{itemize}
%	\item[\IconText] README
%      \end{itemize}
%    \item[\IconFolder] \directory{lib}
%      \begin{itemize}
%	\item[\IconFile] Makefile.am
%      \end{itemize}
%    \item[\IconFolder] \directory{src}
%      \begin{itemize}
%	\item[\IconFileCSource] main.c
%	\item[\IconFileCSource] hello.c
%	\item[\IconFileHeader] hello.h
%      \end{itemize}
%    \item[\IconText] AUTHORS
%    \item[\IconText] NEWS
%    \item[\IconText] README
%    \item[\IconText] ChangeLog
%    \item[\IconFile] acinclude.m4
%    \item[\IconFile] autoconf.sh
%    \item[\IconFile] configure.ac
%    \item[\IconFile] Makefile.am
%  \end{itemize}
%\end{itemize}

\begin{figure}[h!]
\centering
\scalebox{.7}{
\begin{tikzpicture}[
  grow via three points={one child at (0.5,-0.7) and
  two children at (0.5,-0.7) and (0.5,-1.4)},
  edge from parent path={(\tikzparentnode.south) |- (\tikzchildnode.west)}]
  \node[folder] {executor}
    child { node[folder] {bin}
        child { node {Makefile.am}}}	
    child [missing] {}	
    child { node[folder] {config}
        child { node {README}}}	
    child [missing] {}	
    child { node[folder] {lib}
    child { node {Makefile.am}}}
     child [missing] {}	
    child { node[folder] {src}
        child { node {main.c}}
        child { node {hello.c}}
        child { node {hello.h}}
    }	
    child [missing] {}
    child [missing] {}	
    child [missing] {}	
    child { node {AUTHORS}}
    child { node {NEWS}}
    child { node {README}}
    child { node {ChangeLog}}
    child { node {acinclude.m4}}
    child { node {autoconf.sh}}
    child { node {autotools.sh}}
    child { node {configure.ac}}
    child { node {Makefile.am}};
\end{tikzpicture}
}
\caption{Default structure of the project executor.}
\label{fig:default-executor}
\end{figure}
\subsection{Modify the Default Structure for the New Project}
The default structure presented in section~\ref{subsection:default} needs to be modified as follows in order to include and use the source files for the executor.
\begin{itemize}
  \item Delete the \directory{lib} directory: We don't need this directory since the generated libraries will be stored in the same directory as the source files.
  \item Delete \texttt{main.c}, \texttt{hello.c}, and \texttt{hello.h} in the \directory{src} directory: These files are required for a simple ``Hello World'' project and are not needed for our project.
  \item Add \texttt{Makefile.am} in the \directory{bin} directory: This file will be used to generate the executable.
  \item In the \texttt{src} directory:
\begin{itemize}
  \item Add \texttt{Makefile.am}.
  \item Add the \directory{controller} directory with the source files.
  \item Add the \directory{database} directory with the source files.
    \begin{itemize}
      \item Create \texttt{Makefile.am} in this directory.
    \end{itemize}
  \item Add the \directory{interpreter} directory with the source files.
    \begin{itemize}
      \item Create \texttt{Makefile.am} in this directory.
    \end{itemize}
\end{itemize}
  \item Add the \directory{etc} directory along with the files in it.
\item Edit \texttt{configure.ac}
\item Create and edit the script \texttt{bootstrap.sh}.
\end{itemize}


The previous steps generates the new structure of the executor project as depicted in Figure~\ref{fig:new-executor}.
%\begin{itemize}
%\item[\IconFolder] \directory{executor}
%  \begin{itemize}
%    \item[\IconFolder] \directory{bin}
%      \begin{itemize}
%	    \item[\IconFile] Makefile.am
%      \end{itemize}
%    \item[\IconFolder] \directory{config}
%      \begin{itemize}
%	\item[\IconText] README
%      \end{itemize}
%    \item[\IconFolder] \directory{etc}
%    \item[\IconFolder] \directory{src}
%      \begin{itemize}
%	\item[\IconFile] Makefile.am
%	\item[\IconFolder] \directory{controller}
%	\item[\IconFolder] \directory{database}
%	  \begin{itemize}
%	    \item[\IconFile] Makefile.am
%	  \end{itemize}
%	    \item[\IconFolder] \directory{interpreter}
%	  \begin{itemize}
%	    \item[\IconFile] Makefile.am
%	  \end{itemize}
%      \end{itemize}
%    \item[\IconText] AUTHORS
%    \item[\IconText] NEWS
%    \item[\IconText] README
%    \item[\IconText] ChangeLog
%    \item[\IconFile] acinclude.m4
%    \item[\IconFile] autoconf.sh
%    \item[\IconFile] configure.ac
%    \item[\IconFile] Makefile.am
%  \end{itemize}
%\end{itemize}

\begin{figure}[h!]
\centering
\scalebox{.7}{
\begin{tikzpicture}[
  grow via three points={one child at (0.5,-0.7) and
  two children at (0.5,-0.7) and (0.5,-1.4)},
  edge from parent path={(\tikzparentnode.south) |- (\tikzchildnode.west)}]
  \node[folder] {executor}
    child { node[folder] {bin}
        child { node {Makefile.am}}}	
    child [missing] {}	
    child { node[folder] {config}
        child { node {README}}}	
    child [missing] {}	
    child { node[folder] {etc}}
    child { node[folder] {src}
        child { node {Makefile.am}}
        child { node[folder] {controller}}
        child { node[folder] {database}
            child { node {Makefile.am}}}
        child [missing] {}
        child { node[folder] {interpreter}
            child { node {Makefile.am}}}
        child [missing] {}
    }	
    child [missing] {}
    child [missing] {}
    child [missing] {}	
    child [missing] {}
    child [missing] {}	
    child [missing] {}	
    child { node {AUTHORS}}
    child { node {NEWS}}
    child { node {README}}
    child { node {ChangeLog}}
    child { node {acinclude.m4}}
    child { node {autoconf.sh}}
    child { node {autotools.sh}}
    child { node {configure.ac}}
    child { node {Makefile.am}}
    child { node {bootstrap.sh}};
\end{tikzpicture}
}
\caption{New structure of the project executor.}
\label{fig:new-executor}
\end{figure}
The rest of this section describes how to configure \texttt{configure.ac} and the different \texttt{Makefile.am} files.

\subsubsection{The \texttt{configure.ac} File}

\begin{minipage}{.5\paperwidth}
\begin{mylisting}
\begin{Verbatim}[commandchars=\\\{\},fontsize=\scriptsize, numbersep=2pt]
#Initial information about the project
#Our program will be called ``executor'', the version number is 1.0 and bug-reports should go to zeid.kootbally@nist.gov.
AC_INIT([executor], [1.0], [zeid.kootbally@nist.gov])

#We have to add a line AM_INIT_AUTOMAKE directly below AC_INIT.
#This initializes the automake process.
#AM_INIT_AUTOMAKE(executor, 1.0)

#This macro simply checks for the existence of a file in the source directory.
#This is only a safety check but should also be used every time.
AC_CONFIG_SRCDIR([src/main.cc])

#This macro call checks if a compatible C++ compiler is available.
AC_PROG_CXX
#It is also very helpful to check for an install program which will install the resulting binary, documentation or datafiles
#into its corresponding directories.
AC_PROG_INSTALL

#This is required if any libraries are built in the package.
AC_PROG_RANLIB

#Use the C++ compiler for the following checks
AC_LANG([C++])

#Distribute additional compiler and linker flags among Makefiles
# --> set and change these variables instead of CXXFLAGS or LDFLAGS (for user only)

#AC_OUTPUT should substitute every occurrence of @AM_CXXFLAGS@ in input files with the values of \$AM_CXXFLAGS
AC_SUBST([AM_CXXFLAGS])
#AC_OUTPUT should substitute every occurrence of @AM_CPPFLAGS@ in input files with the values of \$AM_CPPFLAGS
AC_SUBST([AM_CPPFLAGS])
#AC_OUTPUT should substitute every occurrence of @AM_LDFLAGS@ in input files with the values of \$AM_LDFLAGS
AC_SUBST([AM_LDFLAGS])
#AC_OUTPUT should substitute every occurrence of @LIBS@ in input files with the values of \$LIBS
AC_SUBST([LIBS])

#Files to generate via autotools (prepare .am or .in source files)
AC_CONFIG_FILES([Makefile])
AC_CONFIG_FILES([src/Makefile])
AC_CONFIG_FILES([src/interpreter/Makefile])
AC_CONFIG_FILES([src/database/Makefile])
AC_CONFIG_FILES([bin/Makefile])

#Automake will look for various helper scripts, such as install-sh, in the directory named in this macro invocation.
#If AC_CONFIG_AUX_DIR is not given, the scripts are looked in their standard locations.
AC_CONFIG_AUX_DIR([config])

#Automake initialization (mandatory) including a check for automake API version >= 1.9
AM_INIT_AUTOMAKE([1.11])

#Checks for header files.
AC_HEADER_STDC
AC_CHECK_HEADERS([string])
AC_CHECK_HEADERS([iostream])

#Automake will generate rules to rebuild these headers
AC_CONFIG_HEADERS([config/config.h])

#Checks for typedefs, structures, and compiler characteristics.
AC_TYPE_SIZE_T

#Checks the platform.
ACX_HOST

#Generates files that are required for building the packages.
AC_OUTPUT
\end{Verbatim}
\end{mylisting}
\end{minipage}

\subsubsection{executor: \texttt{Makefile.am}}

\begin{minipage}{.5\paperwidth}
\begin{mylisting}
\begin{Verbatim}[commandchars=\\\{\},fontsize=\scriptsize, numbersep=2pt]
#The top level Makefile.am must tell Automake which subdirectories are to be built
SUBDIRS = src \textbackslash
            bin

# Directories needed to be distributed even though not covered in the automatic rules
EXTRA_DIST = etc
\end{Verbatim}
\end{mylisting}
\end{minipage}
\subsubsection{src: \texttt{Makefile.am}}

\begin{minipage}{.5\paperwidth}
\begin{mylisting}
\begin{Verbatim}[commandchars=\\\{\},fontsize=\scriptsize, numbersep=2pt]
#The current Makefile.am must tell Automake which subdirectories are to be built
SUBDIRS = database \textbackslash
interpreter
\end{Verbatim}
\end{mylisting}
\end{minipage}
\subsubsection{database: \texttt{Makefile.am}}
\begin{minipage}{.5\paperwidth}
\begin{mylisting}
\begin{Verbatim}[commandchars=\\\{\},fontsize=\scriptsize, numbersep=2pt]
#Additional include pathes necessary to compile the C++ library
AM_CXXFLAGS = -I\$(top\_srcdir)/src @AM\_CXXFLAGS@
AM_CPPFLAGS = -I\$(top\_srcdir)/src @AM\_CPPFLAGS@

#Building a non-installing library
noinst\_LIBRARIES = libdatabase.a

#Where to install the headers on the system
libdatabase\_adir = \$(top\_srcdir)/src/database

libdatabase\_a\_HEADERS = BoxVolume.h BoxyObject.h Connection.h DAO.h \textbackslash
DataThing.h EndEffector.h EndEffectorChangingStation.h EndEffectorHolder.h \textbackslash
GripperEffector.h Kit.h KitDesign.h KitTray.h \textbackslash
KittingWorkstation.h LargeBoxWithEmptyKitTrays.h LargeBoxWithKits.h LargeContainer.h \textbackslash
Part.h PartRefAndPose.h PartsBin.h PartsTray.h \textbackslash
PartsTrayWithParts.h PhysicalLocation.h Point.h PoseLocation.h \textbackslash
PoseLocationIn.h PoseLocationOn.h PoseOnlyLocation.h RelativeLocation.h \textbackslash
RelativeLocationIn.h RelativeLocationOn.h Robot.h ShapeDesign.h \textbackslash
SolidObject.h StockKeepingUnit.h VacuumEffector.h VacuumEffectorMultiCup.h \textbackslash
VacuumEffectorSingleCup.h Vector.h WorkTable.h

libdatabase\_a\_SOURCES = \$(libdatabase\_a\_HEADERS) \
BoxVolume.cpp BoxyObject.cpp Connection.cpp DAO.cpp \textbackslash
DataThing.cpp EndEffector.cpp EndEffectorChangingStation.cpp EndEffectorHolder.cpp \textbackslash
GripperEffector.cpp Kit.cpp KitDesign.cpp KitTray.cpp \textbackslash
KittingWorkstation.cpp LargeBoxWithEmptyKitTrays.cpp LargeBoxWithKits.cpp LargeContainer.cpp \textbackslash
Part.cpp PartRefAndPose.cpp PartsBin.cpp PartsTray.cpp \textbackslash
PartsTrayWithParts.cpp PhysicalLocation.cpp Point.cpp PoseLocation.cpp \textbackslash
PoseLocationIn.cpp PoseLocationOn.cpp PoseOnlyLocation.cpp RelativeLocation.cpp \textbackslash
RelativeLocationIn.cpp RelativeLocationOn.cpp Robot.cpp ShapeDesign.cpp \textbackslash
SolidObject.cpp StockKeepingUnit.cpp VacuumEffector.cpp VacuumEffectorMultiCup.cpp \textbackslash
VacuumEffectorSingleCup.cpp Vector.cpp WorkTable.cpp
\end{Verbatim}
\end{mylisting}
\end{minipage}

\subsubsection{interpreter: \texttt{Makefile.am}}
\begin{minipage}{.5\paperwidth}
\begin{mylisting}
\begin{Verbatim}[commandchars=\\\{\},fontsize=\scriptsize, numbersep=2pt]
#Definition of flags during compilation
#New flags need to be prepend to the other flags, not to append
#-I refers to extra directories needed to build the library
AM_CXXFLAGS = -I$(top_srcdir)/src @AM\_CXXFLAGS@

#Building a non-installing library
noinst_LIBRARIES = libinterpreter.a

#Where to install the headers on the system
libinterpreter_adir = $(top_srcdir)/src/interpreter

#The list of header files that belong to the library
libinterpreter_a_HEADERS = IniFile.h CanonicalRobotCommand.h KittingPDDLProblem.h KittingPlan.h Operator.h

#The list of source files that belong to the library
libinterpreter_a_SOURCES = $(libinterpreter_a_HEADERS) \textbackslash
	IniFile.cc CanonicalRobotCommand.cc KittingPDDLProblem.cc KittingPlan.cc Operator.c
\end{Verbatim}
\end{mylisting}
\end{minipage}
\subsubsection{bin: \texttt{Makefile.am}}

\begin{minipage}{.5\paperwidth}
\begin{mylisting}
\begin{Verbatim}[commandchars=\\\{\},fontsize=\scriptsize, numbersep=2pt]
#Definition of flags during compilation
#New flags need to be prepend to the other flags, not to append
#-I refers to directories needed to build the executor
AM_CXXFLAGS = -O3  -I$(top_srcdir)/src/ -I$(top_srcdir)/src/database -I$(top_srcdir)/src/interpreter @AM_CXXFLAGS@

#The executable file
bin_PROGRAMS = executor

#Main file needed to build the executable file
executor_SOURCES = \$(top\_srcdir)/src/main.cc

#Libraries needed to build the executable
#The most specific libraries are called before the most common ones
executor_LDADD = -L\$(top\_srcdir)/src/database -L\$(top\_srcdir)/src/interpreter  -linterpreter -ldatabase -lmysqlcppconn
\end{Verbatim}
\end{mylisting}
\end{minipage}

\subsubsection{The \texttt{bootstrap.sh} Script}
The \texttt{bootstrap.sh} file contains all the tools needed to generate extra files and to build the project. More information on the roles of each tool is given in section~\ref{section:components}.

\begin{minipage}{.5\paperwidth}
\begin{mylisting}
\begin{Verbatim}[commandchars=\\\{\},fontsize=\scriptsize, numbersep=2pt]
#!/bin/sh -x

aclocal && \textbackslash
autoheader && \textbackslash
automake --gnu --add-missing --copy && \textbackslash
autoconf && exit 0

exit 1
\end{Verbatim}
\end{mylisting}
\end{minipage}
