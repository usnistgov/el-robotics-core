\section{Planner}\label{section:planner}
This section describes the steps to install and run a planner on the PDDL domain and problem files in order to generate a plan. The planner uses a forward-chaining partial-order planning~\cite{Coles.ICAPS.2010}.

\subsection{Requirements}
The planner requires:
\begin{itemize}
\item \texttt{cmake}
\item The CBC mixed integer programming solver (\url{https://projects.coin-or.org/Cbc/})
\item \texttt{perl}, \texttt{bison} and \texttt{flex} to build the parser
\end{itemize}

These are packaged with most Linux distributions - on Ubuntu/Debian, the following should suffice:\\

\texttt{sudo apt-get install cmake coinor-libcbc-dev coinor-libclp-dev \textbackslash \\coinor-libcoinutils-dev bison flex}

\subsection{Download and Install CBC}
The CBC source code can be obtained using subversion:
\begin{itemize}
\item \texttt{svn co https://projects.coin-or.org/svn/Cbc/stable/2.7 coin-Cbc}: Issues the subversion command to obtain the source code.
\item \texttt{cd coin-Cbc}
\item \texttt{./configure -C}: Runs a configure script that generates the make file.
\item \texttt{make}: Builds the Cbc library and executable program.
\item \texttt{make test}: Builds and runs the Cbc unit test program.
\item \texttt{make install}: Installs libraries, executables and header files in directories \texttt{coin-Cbc/lib}, \texttt{coin-Cbc/bin} and \texttt{coin-Cbc/include}.
\end{itemize}

\subsection{Compile the Planner}
Before compiling the planner, the file \textit{compile/CMakeCache.txt} should be edited as follows. Note that \texttt{<path>} is the absolute path that leads to the \texttt{coin-Cbc} directory.
\begin{itemize}
\item \texttt{CBC\_INCLUDES:PATH = <path>/coin-Cbc/build/include}
\item \texttt{CGL\_INCLUDES:PATH = <path>/coin-Cbc/build/include}
\item \texttt{CLP\_INCLUDES:PATH = <path>/coin-Cbc/build/include}
\item \texttt{COINUTILS\_INCLUDES:PATH = <path>/coin-Cbc/build/include/coin}
\item \texttt{OSI\_INCLUDES:PATH = <path>/coin-Cbc/build/include}
\end{itemize}

To compile the planner, one should use:\\
\texttt{./build}


\subsection{Run the Planner}
To run the planner, the path to the PDDL domain and problem files should be identified. The format of the PDDL files must be \texttt{.pddl}. The following command run the planner on the PDDL files.\\
\texttt{./plan <domain> <problem> <solution>}\\

Where \texttt{<domain>} and \texttt{<problem>} are the PDDL domain and problem files, respectively.\texttt{<solution>} is the output file containing the plan.
