\section{State-Variable Representation}
In a SVR, each state is represented by a tuple of values of $n$ state variables $\lbrace x_1,\dots,x_n\rbrace$, and each action is represented by a partial function that maps this tuple into some other tuple of values of the $n$ state variables.\\ \\
To build the SVR, the group has taken a very systematic approach of identifying and modeling the concepts. Because the industrial robot field is so broad, the group decided to limit its efforts to a single type of operation, namely kitting. A scenario was developed that described, in detail, the types of operations that would be performed in kitting, the sequencing of steps, the parts and machines that were needed, constraints on the process such as pre- and post-conditions, etc. For this scenario, a set of concepts were extracted and defined. These concepts served as the initial requirements for the kitting SVR. The concepts were then modeling in our SVR, building off of the definitions and relationships that were identified in the scenario. A SVR relies on the elements of constant variable symbols, object variable symbols, state variable symbols, and planning operators. These are defined for the kitting domain in the rest of this section.


\subsection{Constant Variable Symbols}
For the kitting domain, there is a finite set of constant variable symbols that must be represented. In the SVR, constant variable symbols are partitioned into disjoint classes corresponding to the objects of the domain. The finite set of all constant variable symbols in the kitting domain is partitioned into the following sets:
%\begin{small}
\begin{itemize}
\item A set of \class{Part}: A \class{Part} is the basic item that will be used to fill a kit.

\item A set of \class{PartsTray}: \class{Parts} arrive at the workstation in \class{PartsTrays}. Each \class{Part} is at a known position in the \class{PartsTray}. Each \class{PartsTray} contains one type of \class{Part}.

\item A set of \class{KitTray}:  A \class{KitTray} can hold \class{Parts} in known positions.

\item A set of \class{Kit}: A \class{Kit} consists of a \class{KitTray} and, possibly, some \class{Parts}. A \class{Kit} is empty when it does not contain any \class{Part} and finished when it contains all the \class{Parts} that constitute a kit.

\item A set of \class{WorkTable}: A \class{WorkTable} is an area in the kitting workstation where \class{KitTrays} are placed to build \class{Kits}.

\item A set of \class{LargeBoxWithKits}: A \class{LargeBoxWithKits} contains only finished \class{Kits}.

\item A set of \class{LargeBoxWithEmptyKitTrays}: A \class{LargeBoxWithEmptyKitTrays} is a box that contains only empty \class{KitTrays}.

\item A set of \class{Robot} \{\const{robot\_1},\const{robot\_2},\ldots\}: A \class{Robot} in the kitting workstation is a robotic arm that can move objects in order to build \class{Kits}.

\item A set of \class{EndEffector}: \class{EndEffectors} are used in a kitting workstation to manipulate \class{Parts}, \class{PartsTrays}, \class{KitTrays}, and \class{Kits}. An \class{EndEffector} is attached to a \class{Robot} in order to grasp objects.

\item A set of \class{EndEffectorHolder}: An \class{EndEffectorHolder} is a storage unit that holds one type of \class{EndEffector}.

\item A set of \class{EndEffectorChangingStation}: An \class{EndEffectorChangingStation} is made up of \class{EndEffectorHolders}.
\end{itemize}
%\end {small}

\subsection{Object Variable Symbols}
Object variable symbols are typed variables which range over a class or the union of classes of constant variable symbols. Examples of object variable symbols are \const{r} $\in$ \class{Robots}, \const{kt} $\in$ \class{KitTrays}, etc.

\subsection{State Variable Symbols}
\label{subsect:State_Variable_Symbols}
A state variable symbol is defined as follows:\\
$\mathrm{x: A_1\times \dots\times A_i\times S\rightarrow}$ $\mathrm{B_1\cup\dots\cup B_j}$ $\cup$ \textit{bool} $\cup$ \textit{\{\}} $\cup$ \textit{numeric} ($i, j\geq 1$) is a function from the set of states ($\mathrm{S}$) and at least one set of constant variable symbols $\mathrm{A_1\times \dots\times A_i}$ into a set $\mathrm{B_1\cup\dots\cup B_j}$ $\cup$ \textit{bool} $\cup$ \textit{\{\}} $\cup$ \textit{numeric} where:

\begin{itemize}
\item $\mathrm{B_1\cup\dots\cup B_j}$ is a set of constant variable symbols
\item \textit{bool} is a boolean
\item \{\} is an empty set
\item \textit{numeric} is a numerical value
\end{itemize}

\noindent
The use of state variable symbols reduces the possibility of inconsistent states and generates a smaller state space. The following state variable symbols are used in the kitting domain.

\begin{itemize}
\item \stvar{endeffector-location}\\ \class{EndEffector}$\mathrm{\times S\rightarrow}$\class{Robot} $\cup$ \class{EndEffectorHolder}: designates the location of an \class{EndEffector} in the workstation. An \class{EndEffector} is either attached to a \class{Robot} or placed in an \class{EndEffectorHolder}.

\item \stvar{robot-with-endeffector}\\ \class{Robot}$\mathrm{\times S\rightarrow}$\class{EndEffector} $\cup$ \{\}: designates the \class{EndEffector} attached to a \class{Robot} if there is one attached, otherwise nothing.

\item \stvar{on-worktable}\\ \class{WorkTable}$\mathrm{\times S\rightarrow}$\class{Kit} $\cup$ \class{KitTray} $\cup$ \{\}: designates the object placed on the \class{WorkTable}, i.e., a \class{Kit}, a \class{KitTray}, or nothing.

\item \stvar{kit-location}\\ \class{Kit}$\mathrm{\times S\rightarrow}$\class{LargeBoxWithKits} $\cup$ \class{WorkTable} $\cup$ \class{Robot}: designates the different possible locations of a \class{Kit} in the workstation, i.e., in a \class{LargeBoxWithKits}, on the \class{WorkTable}, or being held by a \class{Robot}.

\item \stvar{kittray-location}\\ \class{KitTray}$\mathrm{\times S\rightarrow}$\class{LargeBoxWithEmptyKitTrays} $\cup$ \class{WorkTable} $\cup$ \class{Robot}: designates the different possible locations of a \class{KitTray} in the workstation, i.e., in a \class{LargeBoxWithEmptyKitTrays}, on a \class{WorkTable} or being held by a \class{Robot}.

\item \stvar{part-location}\\ \class{Part}$\mathrm{\times S\rightarrow}$\class{PartsTray} $\cup$ \class{Kit} $\cup$ \class{Robot}: designates the different possible locations of a \class{Part} in the workstation, i.e., in a \class{PartsTray}, in a \class{Kit}, or being held by a \class{Robot}.

\item \stvar{robot-holds}\\ \class{Robot}$\mathrm{\times S\rightarrow}$\class{KitTray} $\cup$ \class{Kit} $\cup$ \class{Part} $\cup$ \{\}: designates the object being held by a \class{Robot}, i.e., a \class{KitTray}, a \class{Kit}, a \class{Part}, or nothing. It is assumed that the \class{Robot} is already equipped with the appropriate \class{EndEffector}.

\item \stvar{lbwk-full}\\ \class{LargeBoxWithKits}$\mathrm{\times S\rightarrow}$ \textit{bool}: designates if a \class{LargeBoxWithKits} is full or not.

\item \stvar{lbwekt-empty}\\ \class{LargeBoxWithEmptyKitTrays}$\mathrm{\times S\rightarrow}$ \textit{bool}: designates if a \class{LargeBoxWithEmptyKitTrays} is empty or not.

\item \stvar{partstray-empty}\\ \class{PartsTray}$\mathrm{\times S\rightarrow}$ \textit{bool}: designates if a \class{PartsTray} is empty or not.

\item \stvar{endffector-type}\\ \class{EndEffector}$\mathrm{\times S \rightarrow}$\class{KitTray} $\cup$ \class{Kit} $\cup$ \class{Part}: designates the type of object an \class{EndEffector} can hold, i.e., a \class{KitTray}, a \class{Kit}, or a \class{Part}.

\item \stvar{endeffectorholder-holds-endeffector}\\ \class{EndEffectorHolder}$\mathrm{\times S \rightarrow}$\class{EndEffector} $\cup$ \{\}: designates wether an \class{EndEffectorHolder} is holding an \class{EndEffector} or nothing.

\item \stvar{endeffectorholder-location}\\ \class{EndEffectorHolder}$\mathrm{\times S \rightarrow}$\class{EndEffectorChangingStation}: designates the \class{EndEffectorChangingStation} where the \class{EndEffectorHolder} is located.

\item \stvar{endeffectorchangingstation-has-endeffectorholder}\\ \class{EndEffectorChangingStation}$\mathrm{\times S \rightarrow}$\class{EndEffectorHolder}: designates the \class{EndEffectorHolder} the \class{EndEffectorChangingStation} contains.

\item \stvar{found-part}\\ \class{PartsTray}$\mathrm{\times S \rightarrow}$\class{Part} $\cup$ \{\}: designates wether a \class{Part} is found in a \class{PartsTray} or not.

\item \stvar{origin-part}\\ \class{Part}$\mathrm{\times S \rightarrow}$\class{PartsTray}: designates the \class{PartsTray} where the \class{Part} is found.

\item \stvar{quantity-parts-in-partstray}\\ \class{PartsTray}$\mathrm{\times S \rightarrow}$\textit{numeric}: designates the number of parts that \class{PartsTray} contains.

\item \stvar{quantity-parts-in-kit}\\ \class{Kit}$\mathrm{\times}$\class{PartsTray}$\mathrm{\times S \rightarrow}$\textit{numeric}: designates the number of parts from \class{PartsTray} that \class{Kit} contains.

\item \stvar{capacity-parts-in-kit}\\ \class{Kit}$\mathrm{\times}$\class{PartsTray}$\mathrm{\times S \rightarrow}$\textit{numeric}: designates the number of parts from \class{PartsTray} that \class{Kit} \textbf{can} contain.

\end{itemize}


%\subsection{Rigid Relations}
%\label{subsubsect:Rigid_Relation}
%\stvar{efftype} and \stvar{effhold-eff} are rigid relations since their values do not vary from one state to another. In each state, a given \class{EndEffector} will %always hold the same type of object and a given \class{EndEffectorHolder} will always hold the same \class{EndEffectors}.

\subsection{Predicates and Functions}
In PDDL, predicates are used to encode Boolean state variables, while functions are used to model updates of numerical values~\cite{FOXandLong.2003}. This section describes the predicates and functions derived from the state variables described in section~\ref{subsect:State_Variable_Symbols}. We recall the following definition of a state variable ((section~\ref{subsect:State_Variable_Symbols})) $\mathrm{x: A_1\times \dots\times A_i\times S\rightarrow B_1\cup\dots\cup B_j}$ ($i, j\geq 1$) that is used to convert state variables into predicates as follows:

\begin{itemize}
 \item $\mathrm{A_1\times \dots\times A_i\times S\rightarrow B_1\cup\dots\cup B_j}$ ($i, j\geq 1$)
  \begin{itemize}
  \item \stvar{predicate\_1}($\mathcal{A,B}$)
  \item \ldots
  \item \stvar{predicate\_n}($\mathcal{A,B}$)
  \end{itemize}
\end{itemize}

Where $\mathcal{A} \in \mathrm{\{A_1,\ldots,A_i\}}$ and $\mathcal{B} \in \mathrm{\{B_1,\ldots,B_i\}}$ ($i, j\geq 1$)

\subsubsection{Predicates}



The state variables in our current kitting domain contains the following predicates.
\begin{itemize}
 \item \stvar{endeffector-location}
  \begin{itemize}
  \item \stvar{endeffector-location-robot}(\class{EndEffector},\class{Robot}) ;TRUE iff \class{EndEffector} is attached to \class{Robot}
  \item \stvar{endeffector-location-endeffectorholder}(\class{EndEffector},\class{EndEffectorHolder}) ;TRUE iff \class{EndEffector} is in \class{EndEffectorHolder}
  \end{itemize}

 \item \stvar{robot-with-endeffector}
  \begin{itemize}
  \item \stvar{robot-with-endeffector}(\class{Robot},\class{EndEffector}) ;TRUE iff \class{Robot} is equipped with \class{EndEffector}
  \item \stvar{robot-with-no-endeffector}(\class{Robot}) ;TRUE iff \class{Robot} is not equipped with any \class{EndEffector}	
  \end{itemize}

 \item \stvar{on-worktable}
  \begin{itemize}
  \item \stvar{on-worktable-kit}(\class{WorkTable},\class{Kit}) ;TRUE iff \class{Kit} is on the \class{WorkTable}
  \item \stvar{on-worktable-kittray}(\class{WorkTable},\class{KitTray}) ;TRUE iff \class{KitTray} is on the \class{WorkTable}
  \item \stvar{worktable-empty}(\class{WorkTable}) ;TRUE iff there is nothing on the \class{WorkTable}
  \end{itemize}

 \item \stvar{kit-location}
  \begin{itemize}
  \item \stvar{kit-location-lbwk}(\class{Kit},\class{LargeBoxWithKits}) ;TRUE iff \class{Kit} is in the \class{LargeBoxWithKits}
  \item \stvar{kit-location-worktable}(\class{Kit},\class{WorkTable}) ;TRUE iff \class{Kit} is on the \class{WorkTable}
  \item \stvar{kit-location-robot}(\class{Kit},\class{Robot}) ;TRUE iff \class{Kit} is being held by the \class{Robot}	
  \end{itemize}

 \item \stvar{kittray-location}
  \begin{itemize}
  \item \stvar{kittray-location-lbwekt}(\class{KitTray},\class{LargeBoxWithEmptyKitTrays}) ;TRUE iff \class{KitTray} is in the \class{LargeBoxWithEmptyKitTrays}	
  \item \stvar{kittray-location-robot}(\class{KitTray},\class{Robot}) ;TRUE iff \class{KitTray} is being held by the \class{Robot}
  \item \stvar{kittray-location-worktable}(\class{KitTray},\class{WorkTable}) ;TRUE iff \class{KitTray} is on the \class{WorkTable}
  \end{itemize}

 \item \stvar{part-location}
  \begin{itemize}
    \item \stvar{part-location-partstray}(\class{Part},\class{PartsTray}) ;TRUE iff \class{Part} is in the \class{PartsTray}	
    \item \stvar{part-location-kit}(\class{Part},\class{Kit}) ;TRUE iff \class{Part} is in the \class{Kit}
    \item \stvar{part-location-robot}(\class{Part},\class{Robot}) ;TRUE iff \class{Part} is being held by the \class{Robot}
  \end{itemize}

 \item \stvar{robot-holds}
  \begin{itemize}
    \item \stvar{robot-holds-kittray}(\class{Robot},\class{KitTray}) ;TRUE iff \class{Robot} is holding a \class{KitTray}	
    \item \stvar{robot-holds-kit}(\class{Robot},\class{Kit}) ;TRUE iff \class{Robot} is holding a \class{Kit}
    \item \stvar{robot-holds-part}(\class{Robot},\class{Part}) ;TRUE iff \class{Robot} is holding a \class{Part}
    \item \stvar{robot-empty}(\class{Robot}) ;TRUE iff \class{Robot} is not holding anything
  \end{itemize}

 \item \stvar{lbwk-full}
  \begin{itemize}
    \item \stvar{lbwk-not-full}(\class{LargeBoxWithKits}) ;TRUE iff \class{LargeBoxWithKits} is not full	
  \end{itemize}

 \item \stvar{lbwekt-empty}
  \begin{itemize}
    \item \stvar{lbwekt-not-empty}(\class{LargeBoxWithEmptyKitTrays}) ;TRUE iff \class{LargeBoxWithEmptyKitTrays} is not empty
  \end{itemize}

 \item \stvar{partstray-empty}
  \begin{itemize}
    \item \stvar{partstray-not-empty}(\class{PartsTray}) ;TRUE iff \class{PartsTray} is not empty
  \end{itemize}

 \item \stvar{endeffector-type}
  \begin{itemize}
    \item \stvar{endeffector-type-kittray}(\class{EndEffector},\class{KitTray}) ;TRUE iff \class{EndEffector} is capable of holding a \class{KitTray}	
    \item \stvar{endeffector-type-kit}(\class{EndEffector},\class{Kit}) ;TRUE iff \class{EndEffector} is capable of holding a \class{Kit}
    \item \stvar{endeffector-type-part}(\class{EndEffector},\class{Part}) ;TRUE iff \class{EndEffector} is capable of holding a \class{Part}
  \end{itemize}

 \item \stvar{endeffectorholder-holds-endeffector}
  \begin{itemize}
    \item \stvar{endeffectorholder-holds-endeffector}(\class{EndEffectorHolder},\class{EndEffector}) ;TRUE iff \class{EndEffectorHolder} is holding \class{EndEffector}
    \item \stvar{endeffectorholder-empty}(\class{EndEffectorHolder}) ;TRUE iff \class{EndEffectorHolder} is empty (not holding an \class{EndEffector})
  \end{itemize}


 \item \stvar{endeffectorholder-location}
  \begin{itemize}
    \item \stvar{endeffectorholder-location}(\class{EndEffectorHolder},\class{EndEffectorChangingStation}) ;TRUE iff \class{EndEffectorHolder} is in \class{EndEffectorChangingStation}
  \end{itemize}

\item \stvar{endeffectorchangingstation-has-endeffectorholder}
  \begin{itemize}
    \item \stvar{endeffectorchangingstation-has-endeffectorholder}(\class{EndEffectorChangingStation},\class{EndEffectorHolder}) ;TRUE iff \class{EndEffectorChangingStation} contains \class{EndEffectorHolder}
  \end{itemize}

\item \stvar{found-part}
  \begin{itemize}
    \item \stvar{found-part}(\class{Part},\class{PartsTray}) ;TRUE iff \class{Part} is found in \class{PartsTray}
  \end{itemize}

\item \stvar{origin-part}
  \begin{itemize}
    \item \stvar{origin-part}(\class{Part},\class{PartsTray}) ;TRUE iff \class{Part} is from \class{PartsTray}.
  \end{itemize}
\end{itemize}

\subsubsection{Functions}
In a planning model, numeric fluents represent function symbols that can take an infinite
set of values. Introducing functions into planning not only makes it possible to deal with numerical
values in a more general way than allowed for by a purely relational language but makes it possible to model operators in a more compact and sometimes also more natural way. The state variables in our current kitting domain contains the following functions.

\begin{itemize}
 \item \stvar{quantity-parts-in-partstray}
  \begin{itemize}
   \item \stvar{quantity-parts-in-partstray}(\class{PartsTray}) ;Quantity of parts in \class{PartsTray}
  \end{itemize}

 \item \stvar{quantity-parts-in-kit}
  \begin{itemize}
   \item \stvar{quantity-parts-in-kit}(\class{Kit},\class{PartsTray}) ;Quantity of parts from \class{PartsTray} that is in \class{Kit}
  \end{itemize}

     \item \stvar{capacity-parts-in-kit}
  \begin{itemize}
  \item \stvar{capacity-parts-in-kit}(\class{Kit},\class{PartsTray}) ;Quantity of parts from \class{PartsTray} that \class{Kit} \textbf{can} contain
  \end{itemize}
\end{itemize}


\subsection{Planning Operators and Actions}
\label{subsect:Planning_Operators}
The planning operators presented in this section are expressed in classical representation instead of state variable representation. In classical representation, states are represented as sets of logical atoms (predicates) that are true or false within some interpretation. Actions are represented by planning operators that change the truth values of these atoms.


\subsubsection{Planning Operators}
 In classical planning, a planning operator~\cite{NAU.2004} is a triple o=(\op{name}(o), \textit{preconditions}(o), \textit{effects}(o)) whose elements are as follows:
\begin{itemize}
\item \op{name}(o) is a syntactic expression of the form $n(x_1,\dots,x_k)$, where $n$ is a symbol
called an operator symbol, $x_1,\dots,x_k$ are all of the object variable symbols that
appear anywhere in \textit{o}, and $n$ is unique (i.e., no two operators can have the
same operator symbol).
\item \textit{preconditions}(o) and \textit{effects}(o) are sets of literals (i.e., atoms and negations of atoms). Literals that are true in \textit{preconditions}(o) but false in \textit{effects}(o) are removed by using negations of the appropriate atoms.
\end{itemize}

Our kitting domain is composed of ten operators which are defined below.


\begin{enumerate}

%%%%%%%%%%%%%%%%%%%%%%%%%%%%%%%%%%%%% take-kittray
\item \op{take-kittray}(\const{robot},\const{kittray},\const{lbwekt},\const{endeffector},\const{worktable}): The \class{Robot} \const{robot} equipped with the \class{EndEffector} \const{endeffector} picks up the \class{KitTray} \const{kittray} from the \class{LargeBoxWithEmptyKitTrays} \const{lbwekt}.


\begin{tabular}{ l|l }
  \textit{preconditions} & \textit{effects} \\
  \hline
  \stvarsmall{robot-empty}(\constsmall{robot})&$\neg$\stvarsmall{robot-empty}(\constsmall{robot})\\
  \stvarsmall{kittray-location-lbwekt}(\constsmall{kittray},\constsmall{lbwekt})&$\neg$\stvarsmall{kittray-location-lbwekt}(\constsmall{kittray},\constsmall{lbwekt}) \\
  \stvarsmall{lbwekt-not-empty}(\const{lbwekt})&\stvarsmall{kittray-location-robot}(\constsmall{kittray},\constsmall{robot})\\
  \stvarsmall{robot-with-endeffector}(\const{robot},\constsmall{endeffector})&\stvarsmall{robot-holds-kittray}(\constsmall{robot},\constsmall{kittray}) \\

  \stvarsmall{endeffector-location-robot}(\constsmall{endeffector},\constsmall{robot})&\\
  \stvarsmall{worktable-empty}(\constsmall{worktable})& \\
  \stvarsmall{endeffector-type-kittray}(\constsmall{endeffector},\constsmall{kittray})&
\end{tabular}


\begin{itemize}
 \item \textit{preconditions}
 \begin{itemize}
 \item \stvarsmall{robot-empty}(\constsmall{robot}): \constsmall{robot} does not hold anything.
 \item \stvarsmall{kittray-location-lbwekt}(\constsmall{kittray},\constsmall{lbwekt}): \constsmall{kittray} is in \constsmall{lbwekt}.
 \item \stvarsmall{lbwekt-not-empty}(\const{lbwekt}): \constsmall{lbwekt} is not empty (contains at least one kit tray).
 \item \stvarsmall{robot-with-endeffector}(\const{robot},\constsmall{endeffector}): \constsmall{robot} is equipped with \constsmall{endeffector}.
 \item \stvarsmall{endeffector-location-robot}(\constsmall{endeffector},\constsmall{robot}): The end effector is on the robot's arm.
 \item \stvarsmall{worktable-empty}(\constsmall{worktable}): After picking up an empty kit tray from a large box of empty kit trays, the robot would normally place the kit tray on the work table. To put a kit tray on the work table, it is necessary that there is nothing on top of the work table. If the robot is allowed to pick up the kit tray while there is another object on the work table, the planning system may not be able to find a solution when it comes to put the kit tray on the work table. Therefore, it is necessary to check that the top of \constsmall{worktable} is clear even before the robot picks up a kit tray from the large box of empty kit trays.
 \item \stvarsmall{endeffector-type-kittray}(\constsmall{endeffector},\constsmall{kittray}): \constsmall{endeffector} in the robot's arm must be capable of handling \constsmall{kittray}.
 \end{itemize}
\item \textit{effects}
 \begin{itemize}
 \item $\neg$\stvarsmall{robot-empty}(\constsmall{robot}): \constsmall{robot}'s end effector is no longer empty since it contains \constsmall{kittray}.
 \item $\neg$\stvarsmall{kittray-location-lbwekt}(\constsmall{kittray},\constsmall{lbwekt}): \constsmall{kittray} is no longer in \constsmall{lbwekt} since it is in the robot's end effector.
 \item \stvarsmall{kit-tray-location}(\constsmall{kittray},\constsmall{robot}): \constsmall{kittray} is in \constsmall{robot}'s end effector.
 \item \stvarsmall{robot-holds-kittray}(\constsmall{robot},\constsmall{kittray}): \constsmall{robot} is holding \constsmall{kittray}.
 \end{itemize}

\end{itemize}





%%%%%%%%%%%%%%%%%%%%%%%%%%%%%%%%%%%%% put-kittray

\item \op{put-kittray}(\const{robot},\const{kittray},\const{worktable}): The \class{Robot} \const{robot} puts the \class{KitTray} \const{kittray} on the \class{WorkTable} \const{worktable}.

\begin{tabular}{ l|l }
  \textit{preconditions} & \textit{effects} \\
  \hline
  \stvarsmall{kittray-location-robot}(\constsmall{kittray},\constsmall{robot})
  &$\neg$\stvarsmall{kittray-location-robot}(\constsmall{kittray},\constsmall{robot})\\
  \stvarsmall{robot-holds-kittray}(\constsmall{robot},\constsmall{kittray})
  &$\neg$\stvarsmall{robot-holds-kittray}(\constsmall{robot},\constsmall{kittray})\\
  \stvarsmall{worktable-empty}(\constsmall{worktable})
  &$\neg$\stvarsmall{worktable-empty}(\constsmall{worktable})\\
  &\stvarsmall{kittray-location-worktable}(\constsmall{kittray},\constsmall{worktable})\\
  &\stvarsmall{robot-empty}(\constsmall{robot})\\
  &\stvarsmall{on-worktable-kittray}(\constsmall{worktable},\constsmall{kittray})\\
\end{tabular}


\begin{itemize}
 \item \textit{preconditions}
 \begin{itemize}
 \item \stvarsmall{kittray-location-robot}(\constsmall{kittray},\constsmall{robot}): \constsmall{kittray} is in \constsmall{robot}'s end effector.
 \item \stvarsmall{robot-holds-kittray}(\constsmall{robot},\constsmall{kittray}): \constsmall{robot} holds \constsmall{kittray}.
 \item \stvarsmall{worktable-empty}(\constsmall{worktable}): There is nothing on \constsmall{worktable}.
 \end{itemize}
\item \textit{effects}
 \begin{itemize}
 \item $\neg$\stvarsmall{kittray-location-robot}(\constsmall{kittray},\constsmall{robot}): \constsmall{kittray} is no longer it \constsmall{robot}'s end effector since it is placed on \constsmall{worktable}.
 \item $\neg$\stvarsmall{robot-holds-kittray}(\constsmall{robot},\constsmall{kittray}): \constsmall{robot} is not holding \constsmall{kittray} anymore.
 \item $\neg$\stvarsmall{worktable-empty}(\constsmall{worktable}): \constsmall{worktable} is not empty anymore since there is something on top of it.
 \item \stvarsmall{kittray-location-worktable}(\constsmall{kittray},\constsmall{worktable}): \constsmall{kittray} is on \constsmall{worktable}.
 \item \stvarsmall{robot-empty}(\constsmall{robot}): \constsmall{robot} is not holding anything.
 \item \stvarsmall{on-worktable-kittray}(\constsmall{worktable},\constsmall{kittray}): \constsmall{worktable} has \constsmall{kittray} on top of it.
 \end{itemize}

\end{itemize}


%%%%%%%%%%%%%%%%%%%%%%%%%%%%%%%%%%%%% take-kit

\item \op{take-kit}(\const{robot},\const{kit},\const{worktable},\const{endeffector}): The \class{Robot} \const{robot} equipped with the \class{EndEffector} \const{endeffector} picks up the \class{Kit} \const{kit} from the \class{WorkTable} \const{worktable}.

\begin{tabular}{ l|l }
  \textit{preconditions} & \textit{effects} \\
  \hline
  \stvarsmall{kit-location-worktable}(\constsmall{kit},\constsmall{worktable})
  &$\neg$\stvarsmall{kit-location-worktable}(\constsmall{kit},\constsmall{worktable})\\
  \stvarsmall{robot-empty}(\constsmall{robot})
  &$\neg$\stvarsmall{robot-empty}(\constsmall{robot})\\
  \stvarsmall{on-worktable-kit}(\constsmall{worktable},\constsmall{kit})
  &$\neg$\stvarsmall{on-worktable-kit}(\constsmall{worktable},\constsmall{kit})\\
  \stvarsmall{robot-with-endeffector}(\constsmall{robot},\constsmall{endeffector})
  &\stvarsmall{kit-location-robot}(\constsmall{kit},\constsmall{robot})\\
  \stvarsmall{endeffector-type-kit}(\constsmall{endeffector},\constsmall{kit})
  &\stvarsmall{robot-holds-kit}(\constsmall{robot},\constsmall{kit})\\
  &\stvarsmall{worktable-empty}(\constsmall{worktable})
\end{tabular}


\begin{itemize}
 \item \textit{preconditions}
 \begin{itemize}
 \item \stvarsmall{kit-location-worktable}(\constsmall{kit},\constsmall{worktable}): \constsmall{kit} is located on  \constsmall{worktable}.
 \item \stvarsmall{robot-empty}(\constsmall{robot}): \constsmall{robot} is not holding any object.
 \item \stvarsmall{on-worktable-kit}(\constsmall{worktable},\constsmall{kit}): \constsmall{worktable} has \constsmall{kit} on top of it.
 \item \stvarsmall{robot-with-endeffector}(\constsmall{robot},\constsmall{endeffector}): \constsmall{robot} is equipped with \constsmall{endeffector}.
 \item \stvarsmall{endeffector-type-kit}(\constsmall{endeffector},\constsmall{kit}): The type of \constsmall{endeffector} is capable of handling \constsmall{kit}.
 \end{itemize}
\item \textit{effects}
 \begin{itemize}
 \item $\neg$\stvarsmall{kit-location-worktable}(\constsmall{kit},\constsmall{worktable}): \constsmall{kit} is not on \constsmall{worktable}.
 \item $\neg$\stvarsmall{robot-empty}(\constsmall{robot}): \constsmall{robot} is holding an object (\constsmall{kit}).
 \item $\neg$\stvarsmall{on-worktable-kit}(\constsmall{worktable},\constsmall{kit}): \constsmall{worktable} does not have \constsmall{kit} on top of it.
 \item \stvarsmall{kit-location-robot}(\constsmall{kit},\constsmall{robot}): \constsmall{kit} is being held by \constsmall{robot}.
 \item \stvarsmall{robot-holds-kit}(\constsmall{robot},\constsmall{kit}): \constsmall{robot} is holding \constsmall{kit}.
 \item \stvarsmall{worktable-empty}(\constsmall{worktable}): \constsmall{worktable} does not have any object on top of it.
 \end{itemize}

\end{itemize}



%%%%%%%%%%%%%%%%%%%%%%%%%%%%%%%%%%%%% put-kit

\item \op{put-kit}(\const{robot},\const{kit},\const{lbwk}): The \class{Robot} \const{robot} puts down the \class{Kit} \const{kit} in the \class{LargeBoxWithKits} \const{lbwk}.

\begin{tabular}{ l|l }
  \textit{preconditions} & \textit{effects} \\
  \hline
  \stvarsmall{kit-location-robot}(\constsmall{kit},\constsmall{robot})
  &$\neg$\stvarsmall{kit-location-robot}(\constsmall{kit},\constsmall{robot})\\
  \stvarsmall{robot-holds-kit}(\constsmall{robot},\constsmall{kit})
  &$\neg$\stvarsmall{robot-holds-kit}(\constsmall{robot},\constsmall{kit})\\
  \stvarsmall{lbwk-not-full}(\constsmall{lbwk})
  &\stvarsmall{kit-location-lbwk}(\constsmall{kit},\const{lbwk})\\
  &\stvarsmall{robot-empty}(\constsmall{robot})
\end{tabular}


\begin{itemize}
 \item \textit{preconditions}
 \begin{itemize}
 \item \stvarsmall{kit-location-robot}(\constsmall{kit},\constsmall{robot}): \constsmall{kit} is held by \constsmall{robot}.
 \item \stvarsmall{robot-holds-kit}(\constsmall{robot},\constsmall{kit}): \constsmall{robot} is holding \constsmall{kit}.
 \item \stvarsmall{lbwk-not-full}(\constsmall{lbwk}): \constsmall{lbwk} should not be full so it can contain \constsmall{kit}.
 \end{itemize}
\item \textit{effects}
 \begin{itemize}
 \item $\neg$\stvarsmall{kit-location-robot}(\constsmall{kit},\constsmall{robot}): \constsmall{kit} is not being held by \constsmall{robot}.
 \item $\neg$\stvarsmall{robot-holds-kit}(\constsmall{robot},\constsmall{kit}): \constsmall{robot} is not holding \constsmall{kit}.
 \item \stvarsmall{kit-location-lbwk}(\constsmall{kit},\const{lbwk}): \constsmall{kit} has been placed in \constsmall{lbwk}.
 \item \stvarsmall{robot-empty}(\constsmall{robot}): \constsmall{robot} is not holding anything (not holding \constsmall{kit} anymore).
 \end{itemize}
 \end{itemize}

 %%%%%%%%%%%%%%%%%%%%%%%%%%%%%%%%%%%%% look-for-part

\item \op{look-for-part}(\const{robot},\const{part},\const{partstray},\const{kit},\const{worktable},\const{endeffector}): A sensor looks for the \class{Part} \const{part} in the \class{PartTray} \const{partstray}.

\begin{tabular}{ l|l }
  \textit{preconditions} & \textit{effects} \\
  \hline
  \stvarsmall{part-not-searched}
  & $\neg$\stvarsmall{part-not-searched}\\
  \stvarsmall{robot-empty}(\constsmall{robot})
  &\stvarsmall{found-part}(\constsmall{partstray})\\
  \stvarsmall{robot-with-endeffector}(\constsmall{robot},\constsmall{endeffector})
  &\\
  \stvarsmall{on-worktable-kit}(\constsmall{worktable},\constsmall{kit})
  &\\
  \stvarsmall{endeffector-location-robot}(\constsmall{endeffector},\constsmall{robot})
  &\\
  \stvarsmall{part-location-partstray}(\constsmall{part},\constsmall{partstray})
  &\\
  \stvarsmall{kit-location-worktable}(\constsmall{kit},\constsmall{worktable})
  &\\
  \stvarsmall{endeffector-type-part}(\constsmall{endeffector},\constsmall{part})
  &\\
  \stvarsmall{partstray-not-empty}(\constsmall{partstray}) &
\end{tabular}


\begin{itemize}
 \item \textit{preconditions}
 \begin{itemize}
 \item \stvarsmall{part-not-searched}: This flag is set to true in the initial state in the problem file (see section~\ref{S:PDDL-problem}) and means that a part has not been searched yet.
 \item \stvarsmall{robot-empty}(\constsmall{robot}): \constsmall{robot} should not be holding anything. We want the operator \op{look-for-part} to be directly followed by the operator \op{take-part} to simulate a sensor identifying a part before being picked up by a robot. It is necessary to check that \constsmall{robot}'s end effector is empty to prepare for the execution of the operator \op{take-part}.
 \item \stvarsmall{robot-with-endeffector}(\constsmall{robot},\constsmall{endeffector}): \constsmall{robot} is equipped with \constsmall{endeffector}. Again, since we want the operator following \op{look-for-part} to be \op{take-part}, we want to make sure that \constsmall{robot} is already equipped with \constsmall{endeffector}.
 \item \stvarsmall{on-worktable-kit}(\constsmall{worktable},\constsmall{kit}): \constsmall{worktable} has \constsmall{kit} on top of it.
 \item \stvarsmall{endeffector-location-robot}(\constsmall{endeffector},\constsmall{robot}): \constsmall{endeffector} is on \constsmall{robot} so it is ready for the operator \op{take-part}.
 \item \stvarsmall{part-location-partstray}(\constsmall{part},\constsmall{partstray}): \constsmall{part} is in \constsmall{partstray}.
 \item \stvarsmall{kit-location-worktable}(\constsmall{kit},\constsmall{worktable}): \constsmall{kit} is on \constsmall{worktable}.
 \item \stvarsmall{endeffector-type-part}(\constsmall{endeffector},\constsmall{part}): \constsmall{endeffector} can handle \constsmall{part}.
 \item \stvarsmall{partstray-not-empty}(\constsmall{partstray}): \constsmall{partstray} contains at least one part.
 \end{itemize}
\item \textit{effects}
 \begin{itemize}
 \item $\neg$\stvarsmall{part-not-searched}: This flag is set to true so that the operator \op{look-for-part} can be called again to look for another part in the workstation.
 \item \stvarsmall{found-part}(\constsmall{partstray}): A part from \constsmall{partstray} has been found.
 \end{itemize}
 \end{itemize}

%%%%%%%%%%%%%%%%%%%%%%%%%%%%%%%%%%%%% take-part

\item \op{take-part}(\const{robot},\const{part},\const{partstray},\const{endeffector},\const{worktable},\const{kit}): The \class{Robot} \const{robot} uses the \class{EndEffector} \const{endeffector} to pick up the \class{Part} \const{part} from the \class{PartTray} \const{partstray}.

\begin{tabular}{ l|l }
  \textit{preconditions} & \textit{effects} \\
  \hline
  \stvarsmall{part-location-partstray}(\constsmall{part},\constsmall{partstray})
  &$\neg$\stvarsmall{part-location-partstray}(\constsmall{part},\constsmall{partstray})\\
  \stvarsmall{robot-empty}(\constsmall{robot})
  & $\neg$\stvarsmall{robot-empty}(\constsmall{robot})\\
  \stvarsmall{endeffector-location-robot}(\constsmall{endeffector},\constsmall{robot})
  & \stvarsmall{part-location-robot}(\constsmall{part},\constsmall{robot})\\
  \stvarsmall{robot-with-endeffector}(\constsmall{robot},\constsmall{endeffector})
  & \stvarsmall{robot-holds-part}(\constsmall{robot},\constsmall{part})\\
  \stvarsmall{on-worktable-kit}(\constsmall{worktable},\constsmall{kit})
  &  \stvarsmall{decrease}\ \stvarsmall{quantity-parts-in-partstray}(\constsmall{partstray})\\
  \stvarsmall{kit-location-worktable}(\constsmall{kit},\const{worktable})
  &  \\
  \stvarsmall{endeffector-type-part}(\constsmall{endeffector},\const{part})
  &  \\
  \stvarsmall{partstray-not-empty}(\constsmall{partstray})
  &\\
  \stvarsmall{found-part}(\constsmall{part},\constsmall{partstray}) &
\end{tabular}


\begin{itemize}
 \item \textit{preconditions}
 \begin{itemize}
 \item \stvarsmall{part-location-partstray}(\constsmall{part},\constsmall{partstray}): \constsmall{part} to be picked up is in \constsmall{partstray}.
 \item \stvarsmall{robot-empty}(\constsmall{robot}): \constsmall{robot} is not holding any object.
 \item \stvarsmall{endeffector-location-robot}(\constsmall{endeffector},\constsmall{robot}): \constsmall{endeffector} is on \constsmall{robot}.
 \item \stvarsmall{robot-with-endeffector}(\constsmall{robot},\constsmall{endeffector}): \constsmall{robot} is equipped with \constsmall{endeffector}.
 \item \stvarsmall{on-worktable-kit}(\constsmall{worktable},\constsmall{kit}): \constsmall{worktable} has \constsmall{kit} on top of it.
 \item \stvarsmall{kit-location-worktable}(\constsmall{kit},\const{worktable}): \constsmall{kit} is on \constsmall{worktable}. Once a part is picked up by the robot, the next logical action would be to put the part in the kit. For this to happen, the kit needs to be already on the work table so it can hold the part.
 \item \stvarsmall{endeffector-type-part}(\constsmall{endeffector},\const{part}): \constsmall{endeffector} is the type for \constsmall{part} handling.
 \item \stvarsmall{partstray-not-empty}(\constsmall{partstray}): \constsmall{partstray} is not empty and contains at least one part.
 \item \stvarsmall{found-part}(\constsmall{part},\constsmall{partstray}): \constsmall{part} has been found in \constsmall{partstray}. \stvarsmall{found-part} is set to true in the \textit{effects} of the operator \op{look-for-part}.
 \end{itemize}
\item \textit{effects}
 \begin{itemize}
 \item $\neg$\stvarsmall{part-location-partstray}(\constsmall{part},\constsmall{partstray}): \constsmall{part} is not in \constsmall{partstray} anymore since it was picked up by \constsmall{robot}.
 \item $\neg$\stvarsmall{robot-empty}(\constsmall{robot}): \constsmall{robot} is now holding \constsmall{part} and is not empty anymore.
 \item  \stvarsmall{part-location-robot}(\constsmall{part},\constsmall{robot}): \constsmall{part} is held by \constsmall{robot}.
 \item \stvarsmall{robot-holds-part}(\constsmall{robot},\constsmall{part}): \constsmall{robot} is holding \constsmall{part}.
 \item \stvarsmall{decrease}\ \stvarsmall{quantity-parts-in-partstray}(\constsmall{partstray}): After picking up a part from \constsmall{partstray} the number of parts in \constsmall{partstray} is decreased by one. This is expressed with the \stvarsmall{decrease} function.
 \end{itemize}
 \end{itemize}
%%%%%%%%%%%%%%%%%%%%%%%%%%%%%%%%%%%%%

\item \op{put-part}(\const{robot},\const{part},\const{kit},\const{worktable},\const{partstray}): The \class{Robot} \const{robot} puts the \class{Part} \const{part} in the \class{Kit} \const{kit}.

\begin{tabular}{ l|l }
  \textit{preconditions} & \textit{effects} \\
  \hline
  \stvarsmall{part-location-robot}(\constsmall{part},\constsmall{robot}) &$\neg$\stvarsmall{part-location-robot}(\constsmall{part},\constsmall{robot})\\
  \stvarsmall{robot-holds-part}(\constsmall{robot},\constsmall{part})
  & $\neg$\stvarsmall{robot-holds-part}(\const{robot},\const{part})\\
  \stvarsmall{on-worktable-kit}(\constsmall{worktable},\constsmall{kit})
  & $\neg$\stvarsmall{found-part}(\const{part},\const{partstray})\\
  \stvarsmall{kit-location-worktable}(\constsmall{kit},\constsmall{worktable})
  &\stvarsmall{robot-empty}(\constsmall{robot})\\
  \stvarsmall{origin-part}(\constsmall{part},\constsmall{partstray})
  & \stvarsmall{part-location-kit}(\constsmall{part},\constsmall{kit})\\
  (< (\stvarsmall{quantity-parts-in-kit}(\constsmall{kit},\constsmall{partstray})) (\stvarsmall{capacity-kit}(\constsmall{kit},\constsmall{partstray})))
  & (\stvarsmall{increase}\ (\stvarsmall{quantity-kit}(\constsmall{kit},\constsmall{partstray})))\\
  & \stvarsmall{part-not-searched}
\end{tabular}

\begin{itemize}
 \item \textit{preconditions}
 \begin{itemize}
 \item \stvarsmall{part-location-robot}(\constsmall{part},\constsmall{robot}): \constsmall{part} is held by \constsmall{robot}.
 \item \stvarsmall{robot-holds-part}(\constsmall{robot},\constsmall{part}): \constsmall{robot} is holding \constsmall{part}.
 \item \stvarsmall{on-worktable-kit}(\constsmall{worktable},\constsmall{kit}): \constsmall{worktable} has \constsmall{kit} on top of it.
 \item \stvarsmall{kit-location-worktable}(\constsmall{kit},\constsmall{worktable}): \constsmall{kit} is on \constsmall{worktable}.
 \item \stvarsmall{origin-part}(\constsmall{part},\constsmall{partstray}): \constsmall{part} is from \constsmall{partstray}. This is used to tell the type of \constsmall{part}.
 \item (< (\stvarsmall{quantity-kit}(\constsmall{kit},\constsmall{partstray})) (\stvarsmall{capacity-kit}(\constsmall{kit},\constsmall{partstray}))): The quantity of parts of type \constsmall{partstray} in \constsmall{kit} should be lesser than the capacity \constsmall{kit} can hold for this type of part.
 \end{itemize}
\item \textit{effects}
 \begin{itemize}
 \item $\neg$\stvarsmall{part-location-robot}(\constsmall{part},\constsmall{robot}): \constsmall{part} is not held by \constsmall{robot}.
 \item $\neg$\stvarsmall{robot-holds-part}(\const{robot},\const{part}): \constsmall{robot} is not holding \constsmall{part}.
 \item \stvarsmall{robot-empty}(\constsmall{robot}): \constsmall{robot} is not holding any object.
 \item \stvarsmall{part-location}(\constsmall{part},\constsmall{kit}): \constsmall{part} is located in \constsmall{kit}.
 \item (\stvarsmall{increase}\ (\stvarsmall{quantity-kit}(\constsmall{kit},\constsmall{partstray}))): Once \constsmall{part} is placed in \constsmall{kit}, the quantity of \constsmall{parts} in \constsmall{kit} is increased by one.
 \item \stvarsmall{part-not-searched}: This flag is set to true so another part search (through the operator \op{look-for-part}) is made after \constsmall{part} is placed in \constsmall{kit}.
 \end{itemize}
 \end{itemize}
%%%%%%%%%%%%%%%%%%%%%%%%%%%%%%%%%%%%%

\item \op{attach-endeffector}(\const{robot},\const{endeffector},\const{endeffectorholder},\const{endeffectorchangingstation}): The \class{Robot} \const{robot} attaches the \class{EndEffector} \const{endeffector} which is situated in the \class{EndEffectorHolder} \const{endeffectorholder}.

\begin{tabular}{l}
  \textit{preconditions}\\
  \hline
  \stvarsmall{endeffector-location-endeffectorholder}(\constsmall{endeffector},\constsmall{endeffectorholder})\\
  \stvarsmall{robot-with-no-endeffector}(\constsmall{robot})\\
  \stvarsmall{endeffectorholder-holds-endeffector}(\constsmall{endeffectorholder},\constsmall{endeffector})\\
  \stvarsmall{endeffectorholder-location}(\constsmall{endeffectorholder},\constsmall{endeffectorchangingstation}) \\
  \stvarsmall{endeffectorchangingstation-contains-endeffectorholder}(\constsmall{endeffectorchangingstation},\constsmall{endeffectorholder})
\end{tabular}

\begin{tabular}{l}
  \textit{effects}\\
  \hline
  $\neg$\stvarsmall{endeffector-location-endeffectorholder}(\constsmall{endeffector},\constsmall{endeffectorholder})\\
  $\neg$\stvarsmall{endeffectorholder-holds-endeffector}(\constsmall{endeffectorholder},\constsmall{endeffector})\\
  $\neg$\stvarsmall{robot-with-no-endeffector}(\constsmall{robot})\\
  \stvarsmall{robot-empty}(\constsmall{robot})\\
  \stvarsmall{endeffector-location-robot}(\constsmall{endeffector},\constsmall{robot})\\
  \stvarsmall{robot-with-endeffector}(\constsmall{robot},\constsmall{endeffector})\\
  \stvarsmall{endeffectorholder-empty}(\constsmall{endeffectorholder})
\end{tabular}

\begin{itemize}
 \item \textit{preconditions}
 \begin{itemize}
 \item \stvarsmall{endeffector-location-endeffectorholder}(\constsmall{endeffector},\constsmall{endeffectorholder}): \constsmall{endeffector} is located in \constsmall{endeffectorholder}.
 \item \stvarsmall{robot-with-no-endeffector}(\constsmall{robot}): \constsmall{robot} is not equipped with any \constsmall{endeffector}.
 \item \stvarsmall{endeffectorholder-holds-endeffector}(\constsmall{endeffectorholder},\constsmall{endeffector}): \constsmall{endeffectorholder} is holding \constsmall{endeffector}.
 \item \stvarsmall{endeffectorholder-location}(\constsmall{endeffectorholder},\constsmall{endeffectorchangingstation}): \constsmall{endeffectorholder} is in \constsmall{endeffectorchangingstation}.
 \item \stvarsmall{endeffectorchangingstation-contains-endeffectorholder}(\constsmall{endeffectorchangingstation},\constsmall{endeffectorholder}): \constsmall{endeffectorchangingstation} contains \constsmall{endeffectorholder}.
 \end{itemize}
\item \textit{effects}
 \begin{itemize}
 \item $\neg$\stvarsmall{endeffector-location-endeffectorholder}(\constsmall{endeffector},\constsmall{endeffectorholder}): \constsmall{endeffector} is not in \constsmall{endeffectorholder} anymore since it has been attached to \constsmall{robot}.
 \item $\neg$\stvarsmall{endeffectorholder-holds-endeffector}(\constsmall{endeffectorholder},\constsmall{endeffector}): \constsmall{endeffectorholder} is not holding \constsmall{endeffector} anymore.
 \item $\neg$\stvarsmall{robot-with-no-endeffector}(\constsmall{robot}): \constsmall{robot} is now equipped with \constsmall{endeffector}.
 \item \stvarsmall{robot-empty}(\constsmall{robot}): \constsmall{robot} is not holding any object.
 \item \stvarsmall{endeffector-location-robot}(\constsmall{endeffector},\constsmall{robot}): \constsmall{endeffector} is on \constsmall{robot}.
 \item \stvarsmall{robot-with-endeffector}(\constsmall{robot},\constsmall{endeffector}): \constsmall{robot} is equipped with \constsmall{endeffector}.
 \item \stvarsmall{endeffectorholder-empty}(\constsmall{endeffectorholder}): \constsmall{endeffectorholder} is not holding any \constsmall{endeffector}.
 \end{itemize}
 \end{itemize}
%%%%%%%%%%%%%%%%%%%%%%%%%%%%%%%%%%%%%

\item \op{remove-endeffector}(\const{robot},\const{endeffector},\const{endeffectorholder},\const{endeffectorchangingstation}): The \class{Robot} \const{robot} removes the \class{EndEffector} \const{endeffector} and puts it in the \class{EndEffectorHolder} \const{endeffectorholder}.

\begin{tabular}{l}
  \textit{preconditions}\\
  \hline
  \stvarsmall{endeffector-location-robot}(\constsmall{endeffector},\constsmall{robot})\\
  \stvarsmall{robot-with-endeffector}(\constsmall{robot},\constsmall{endeffector})\\
  \stvarsmall{robot-empty}(\constsmall{robot})\\
  \stvarsmall{endeffectorholder-location}(\constsmall{endeffectorholder},\constsmall{endeffectorchangingstation})\\
  \stvarsmall{endeffectorchangingstation-contains-endeffectorholder}(\constsmall{endeffectorchangingstation},\constsmall{endeffectorholder})\\
  \stvarsmall{endeffectorholder-empty}(\constsmall{endeffectorholder})
\end{tabular}



\begin{tabular}{l}
  \textit{effects}\\
  \hline
  $\neg$\stvarsmall{endeffector-location-robot}(\constsmall{endeffector},\constsmall{robot})\\
  $\neg$\stvarsmall{robot-with-endeffector}(\constsmall{robot},\constsmall{endeffector})\\
  $\neg$\stvarsmall{endeffectorholder-empty}(\constsmall{endeffectorholder})\\
  \stvarsmall{endeffector-location-endeffectorholder}(\constsmall{endeffector},\constsmall{endeffectorholder})\\
  \stvarsmall{endeffectorholder-holds-endeffector}(\constsmall{endeffectorholder},\constsmall{endeffector})\\
  \stvarsmall{robot-with-no-endeffector}(\constsmall{robot})
\end{tabular}

\begin{itemize}
 \item \textit{preconditions}
 \begin{itemize}
 \item \stvarsmall{endeffector-location-robot}(\constsmall{endeffector},\constsmall{robot}): \constsmall{endeffector} is on \constsmall{robot}.
 \item  \stvarsmall{robot-with-endeffector}(\constsmall{robot},\constsmall{endeffector}): \constsmall{robot} is holding \constsmall{endeffector}.
 \item   \stvarsmall{robot-empty}(\constsmall{robot}): \constsmall{robot} is not holding anything.
 \item   \stvarsmall{endeffectorholder-location}(\constsmall{endeffectorholder},\constsmall{endeffectorchangingstation}): \constsmall{endeffectorholder} is in \constsmall{endeffectorchangingstation}.
 \item \stvarsmall{endeffectorchangingstation-contains-endeffectorholder}(\constsmall{endeffectorchangingstation},\constsmall{endeffectorholder})
 \item   \stvarsmall{endeffectorholder-empty}(\constsmall{endeffectorholder}): \constsmall{endeffectorholder} does not contain \constsmall{endeffector}.
 \end{itemize}
\item \textit{effects}
 \begin{itemize}
 \item  $\neg$\stvarsmall{endeffector-location-robot}(\constsmall{endeffector},\constsmall{robot}): \constsmall{endeffector} is not on \constsmall{robot} anymore.
 \item  $\neg$\stvarsmall{robot-with-endeffector}(\constsmall{robot},\constsmall{endeffector}): \constsmall{robot} does not have \constsmall{endeffector} anymore.
 \item  $\neg$\stvarsmall{endeffectorholder-empty}(\constsmall{endeffectorholder}): \constsmall{endeffectorholder} does not contain any \constsmall{endeffector}.
 \item  \stvarsmall{endeffector-location-endeffectorholder}(\constsmall{endeffector},\constsmall{endeffectorholder}): \constsmall{endeffector} is situated in \constsmall{endeffectorholder}.
 \item  \stvarsmall{endeffectorholder-holds-endeffector}(\constsmall{endeffectorholder},\constsmall{endeffector}): \constsmall{endeffectorholder} holds \constsmall{endeffector}.
 \item  \stvarsmall{robot-with-no-endeffector}(\constsmall{robot}): \constsmall{robot} is not equipped with \constsmall{endeffector}.
 \end{itemize}
 \end{itemize}

%%%%%%%%%%%%%%%%%%%%%%%%%%%%%%%%%%%%%

\item \op{create-kit}(\const{kit},\const{kittray},\const{worktable}): The \class{KitTray} \const{kittray} is converted into the \class{Kit} \const{kit} once the \class{KitTray} \const{kittray} is on the \class{WorkTable} \const{worktable}.

\begin{tabular}{ l|l }
  \textit{preconditions} & \textit{effects} \\
  \hline
  \stvarsmall{on-worktable-kittray}(\constsmall{worktable},\constsmall{kittray})
  &$\neg$\stvarsmall{on-worktable-kittray}(\constsmall{worktable},\constsmall{kittray})\\
  &\stvarsmall{on-worktable-kit}(\constsmall{worktable},\constsmall{kit})\\
  &\stvarsmall{kit-location-worktable}(\constsmall{kit},\constsmall{worktable})\\
\end{tabular}

\begin{itemize}
 \item \textit{preconditions}
 \begin{itemize}
 \item \stvarsmall{on-worktable-kittray}(\constsmall{worktable},\constsmall{kittray}): \constsmall{worktable} has \constsmall{kittray} on top of it.
 \end{itemize}
\item \textit{effects}
 \begin{itemize}
 \item $\neg$\stvarsmall{on-worktable-kittray}(\constsmall{worktable},\constsmall{kittray}): The object \constsmall{kittray} is destroyed and is thus not on \constsmall{worktable} anymore.
 \item \stvarsmall{on-worktable-kit}(\constsmall{worktable},\constsmall{kit}): \constsmall{worktable} now has \constsmall{kit} on top of it.
 \item \stvarsmall{kit-location-worktable}(\constsmall{kit},\constsmall{worktable}): \constsmall{kit} is on \constsmall{worktable}.
 \end{itemize}
 \end{itemize}
%%%%%%%%%%%%%%%%%%%%%%%%%%%%%%%%%%%%%
\end{enumerate}

\subsubsection{Actions}
An action \textit{a} can be obtained by substituting the object variable symbols that
appear anywhere in the operator with constant variable symbols. For instance, the operator \op{take-part}(\const{robot},\const{part},\const{partstray},\const{endeffector}) in the kitting domain can be translated into the action \op{take-part}(\const{robot\_1},\const{part\_1},\const{partstray\_1},\const{endeffector\_2}) where \const{robot\_1}, \const{part\_1}, \const{partstray\_1}, and \const{endeffector\_2} are constant variable symbols in the classes \class{Robot}, \class{Part}, \class{PartsTray}, and \class{EndEffector}, respectively.





