%%%%%%%%%%%%%%%%%%%%%%%%%%%%%%%%%%%%%%%%%%%%%%%%%%%%%%%%%%%%%%%%%%%%%%%%%%%%%%%%
%2345678901234567890123456789012345678901234567890123456789012345678901234567890
%        1         2         3         4         5         6         7         8

\documentclass[letterpaper, 10 pt, conference]{ieeeconf}  % Comment this line out
                                                          % if you need a4paper
%\documentclass[a4paper, 10pt, conference]{ieeeconf}      % Use this line for a4
                                                          % paper

\IEEEoverridecommandlockouts                              % This command is only
                                                          % needed if you want to
                                                          % use the \thanks command
\overrideIEEEmargins
% See the \addtolength command later in the file to balance the column lengths
% on the last page of the document



% The following packages can be found on http:\\www.ctan.org
%\usepackage{graphics} % for pdf, bitmapped graphics files
%\usepackage{epsfig} % for postscript graphics files
%\usepackage{mathptmx} % assumes new font selection scheme installed
%\usepackage{times} % assumes new font selection scheme installed
%\usepackage{amsmath} % assumes amsmath package installed
%\usepackage{amssymb}  % assumes amsmath package installed

\title{\LARGE \bf
Assessment of the Performance Evaluation for Knowledge-based Kitting Applications via Analysis of Metrics Scoring Methods
}

%\author{ \parbox{3 in}{\centering Huibert Kwakernaak*
%         \thanks{*Use the $\backslash$thanks command to put information here}\\
%         Faculty of Electrical Engineering, Mathematics and Computer Science\\
%         University of Twente\\
%         7500 AE Enschede, The Netherlands\\
%         {\tt\small h.kwakernaak@autsubmit.com}}
%         \hspace*{ 0.5 in}
%         \parbox{3 in}{ \centering Pradeep Misra**
%         \thanks{**The footnote marks may be inserted manually}\\
%        Department of Electrical Engineering \\
%         Wright State University\\
%         Dayton, OH 45435, USA\\
%         {\tt\small pmisra@cs.wright.edu}}
%}

\author{Thomas Kramer$^{1}$, Zeid Kootbally$^{2}$, Steve Balakirsky$^{3}$, Craig Schlenoff$^{4}$, Anthony Pietromartire$^{4}$, and Satyandra Gupta$^{5}$% <-this % stops a space
%\thanks{*This work was not supported by any organization}% <-this % stops a space
\thanks{$^{1}$T. Kramer is with the Department of Mechanical Engineering, Catholic University of America, Washington, DC, USA
        {\tt\small thomas.kramer@nist.gov}}%
\thanks{$^{2}$Z. Kootbally is with the Department of Mechanical Engineering, University of Maryland, College Park, MD, USA
        {\tt\small zeid.kootbally@nist.gov}}
\thanks{$^{3}$S. Balakirsky is with the Georgia Tech Research Institute, Atlanta, GA
{\tt\small stephen.balakirsky@gtri.gatech.edu}}%}
\thanks{$^{4}$C. Schlenoff and A. Pietromartire are with the Intelligent Systems Division,
National Institute of Standards and Technology, Gaithersburg, MD, USA
{\tt\small craig.schlenoff@nist.gov \& pietromartire.anthony@nist.gov}}%}
\thanks{$^{5}$S. Gupta is with the Maryland Robotics Center, University of Maryland,
College Park, MD, USA
        {\tt\small skgupta@umd.edu}}%
}


\begin{document}



\maketitle
\thispagestyle{empty}
\pagestyle{empty}


%%%%%%%%%%%%%%%%%%%%%%%%%%%%%%%%%%%%%%%%%%%%%%%%%%%%%%%%%%%%%%%%%%%%%%%%%%%%%%%%
\begin{abstract}

The IEEE Robotics and Automation Society's (RAS) Ontologies for Robotics and Automation Working Group is dedicated to developing a methodology for knowledge representation and reasoning in robotics and automation. As part of this working group, the Industrial Robots sub-group is tasked with studying industrial applications of the knowledge representation. One of the first areas of interest for this subgroup is the area of kit building or kitting. This is a process that brings parts that will be used in assembly operations together in a kit and then moves the kit to the assembly area where the parts are used in the final assembly. It is anticipated that utilization of the knowledge representation will allow for the development of higher performing kitting systems.
While our previous effort aimed at designing the basis for performance methods and metrics to determine higher performing kitting systems, the work presented in this paper assesses the performance evaluation of our kitting system through the analysis of metrics scoring methods.

\end{abstract}


%%%%%%%%%%%%%%%%%%%%%%%%%%%%%%%%%%%%%%%%%%%%%%%%%%%%%%%%%%%%%%%%%%%%%%%%%%%%%%%%
\section{INTRODUCTION}
zeid: 0.5-1 page: Industrial kitting, test methods and metrics for assembly from the literature review.

\section{DESIGN METHODOLOGY}
zeid: 0.5-1 page
\section{KNOWLEDGE MODEL}
zeid: 0.5-1 page
\section{TEST METHODS AND METRICS}
Tom - 1 page: Take from NISTIR and augment with new stuff. May
\section{KITTING VIEWER}
Tom - 1 page: Take from NISTIR and augment with new stuff. Maybe also discuss about the methods developed to score metrics.
\section{RESULTS}
Tom - 1 page: Maybe discuss the results for scoring metrics. This section should be the highlight of this paper.

\section{CONCLUSIONS AND FUTURE WORK}
zeid: 0.5 page

%\section{REFERENCES}

%%%%%%%%%%%%%%%%%%%%%%%%%%%%%%%%%%%%%%%%%%%%%%%%%%%%%%%%%%%%%%%%%%%%%%%%%%%%%%%%


\begin{thebibliography}{99}
\bibitem{c20} J. P. Wilkinson, �Nonlinear resonant circuit devices (Patent style),� U.S. Patent 3 624 12, July 16, 1990.
\end{thebibliography}




\end{document}
