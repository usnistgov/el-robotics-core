Applications for assembly robots have been primarily implemented in fixed
and programmable automation. Fixed automation is a process using mechanized
machinery to perform repetitive operations in order to produce a
high volume of similar parts. In programmable automation, products are made
in batch quantities ranging from several dozen to several thousand units at
a time. Although today's state-of-the-art industrial robots are capable of
sub-millimeter accuracy~\cite{RobotAccuracy}, they are often programmed by
an operator using crude positional controls from a teach pendant. Moreover,
drastic modifications in the design process are realized when parts need
major changes or become too complicated in design, thus resulting in long
set up time to accommodate the new product style.

Reprogramming these robots when the aforementioned conditions arise
requires that the robot cell be taken off-line for a human-led teaching
period. For small batch processors or other customers who must frequently
change their line configuration, this frequent down time may be
unacceptable. The robotic systems of tomorrow need to be capable, flexible,
and agile. These systems need to perform their duties at least as well as
human counterparts, be able to be quickly re-tasked to other operations,
and be able to cope with a wide variety of unexpected environmental and
operational changes. In order to be successful, these systems need to
combine domain expertise, knowledge of their own skills and limitations,
and both semantic and geometric environmental information. The challenge of
expanding industrial use of robots is through flexible automation where
minimized setup times can lead to more output and generally better
throughput. Flexible robots are required to be quickly reconfigured for
design and process modifications. Furthermore, they need to be information
driven so that they can reliably perform different sequences of actions on
command. The cost for generating the information that drives the actions
must be kept down and this information must be accurate.

This paper discusses a flexible kitting system. Kitting is the process in
which several different, but related items are placed into a container and
supplied together as a single unit (kit). Kitting itself may be viewed as a
specialization of the general bin-picking problem. In industrial assembly
of manufactured products, kitting is often performed prior to final
assembly. Kitting, when applied properly, has been observed to show
numerous benefits for the assembly line, such as cost
savings \cite{Carlsson_2008} including saving manufacturing or assembly
space \cite{Medbo2003}, reducing assembly workers' walking and searching
times \cite{Schwind1992}, and increasing line flexibility \cite{Bozer1992}
and balance \cite{Jiao2000}.

The increasing global competition demands the manufacturing industry move towards state-of-the-art flexible systems. Therefore, it is important
not only to know how to measure the complexity of an assembly operation --
the tasks that form a plan -- but also to evaluate how well these
operations are carried out by a manufacturing process. Recently, some
theoretical measures have been explored to evaluate assembly sequences
under different contexts. For example, Sanderson and Homem de
Mello~\cite{SANDERSON.1987} used the And/Or graph approach to represent
feasible assembly/disassembly sequences and a function based on parts
entropy measures as evaluation criteria for search in the And/Or graph
space. Park \textit{et al.}~\cite{PARK.1991} described the evaluation
criteria based on four evaluation functions to minimize the chances of
assembly failures caused by spatial errors of parts and assembly equipment.
Bonneville \textit{et al.}~\cite{BONNEVILLE.1995} presented a genetic
algorithm that uses the assembly trees as models for the plans to encode
them in chromosomes. The liaisons graph and the geometric constraints of
the model are then used to evaluate the assembly operations. Su\'{a}rez and
Lee~\cite{SUAREZ.1997} proposed a statistical measure of an assembly cost
index to evaluate the assembly sequences in order to determine the optimal
one.


The reader should note that the effort presented in this paper depends on the existence of a knowledge-based representation of the environment. As such, comparing the proposed methods with the ones that do not involve a knowledge-based approach is out of the scope of this paper. This paper presents a system that evaluates the performance of kitting plans through simulation using specific metrics.  It is anticipated that kitting
plans can be translated into canonical robot control language
(CRCL)~\cite{NISTIR.Balakirsky} command sets which may then be evaluated
by standardized metric software. The CRCL command sets may then be
translated into a specific robot platform's language.

The organization of the remainder of this paper is as follows:
Section~\ref{sect:knowledge_driven_methodology} presents an overview of the
knowledge driven methodology that has been developed for this effort.
Section~\ref{sect:KittingViewer} describes the Kitting Viewer tool that
simulates the execution of a plan (CRCL command file) for changing a
kitting workstation from an initial state to a goal state.
Section~\ref{sect:Metrics} reviews the test method and metrics that are
used to compare the performance of different kitting planning systems.
Section~\ref{sect:Scoring} describes a configurable scoring system that
uses five factors combined as specified by a scoring file to compute the
score of the plan, and Section~\ref{sect:Conclusions} concludes this paper
and discusses future work.


