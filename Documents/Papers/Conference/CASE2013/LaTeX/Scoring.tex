The Kitting Viewer has a configurable scoring system that uses the
following five factors combined as specified by a scoring file to compute
the score. The factors and the score are recomputed after each movable goal
object is checked.

\begin{itemize}

\item \small \sf Right Stuff \rm \normalsize -- The Right Stuff value is
[(the number of objects in the goal file placed correctly so far) minus
(the number of objects in the goal file placed incorrectly so far)] divided
by [the number of objects in the goal file checked so far]. If that is less
than zero, it is set to zero.

\item \small \sf Command Execution \rm \normalsize -- The Command Execution
value is the fraction of all commands in the command file that were
executed correctly. That is [(the total number of commands executed
successfully) divided by (the total number of commands executed
successfully plus all errors except location errors)]. This factor does
not change during the second phase of Kitting Viewer operation.

\item \small \sf Distance \rm \normalsize -- The Distance value is [(two
times the total distance moved from initial position to goal position by
all basic goal objects that have been checked so far) divided by (the total
distance moved by the robot times the fraction of movable objects that have
been checked)]. If the calculated Distance value is greater than 1, it is
set to 1. The numerator is a crude measure of a short distance to move the
robot in order to move the objects that have been moved so far to their
goal positions.

\item \small \sf Time \rm \normalsize -- The Time value is [(the teleport
time) divided by (the total execution time multiplied by the fraction of
movable objects that have been checked)]. If the calculated Time value is
greater than 1, it is set to 1. The teleport time is [(two times the total
distance moved from initial position to goal position by all basic goal
objects that have been checked so far) divided by (the robot maximum
speed)]. The teleport time is a crude measure of a fast time for moving the
objects that have been moved so far to their goal positions.

\item \small \sf Useless Commands \rm \normalsize -- The Useless Command
factor is the value of the useless commands metric. This factor does
not change during the second phase of Kitting Viewer operation.

\end{itemize}

A scoring file used in the Kitting Viewer is an XML data file conforming to
the scoreKitting.xsd XML schema file. C++ classes and a parser for scoring
files were generated automatically using the GenXMiller mentioned earlier.

The scoring file, as prepared by a user, designates each of the five
factors as being either multiplicative or additive. For additive factors, a
weight may be assigned in the file. For each factor, a valuation function
may be assigned. A valuation function takes a raw factor with an arbitrary
range and produces a number between 0 and 1. The final stage of producing
a score combines numbers between 0 and 1. If a raw factor (useless
commands, for example) is not necessarily between 0 and 1, a valuation
function must be assigned to that factor. If a raw factor is always between
0 and 1 (right stuff, for example), a valuation function may be assigned,
but is not required.

The scoring system is designed so that the score it produces is always
between 0 and 100.

Each factor is designated as additive or multiplicative. A factor
value $V_i$ between 0 and 1 is found for each additive factor and a
factor value $U_i$ between 0 and 1 is found for each multiplicative
factor. Each additive factor is assigned a non-negative weight $W_i$. An
additive score $S_a$ is produced by multiplying each additive value by
its weight, adding the products together, and dividing by the sum of the
weights. If there are no additive factors or their weights are all
zero, $S_a = 1$.
\[
S_a = \frac{(V_1\times W_1)+(V_2\times W_2)\cdots + (V_n\times W_n)}{W_1+W_2+\cdots+W_n}
\]
The value of $S_a$ will be between 0 and 1 since all the components of the
equation are positive and the largest the numerator can be is the size
of the denominator.

Then the total score S is found by finding the product of $S_a$, 100, and
all the multiplicative factors.

$S = (100\times S_a\times U_1\times U_2\cdots \times U_m)$



