The effort presented in this paper is designed to support the IEEE Robotics and Automation Society's Ontologies for Robotics and Automation Working Group. Kitting is the process in which several different, but related items are placed into a container and supplied together as a single unit (kit). Kitting itself may be viewed as a specialization of the general bin-picking problem. Industrial assembly of manufactured products is often performed by first bringing parts together in a kit and then moving the kit to the assembly area where the parts are used to assemble products. Agile and flexible kitting, when applied properly, has been observed to show numerous benefits for the assembly line, such as cost savings~\cite{Carlsson_2008} including saving manufacturing or assembly space~\cite{Medbo2003}, reducing assembly workers' walking and searching times~\cite{Schwind1992}, and increasing line flexibility~\cite{Bozer1992} and balance~\cite{Jiao2000}.

Applications for assembly robots have been primarily implemented in fixed and programmable automation. Fixed automation is a process using mechanized machinery to perform fixed and repetitive operations in order to produce a high volume of similar parts. Although fixed automation provides maximum efficiency at a low unit cost, drastic modifications of the machines are realized when parts need major changes or become too complicated in design. In programmable automation, products are made in batch quantities ranging from several dozen to several thousand units at a time. However, each new batch requires long set up time to accommodate the new product style. The time and therefore the cost of developing applications for fixed and programmable automation is usually quite high. The challenge of expanding industrial use of robots is through agile and flexible automation where minimized setup times can lead to more output and generally better throughput.

The effort presented in this paper describes an approach based on a simulated sensor in an attempt to move towards an agile system. Tasking a system sensor to retrieve information on objects of interest should be performed in a timely manner before the robot carries out actions that involve these objects of interest. Objects in a kitting workcell are likely susceptible to be moved by external agents, parts trays may be depleted, and objects of different types can be unintentionally mixed with other types. Consequently, the system should be able to detect any of the aforementioned cases by tasking the sensor system to retrieve pose estimations of objects of interest.

Pose estimation is an important capability for grasping and manipulation. A wide variety of solutions have been proposed in order to extend the current structure of the systems to an agile system. Most of the efforts in the literature have focused primarily on solutions for robots whose mobility is restricted to the ground plane. Lysenkov \textit{et al.} \cite{Lysenkov-RSS-12} presented new algorithms for segmentation, pose estimation and recognition of transparent objects. Their system showed that a robot is able to grasp 80\% of known transparent objects with the proposed algorithm and this result is robust across non-specular backgrounds behind the objects. Dzitac and Mazid~\cite{Dzitac.ICARCV.2012} proposed a flexible and inexpensive object detection and localization method for pick-and-place robots based on the Xtion and Kinect. The authors relied on depth sensors to provide the robots with flexible and powerful means of locating objects, such as boxes, without the need to hard code the exact coordinates of the box in the robot program. Rusu \textit{et al.} \cite{RUSU.IROS.2010}  presented a novel 3D feature descriptor, the Viewpoint Feature Histogram (VFH), for object recognition and 6DOF pose identification for application where a priori segmentation is possible. 

 The organization of the remainder of this paper is as follows. Section \ref{section:architecture} presents an overview of the knowledge driven methodology used in this effort. Section \ref{section:simulation} describes the simulation environment used for kitting applications. Section \ref{section:sensor} details the approach that tasks a sensor system to retrieve information on objects of interest, and Section \ref{section:conclusions} concludes this paper and analyzes future work.
% 
% However, our architecture
% must be designed to represent the required knowledge base
% in several different abstractions that are likely to be required
% by these systems as the knowledge flows from domain and
% process specification, to plan generation, to plan execu-
% tion.
