
\begin{figure}[!t!h!b!]
\centering
\includegraphics[width=10cm]{images/flowchart.pdf}
\caption{Flowchart diagram for tasking the simulated sensor.}
\label{fig:sensor}
\end{figure}

As seen previously, Section~\ref{subsection:usartruth} describes how a simulated sensor system operates to retrieve 6DOF poses of objects in the kitting workcell. This section describes when the simulated sensor system is used. Figure~\ref{fig:sensor} is a flowchart that represents some of the steps used for kitting, from parsing the \textsf{Plan Instance File} to the execution of each action from this file. Since the focus of this paper is on the sensor system, the authors have limited the representation and description of Figure~\ref{fig:sensor} around the sensor system and did not include the steps prior to the \textsf{Plan Instance File} generation. The reader may find this missing information in the description of Figure~\ref{fig:methodology} in Section~\ref{section:architecture}. The different steps depicted in Figure~\ref{fig:sensor} are categorized into main components that are numbered. A description of each main component is given in the following subsections.

\subsection{Read Plan Instance File}
As described in Section~\ref{section:architecture}, the \textsf{Plan Instance File} is generated by the \textsf{Domain Independant Planning System} from the \textsf{PDDL Domain File} and the \textsf{PDDL Problem File}. An example of a plan is given in Figure~\ref{fig:Solution}. This plan describes the PDDL actions that a robot will need to execute in order to build a kit that consists of one part of type D and one part of type E. At the beginning of the plan (line 1), the end effector that is capable of grasping parts is taken from the end effector changing station and attached to the robot. Lines 2 and 4 display the actions for picking up a part of type E and D, respectively. Lines 3 and 5 display the actions for putting parts E and D in the kit, respectively. Finally, at line 6, the end effector is put back in the end effector changing station.
\begin{figure}[t!h!]
\begin{minipage}{.5\paperwidth}
\begin{list}{}{\setlength{\leftmargin}{0em}}\item\small
\begin{Verbatim}[commandchars=\\\{\},fontsize=\normalsize, numbers=left, numbersep=0pt]
(\opsmall{attach-endeffector} \constsmall{robot_1} \constsmall{part_gripper} \constsmall{part_gripper_holder} \constsmall{changing_station_1})
(\opsmall{take-part} \constsmall{robot_1} \constsmall{part_e1} \constsmall{ptr_e} \constsmall{part_gripper})
(\opsmall{put-part} \constsmall{robot_1} \constsmall{part_e1} \constsmall{kit_a2b3c3d1e1} \constsmall{work_table_1} \constsmall{ptr_e})
(\opsmall{take-part} \constsmall{robot_1} \constsmall{part_d1} \constsmall{ptr_d} \constsmall{part_gripper})
(\opsmall{put-part} \constsmall{robot_1} \constsmall{part_d1} \constsmall{kit_a2b3c3d1e1} \constsmall{work_table_1} \constsmall{ptr_d})
(\opsmall{remove-endeffector} \constsmall{robot_1} \constsmall{part_gripper} \constsmall{part_gripper_holder} \constsmall{changing_station_1})
\end{Verbatim}
\end{list}
\end{minipage}
\caption{Excerpt of the PDDL solution file for kitting.}
\label{fig:Solution}
\end{figure}


\subsection{Generate CRCL Commands}
\label{subsection:CRCL}
Each action of the plan is sequentially interpreted and then directly executed by the robot. The \textsf{Interpreter} takes as input a PDDL action from the \textsf{Plan Instance File} and outputs a set of CRCL commands for this action. To facilitate late binding, the PDDL actions within the plan  do not specify the exact locations of the parts and components that are involved. This kind of knowledge detail is maintained by sensor processing and is stored in the \textsf{MySQL Database}. As described in Section~\ref{section:architecture}, the generation of the tables in the \textsf{MySQL Database} is followed by data insertion in these tables for all the objects in the environment. However, there is no guarantee that the poses of these objects are still accurate as they may have been altered in different ways. At this point, the sensor system is tasked to retrieve information about objects of interest. Objects of interest are the ones for which the poses are needed to execute some CRCL commands. Before tasking the sensor to retrieve the poses of objects of interest via the \textsf{get} message, the external shape of each object of interest must be retrieved from the \textsf{ExternalShape} MySQL table.

An external shape is a shape defined in an external file. An external shape has a model format name, stored in the field \textsf{hasExternalShape\_ModelName} of the ontology, a model type name, stored in the field \textsf{hasExternalShape\_ModelTypeName}, and a model file name which is the name of the file containing the model, stored in the field \textsf{hasExternalShape\_ModelFileName}. Using the information retrieved from the aforementioned fields for each object of interest, the system then parses the data coming in from USARTruth and updates the relative pose in the \textsf{MySQL Database} for each object returned. This is performed via the \textsf{set} message. Since USARTruth returns object locations in global coordinates, the relative pose for each object is updated without changing its transformation tree; that is, the object of reference for its physical location is unchanged.


The actual updated relative pose is computed according to Equation~\ref{eq:one}.
 \begin{equation}
\label{eq:one}
 L' = LG^{-1}G'
 \end{equation}

where $L'$ is the updated relative transformation, $L$ is the old relative transformation (read from the \textsf{MySQL Database}), $G$ is the old global transformation (computed from the transformation tree in the \textsf{MySQL Database}), and $G'$ is the updated global transformation (retrieved from USARTruth).

Once the above process is performed, the \textsf{Interpreter} uses the new data to generate a set of CRCL commands for the current action, as depicted by \mycirc{A} in Figure~\ref{fig:sensor}. The \op{take-part} action at line 2 in Figure~\ref{fig:Solution} is interpreted as the sequence of CRCL commands displayed in Table~\ref{tab:takepart}, where the numerical data used in the \texttt{MoveTo} commands are computed with the new 6DOF poses. The reader may find more information about the whole set of CRCL commands in~\cite{Balakirsky2012-1}.

\begin{table}[h!]

\caption{A set of CRCL commands for the action \op{take-part}.}
\centering
  \begin{tabular}{l}
    \hline
    \texttt{initCannon()}\\
    \texttt{Message (``take part part\_e1'')}\\
    \texttt{MoveTo(\{\{-0.03, 1.62, -0.25\}, \{0, 0, 1\}, \{1, 0, 0\}\})}\\
    \texttt{Dwell (0.05)}\\
    \texttt{MoveTo(\{\{-0.03, 1.62, 0.1325\}, \{0, 0, 1\}, \{1, 0, 0\}\})} \\
    \texttt{CloseGripper ()} \\
    \texttt{MoveTo(\{\{-0.03, 1.62, -0.25\}, \{0, 0, 1\}, \{1, 0, 0\}\})}\\
    \texttt{Dwell (0.05)}\\
    \texttt{endCannon()}\\
    \hline
  \end{tabular}
  \label{tab:takepart}
\end{table}

Once a set of CRCL commands is generated for a PDDL action, it is sent to the \textsf{Robot Controller} to be executed by the robot. The \textsf{Predicate Evaluation} process is then called before and after each set of CRCL commands is carried out by the robot.

\subsection{Predicate Evaluation (Preconditions)}
As mentioned in Section~\ref{section:architecture}, a PDDL action consists of one precondition section and one effect section that are defined in the \textsf{OWL/XML SOAP Ontology}. Preconditions and effects consist of a set of predicates. For instance, the predicates in the precondition and effect for the action \op{take-part}(\const{robot\_1},\const{part\_e1},\\\const{ptr\_e},\const{part\_gripper}) are defined in Table~\ref{tab:takepartpddl}.


\begin{table}[h!]
\caption{The precondition and the effect for the action \op{take-part}.}
\centering
\begin{tabular}{ l|l }
  \textit{precondition} & \textit{effect} \\
  \hline
  \stvarsmall{part-location-partstray}(\constsmall{part\_e1},\constsmall{ptr\_e})
  &$\neg$\stvarsmall{part-location-partstray}(\constsmall{part\_e1},\constsmall{ptr\_e})\\
  \stvarsmall{robot-empty}(\constsmall{robot\_1})
  & $\neg$\stvarsmall{robot-empty}(\constsmall{robot\_1})\\
  \stvarsmall{endeff-location-robot}(\constsmall{part\_gripper},\constsmall{robot\_1})
  & \stvarsmall{part-location-robot}(\constsmall{part\_e1},\constsmall{robot\_1})\\
  \stvarsmall{robot-with-endeff}(\constsmall{robot\_1},\constsmall{part\_gripper})
  & \stvarsmall{robot-holds-part}(\constsmall{robot\_1},\constsmall{part\_e1})\\
  \stvarsmall{endeff-type-part}(\constsmall{part\_gripper},\const{part\_e1})
  &  \\
  \stvarsmall{partstray-not-empty}(\constsmall{ptr\_e})
  &
\end{tabular}
  \label{tab:takepartpddl}
\end{table}

\subsubsection{Spatial Relations -- }
 The evaluation of the predicates in the precondition section assures that all the requirements are met in the environment before the robot carries out the action. As such, the output of the \textsf{Predicate Evaluation} process is a Boolean value. The kitting system relies on the representation of spatial relations that are stored in the \textsf{OWL/XML SOAP Ontology} to compute the truth-value of each predicate. A brief description of each spatial relation is given below. A thorough analysis of spatial relations used for kitting is well documented in~\cite{SCHLENOFF.RAS.2013}.
\begin{itemize}
 \item Predicates: These are domain-specific states that are of interest to the current activity. The truth-value of predicates can be determined through the logical combination of intermediate state relations.
\item Intermediate state relations: These are generic, re-usable level state relations that can be inferred from the combination of Region Connected Calculus (RCC8)~\cite{Wolter2000} and cardinal direction relations.
\item RCC8 relations: RCC8 is a well-known and cited approach for representing the relationship between two regions in Euclidean space or in a topological space. RCC8 was initially developed for a two-dimensional space, but has been extended for the purpose of the kitting effort to a three-dimensions space by applying it along all three planes (x-y, x-z, y-z). Each of the intermediate state relations consists of a set of logical rules that associate these RCC8 relations to them. There are 24 RCC8 relations and 6 cardinality direction operators.
\end{itemize}

\subsubsection{Sensor System -- }
To evaluate the truth-value of any given predicate, the sensor system is tasked to retrieve the 6DOF pose estimation of the predicate's parameters. This is performed the same way it is described in Section~\ref{subsection:CRCL} and is represented by \mycirc{B} in Figure~\ref{fig:sensor}. The \textsf{Predicate Evaluation} process then proceeds as follows:

%The predicates developed for the kitting effort have at least one parameter and at most two parameters. In the case the predicate has two parameters, the external shape of each parameter is fetched from the \textsf{MySQL Database} and is passed to USARTruth to get the pose information of each object. The \textsf{Predicate Evaluation} process then proceeds as follows:
\begin{itemize}
\item[] In the \textsf{OWL/XML SOAP Ontology}:
\begin{enumerate}
 \item Identify the predicate.
 \item Identify the intermediate state relation for the predicate from step 1.
 \item Identify the set of logical rules that associate RCC8 relations to the intermediate state relation from step 2.
\end{enumerate}
\end{itemize}

\subsubsection{RCC8 Evaluation -- }
Next, the poses of the predicate's parameters are used to compute the truth-value of the identified set of logical rules of RCC8 relations for this predicate. The predicates developed for the kitting effort have at least one parameter and at most two parameters. To evaluate the set of RCC8 relations, two methods are used.


\begin{enumerate}
\item In the case the predicate has two parameters, the poses of these two parameters are used to compute the truth-value of the set of RCC8 relations. For instance, the predicate \stvar{part-location-partstray}(\const{part\_e1},\const{ptr\_e}) from Table~\ref{tab:takepartpddl} is true if and only if the part \const{part\_e1} is \textbf{Partially-In} the parts tray \const{ptr\_e}.  \textbf{Partially-In} (formula~\ref{formula:partiallyin1}) is a state relation that represents an object fully inside of a second object in two dimensions and partially in the third dimension. Please note that the state relation presented in formula~\ref{formula:partiallyin1} is used to demonstrate how the parameters of a predicate are used to compute the truth-value of a set of RCC8 relations. Descriptions of each state relation and each RCC8 relation (\texttt{Z-Plus}, \texttt{Z-NTPP}, \texttt{Z-NTPPi}, etc) are available in~\cite{SCHLENOFF.RAS.2013}.


\begin{gather}
\label{formula:partiallyin1}
\textbf{Partially-In}(\textit{part\_e1}, \textit{ptr\_e}) \rightarrow   \\
(\texttt{Z-Plus}(\textit{part\_e1}, \textit{ptr\_e}) \wedge (\texttt{Z-NTPP}(\textit{part\_e1}, \textit{ptr\_e}) \vee \texttt{Z-NTPPi}(\textit{part\_e1}, \textit{ptr\_e}) \vee  \notag\\\texttt{Z-PO}(\textit{part\_e1}, \textit{ptr\_e}) \vee \texttt{Z-TPP}(\textit{part\_e1}, \textit{ptr\_e}) \vee \texttt{Z-TPPi}(\textit{part\_e1}, \textit{ptr\_e}))) \wedge \notag\\
(\texttt{X-NTPP}(\textit{part\_e1}, \textit{ptr\_e}) \vee \texttt{X-NTPPi}(\textit{part\_e1}, \textit{ptr\_e}) \vee \texttt{X-TPP}(\textit{part\_e1}, \textit{ptr\_e}) \vee  \notag\\ \texttt{X-TPPi}(\textit{part\_e1}, \textit{ptr\_e}))\wedge
(\texttt{Y-NTPP}(\textit{part\_e1}, \textit{ptr\_e}) \vee \texttt{Y-NTPPi}(\textit{part\_e1}, \textit{ptr\_e}) \vee  \notag\\ \texttt{Y-TPP}(\textit{part\_e1}, \textit{ptr\_e}) \vee \texttt{Y-TPPi}(\textit{part\_e1}, \textit{ptr\_e})) \notag
\end{gather}

\item In the case the predicate has only one parameter, a second object is needed to compute the truth-value of the predicate between the parameter and the other object. The authors remind the reader that RCC8 relations necessarily require two objects. In this case, the sensor system is tasked to retrieve the pose for each other object in the workcell to be used as the second object. Depending on the predicate, the search space for the other object can be narrowed down. For instance, the predicate \stvar{partstray-not-empty}(\const{ptr\_e}) (Table~\ref{tab:takepartpddl}) is true if and only if at least one part of type E is in the parts tray \const{ptr\_e}. It is not necessary to extend the search for the second object among the other types of object present in the workcell. It is not relevant, for instance, to check if the parts tray contains an end effector. On the other hand, to check the truth value of the predicate \stvar{robot-empty}(\const{robot\_1}) (Table~\ref{tab:takepartpddl}), which is true if and only if the robot \const{robot\_1} is not holding anything in the end effector attached to the robot, the predicate evaluation needs to scan a wider search space than the one described for the predicate \stvar{partstray-not-empty}(\const{ptr\_e}), i.e., the search space includes all parts of each type, all types of kit trays, etc. Once the search space has been defined, the external shape for each object in the search space is retrieved from the appropriate table from the \textsf{MySQL Database}. As described previously, the external shape is used by the sensor system to retrieve the 6DOF pose for the corresponding object. The truth-value of the set of logical rules that associate RCC8 relations to the intermediate state relation for the predicate is then computed for these two objects, that is the predicate parameter and the searched object.

\end{enumerate}

If all the predicates within the precondition section are true, the \textsf{MySQL Database} is updated with the current poses of these predicates' parameters. It is important that all the predicates are true in order to update the \textsf{MySQL Database}. This ensures that a complete (stable) state of the environment is stored in the \textsf{MySQL Database}. If at least one predicate is evaluated to false, it is considered a failure and the kitting process is terminated.

\subsection{Execute Action}
When all the predicates within the precondition of an action have been evaluated to true, this action is executed by the robot. To confirm that the action was successfully accomplished, the \textsf{Predicate Evaluation} process evaluates the predicates within the effect section for this action. The effect section consists of predicates that are expected to be true after performing a PDDL action. The evaluation of the predicates within the effect section is performed to confirm that these expectations are attained.

\subsection{Predicate Evaluation (Effects)}
The methodology previously described to evaluate the predicates within the precondition section of an action is also used to evaluate the predicates within the effect section of this action. This is depicted by \mycirc{C} in Figure~\ref{fig:sensor}. As mentioned earlier, all the predicates within the effect section must be evaluated to true to define the action as successful. In the case of a successful action, the next action within the \textsf{Plan Instance File} is processed. Once all the actions within the \textsf{Plan Instance File} have been executed by the robot, the kitting process is complete.
