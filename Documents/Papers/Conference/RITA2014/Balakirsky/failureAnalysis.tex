\section{Failure Analysis}
As seen in Figure~\ref{fig:predicateEval}, failures are identified during the evaluation of preconditions and effects. 
In addition to recognizing failures, it is desired to be able to respond to them.
In order to properly model this response, additional information must be modeled
in the ontology. This information is structured as shown below.
\begin{description}
	\item{\it Action} -- The action being performed when the failure was detected.
	\begin{description}
  		\item{\it Failure Modes} -- A description of the known failure modes that can occur during the execution of a specific action.
  		\begin{description}
			\item{\it Causes} -- One or more root causes that could cause the failure. 
  			\begin{description}
  				\item{\it Consequences} -- What are the particular consequences of the detected failure?
  		   		\item{\it Severity} -- An assessment of how serious the consequences of a particular failure are. Each consequence is given a rank of severity ranging from 1 (minor) to 10 (very high). The severity rank is used to trigger the appropriate contingency plan.
 				\item{\it Remediation} -- What should the system do to remediate this
 				failure?
  			\end{description}
   		\end{description}
  		\item{\it Predicates} -- The predicates that were detected as invalid during the failure analysis.
	\end{description}
\end{description}

Table~\ref{tab:putpartfailure} shows an example of failure modes associated to the PDDL action \textsl{put-part}($\mathit{robot}$,$\mathit{part}$,$\mathit{kit}$,$\mathit{worktable}$,$\mathit{partstray}$) which is defined as ``The \textit{Robot} $\mathit{robot}$ puts the \textit{Part} $\mathit{part}$ in the \textit{Kit} $\mathit{kit}$''.
%previously presented in Figure~\ref{fig:put-part}.

%%%%%%%%%%%%%%%%%%%%%%%%%%%%%%%%%%%%
%%%%%%%%%%% put-part %%%%%%%%%%%
%%%%%%%%%%%%%%%%%%%%%%%%%%%%%%%%%%%%
\begin{table}[h!t!]
  %\centering
  \caption{Failure modes for the PDDL action \textit{put-part}.}
  \label{tab:putpartfailure}
  \scalebox{0.7}{
  \begin{tabular}{|l|l|l|l|c|c|c|}
    \hline
    \multicolumn{1}{|c|}{\begin{sideways}Action\end{sideways}} &
    \multicolumn{1}{c|}{\begin{sideways}Failure Mode(s) \,\end{sideways}} &
    \multicolumn{1}{c|}{\begin{sideways}Cause(s) \,\end{sideways}} &
    \multicolumn{1}{c|}{\begin{sideways}Effect(s) \,\end{sideways}} &
    \multicolumn{1}{c|}{\begin{sideways}Severity \,\end{sideways}} &
    \multicolumn{1}{c|}{\begin{sideways}Occurrence (\%) \,\end{sideways}} &
    \multicolumn{1}{c|}{\begin{sideways}Predicate(s) \,\end{sideways}} \\
    \hline
    \multirow{5}{*}{\textit{\small{put-part}}} &
    \multirow{2}{*}{\stvar{part} falls off of the end effector} &
    \multirow{2}{*}{end effector hardware issues} &
    downtime & %\stvar{endeffector} replacement &
    9 &
    \multirow{2}{*}{60} &
    \small {\textsf{part-location-robot}}\\\cline{4-5}

     &
     &
     &
    \stvar{part} damage & %\stvar{endeffector} replacement &
    5 &
     &
    \small {\textsf{robot-holds-part}}\\\cline{2-7}

     &
     \multirow{3}{*}{\stvar{part} not released at all}&
     end effector hardware issues &
     downtime &
     9 &
     \multirow{3}{*}{8} &
     $\neg$(\small{\textsf{part-location-robot}})\\\cline{3-5}

     &
     &
     \multirow{2}{*}{wrong/inexistant CRCL command} &
     \multirow{2}{*}{downtime} &
     \multirow{2}{*}{7} &
      &
     $\neg$(\small{\textsf{robot-holds-part}})\\
%
     &
     &
     &
     &
     &
     &
     \small {\textsf{robot-empty}}\\\hline

\end{tabular}
}
\end{table}
The column \textit{Predicate(s)} shows the predicates from the action \textit{put-part} that can activate the corresponding failure mode (column \textit{Failure Mode(s)}) if their truth-value is evaluated to false.

The classes discussed below are used to represent action failures in the SOAP ontology. All these classes are subclasses of \class{DataThing}.
\begin{enumerate}
\item \class{Action} -- An \class{Action} has at least one \class{FailureMode} (\emph{hasAction\_FailureMode}).
\item \class{FailureMode} -- A \class{FailureMode} has at least one \class{FailureEffect} (\emph{hasFailureMode\_FailureEffect}). A \class{FailureMode} has a description (\emph{hasFailureMode\_Description}) of type \texttt{Literal} which represents the nature of the failure mode. The cause of the failure mode is expressed with \emph{hasFailureMode\_Cause} and is of type \texttt{Literal}. The occurrence of the failure mode is expressed with \emph{hasFailureMode\_Occurrence} and is of type \texttt{Integer}. A \class{FailureMode} has at least one \class{Predicate} (\emph{hasFailureMode\_Predicate}) that can be responsible for the occurrence of the failure mode. The \class{Predicate} should be already defined \textit{prior} to its association with the class \class{FailureMode}.
\item \class{FailureEffect} -- A \class{FailureEffect} has one failure severity (\emph{hasFailureEffect\_FailureSeverity}) of type \texttt{integer} and a description (\emph{hasFailureEffect\_Description}) of type \texttt{Literal}.
\end{enumerate}
%
%