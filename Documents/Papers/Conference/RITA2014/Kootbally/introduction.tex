\section{Introduction}
Applications for assembly robots have been primarily implemented in fixed and programmable automation. Fixed automation is a process using mechanized machinery to perform fixed and repetitive operations in order to produce a high volume of similar parts. Although fixed automation provides maximum efficiency at a low unit cost, drastic modifications of the machines are realized when parts need major changes or become too complicated in design. In programmable automation, products are made in batch quantities ranging from several dozen to several thousand units at a time. However, each new batch requires long set up time to accommodate the new product style. The time and therefore the cost of developing applications for fixed and programmable automation is usually quite high. The challenge of expanding industrial use of robots is through agile and flexible automation where minimized setup times can lead in to more output and generally better throughput.

%The concept of agile manufacturing~\cite{SHERIDAN.1993,Struebing.1995,NAGEL.1991} was first introduced in 1991 at the end of a government-sponsored research effort at Lehigh University. This effort reunited a group of more than 150 industry executives to discuss the evolution of US industrial competitiveness in the next 15 years.

%Many definitions of agile manufacturing have emerged during the last 30 years without being opposed or contradictory to each other~\cite{Dahmardeh.2010}. Kidd~\cite{KIDD.2000} defined agility as a way for companies to rapidly adapt to tomorrow's surprises where agile production can be considered as a structure in the company which has the ability of product developments and some business methods. For Maskell~\cite{Maskell.2001}, the three main components of agile production are customer's growth and flourish, compatibility of individuals and information, and cooperation and change ability. According to Gunasekaran~\cite{GUNASEKARAN.1999}, agile production is a new production model resulting from changes in environment which links innovations in production, information technology and communication by fundamental organizational redesigning and new marketing strategies.

Agility mainly represents the idea of ``speed and change in business environment" and consists of two main factors: 1) Responding to change in proper ways and due time and 2) Exploiting changes and taking advantage of them as opportunities~\cite{SHARIFI.1999}. Agile manufacturing demands a manufacturing system that is able to produce effectively a large variety of products and to be reconfigurable to accommodate changes in the product mix and product designs~\cite{GUNASEKARAN.1999}. Reconfigurability of manufacturing systems, product variety, velocity and flexibility in production are critical aspects to achieving and maintaining competitive advantage. The concept of agility has an impact on the design of assemblies and therefore, methodologies for the design of agile manufacturing are needed.

Some of the key words and terms linked to ``agile manufacturing" are~\cite{KIDD.2000}:
\begin{itemize}
\item Fast -- A very high speed of response to new opportunities.
\item Adaptable -- The ability to change the current approach and change direction with ease.
\item Reconfiguration -- The ability to quickly reconfigure a firm structures, facilities, organization, and technology to meet unexpected and short lived market opportunities.
\item Transformation of knowledge -- The ability to explicitly transform raw ideas into a range of competence which is then used in both products and services.
\end{itemize}

The effort presented in this paper describes a methodology and tools in the area of industrial assembly of manufactured products. This effort is designed to support the IEEE Robotics and Automation Society's Ontologies for Robotics and Automation Working Group. Industrial assembly of manufactured products is often performed by first bringing parts together in a kit and then moving the kit to the assembly area where the parts are used to assemble products. Kitting is the process in which several different, but related items are placed into a container and supplied together as a single unit (kit). Kitting itself may be viewed as a specialization of the general bin-picking problem. In industrial assembly of manufactured products, kitting is often performed prior to final assembly. Agile and flexible kitting, when applied properly, has been observed to show numerous benefits for the assembly line, such as cost savings~\cite{Carlsson_2008} including saving manufacturing or assembly space~\cite{Medbo2003}, reducing assembly workers walking and searching times~\cite{Schwind1992}, and increasing line flexibility~\cite{Bozer1992} and balance~\cite{Jiao2000}.

The effort described in this paper tends to move towards the ideas of a fast and adaptable system. Tasking a system sensor to retrieve information on objects of interest should be performed in a timely manner before performing any actions of the plan that involve these objects of interest. Parts and components in a kitting work cell are likely susceptible to be moved during the course of time by external agents and thus, the system should be able to cope with these unexpected changes.




The organization of the remainder of this paper is as follows. Section \ref{sect:architecture} presents an overview of the knowledge driven methodology that has been developed for this effort with a particular focus on the areas where the sensor system interacts with. Section \ref{sect:simulation} describes the simulation environment for the domain of kitting . Section \ref{sec:sensor} details the tools and the new methodology developed to task a sensor system to retrieve information on objects of interest, and Section \ref{sect:future} presents conclusions and future work.