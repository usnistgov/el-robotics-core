% LaTex file for Stephen Balakirsky's RITA 2014 submission
%
\documentclass{llncs}
%
\usepackage{makeidx}  % allows for indexgeneration
%
% my additional packages
\usepackage{graphicx}
\usepackage{multirow,array}
\usepackage{rotating} % for sideways
\usepackage{amsmath}
\usepackage{amsthm}
\usepackage{subfigure}

%\usepackage{algorithm2e}
\usepackage[linesnumbered, boxed, figure]{algorithm2e}
%
% zeid's definitions
\newcommand{\class}[1] {\textit{#1}}
\newcommand{\const}[1] {$\mathit{#1}$}
\newcommand{\objvar}[1] {$\mathsf{#1}$}
\newcommand{\stvar}[1] {\textsf{#1}}
%
%
\begin{document}
%
\frontmatter          % for the preliminaries
%
\pagestyle{headings}  % switches on printing of running heads
\addtocmark{Hamiltonian Mechanics} % additional mark in the TOC
%
%
\mainmatter              % start of the contributions
%
%\title{Development of a Canonical Sensor Command/Response Language for Industrial Kit Building Applications}
\title{Improve the Agility of Kit Building Applications via Virtual Sensor Tasking}
%
\titlerunning{Agile Manufacturing}  % abbreviated title (for running head)
%                                     also used for the TOC unless
%                                     \toctitle is used
%
\author{Zeid Kootbally\inst{1}, Craig Schlenoff\inst{2}, Teddy Weisman\inst{3}, Stephen Balakirsky\inst{4}, and Anthony Pietromartire\inst{2}}
%
\authorrunning{Zeid Kootbally et al.} % abbreviated author list (for running head)
%
%%%% list of authors for the TOC (use if author list has to be modified)
\tocauthor{Zeid Kootbally, Craig Schlenoff, Teddy Weisman, Stephen Balakirsky, and Anthony Pietromartire}
%
\institute{
University of Maryland, College Park, MD 20740, USA,\\
\email{zeid.kootbally@nist.gov},\\ WWW home page:
\texttt{www.nist.gov/el/isd/ks/kootbally.cfm}
\and 
Intelligent Systems Division, National Institute of Standards and Technology, Gaithersburg, MD, USA,\\
\email{craig.schlenoff@nist.gov, anthony.pietromartire@nist.gov},\\ 
WWW home page:
\texttt{www.nist.gov/el/smartcyber.cfm}
\and
Yale University, New Haven, CT 06520,\\
\email{tjweisman@gmail.com}
\and
Georgia Tech Research Institute, Atlanta, GA 30332, USA,\\
\email{stephen.balakirsky@gtri.gatech.edu},\\ 
WWW home page: 
\texttt{unmannedsystems.gtri.gatech.edu}
}

\maketitle              % typeset the title of the contribution

\begin{abstract}

6 degree of freedom (6DOF) pose estimation of objects are required for robotic grasping and manipulation during kit building. Kit building or kitting is a process in which individually separate but related items are grouped, packaged, and supplied together as one unit (kit). It is anticipated that utilization of the knowledge representation will allow for the development of higher performing kitting systems.

In this paper, kitting is performed by a robotic arm that manipulates parts/components (objects of interest) by following an ordered sequence of steps that constitute a plan. To prevent availability failures of objects of interest, it is necessary to know their pose estimation prior executing each step of the plan. This is performed by querying a simulated sensor system thus providing an agile system where the system can cope with objects of interest in the case they are moved within robotic work cells by external agents.

This paper describes advances in developing key software, sensing/control, and parts-handling technologies enabling robust operation of kitting applications in simulation. In particular, an interface is presented and a description of the methodology tasking a sensor system to retrieve 6DOF pose estimation of objects of interest is given. \end{abstract}
%
%
\section{Introduction}
A failure is any change, design, or manufacturing error that renders a component, assembly, or system incapable of performing its intended function \cite{Collins93}. In kitting, as described in Section \ref{sect:kitting}, failures can occur for multiple reasons that include equipment not being set up properly, tools and/or fixtures not being properly prepared, and improper equipment maintenance. Part/component availability failures can be triggered by inaccurate information on the location of the part, part damage, incorrect part types, or part shortage due to delays in internal logistics. In order to prevent or minimize failures, a disciplined approach needs to be implemented to identify the different ways a process design can fail and to allow for corrective actions to be taken before the failure impacts productivity.

Even though today's state-of-the-art industrial robots are capable of sub-millimeter accuracy \cite{RobotAccuracy}, they often lack the sensing
necessary to detect failures and the programming required to cope with and correct the failure. This is due to the fact that they are often programmed
by an operator using imprecise positional controls from a teach pendant. These teach pendant programs are highly repeatable, which provides 
utility for large-batch, error-free operation. However, the cyclic program that repeats identical operations does not lend itself well to adaptation for 
failure mitigation. In fact, producing a program to correct a perceived failure would require that the cell be taken off-line
for additional human-led teach pendant programming. In addition, 
most cells lack the ability to sense that a failure occurred and  lack programming (that would have had to be teach pendant entered) to cope
with failure conditions, thus making it impossible for the cell to recover from failures.
This leads to faulty products being sent down the line, and/or downtime for the cell as failures are detected and corrected.

For small batch processors or other customers who must frequently change their line configuration or desire to perform complex operations
with their robots, this frequent downtime and lack of failure correction/detection may be unacceptable. The robotic systems of tomorrow need to be capable, flexible, and agile.  
These systems need to perform their duties at least  as well as human counterparts, be quickly re-tasked to other operations, cope with a wide 
variety of unexpected environmental and operational changes, and be able to detect and correct errors in operation. 
To be successful, these systems need to combine domain expertise, knowledge of their own skills and limitations, and both semantic and geometric 
environmental information.

The IEEE Robotics and Automation Society's Ontologies for Robotics and Automation Working Group has taken the first steps in creating the 
infrastructure necessary for such a system, while the Industrial Subgroup has applied this infrastructure to create a sample kit building
system.  This work is presented in Balakirsky et al. \cite{balakirsky2013} which describes the construction of a robotic kit building
system that is able to cope with environmental and task changes without operator intervention. This article extends that work to utilize
the same infrastructure to allow for the detection and correction of action failures in the system.

The organization of the remainder of this paper is as follows. Section \ref{sect:kitting} describes the domain of kit building. Section \ref{sect:overview} presents
an overview of the software system architecture as well as details of the ontology and world model for the robot cell. Section \ref{sect:operation} discusses the detailed operation of cell, and Section \ref{sect:failure} discusses how failures are handled by the ontology. Finally, Section \ref{sect:future} presents
conclusions and future work.
%
%
\section{Kitting}
\label{sect:kitting}
Today's advanced manufacturing plants utilize mixed-model assembly where multiple product variants are built on the same line.  
According to Jim Tetreault, Ford's vice president of North America Manufacturing, 
new Ford\footnote{No approval or endorsement of any commercial product by the authors is intended or implied. Certain commercial software systems are identified in this paper to facilitate understanding. Such identification does not imply that these software systems are necessarily the best available for the purpose.} assembly facilities are able to build a full spectrum of vehicles on the same assembly line \cite{James2011}. One of the technologies that makes this possible
is the use of assembly kits.  Bozer and McGinnis \cite{Bozer1992} describe a kit as ``a specific
collection of components and/or subassemblies that together (i.e., in the same container) support one or more assembly
operations for a given product or shop order''. These  kits provide a synchronous material flow, where parts and components move to 
assembly stations in a just-in-time manner. The kits provide workers with the parts and tools that they need (which may vary from 
vehicle model to vehicle model) in the sequence that they need them. The use of kitting also allows a single delivery system to feed
multiple assembly stations thus saving manufacturing or assembly space \cite{Medbo2003} and provides an additional inspection opportunity 
that allows for the detection of part defects before they impact assembly operations. The individual operations of the station 
that builds the kits may be viewed as a specialization of the general
bin-picking problem \cite{Schyja2012} where parts are picked from one or more part bins or trays and placed into specific slots in a kit tray.

For our sample implementation, we assume that the robot cell is building one of several possible kit configurations. At execution time, the
cell has a set kit to build, but does not know the precise location of the kit tray, the part trays, or the location of individual parts in the part tray.
When a human builds a kit, they are able to inspect each part before adding it to the kit tray. This provides an additional level of quality control and
is an aspect that is desirable to have in our robotic system. During kit construction,
a robot performs a series of pick-and-place operations
in order to construct the kit. These operations include:
\begin{enumerate}
\item Pick up an empty kit and place it on the work table.
\item Pick up multiple component parts, inspect them, and place them in the kit.
\item Pick up the completed kit and place it in the full kit storage area.
\end{enumerate}
Each of these steps may be a compound action that includes
other actions such as end-of-arm tool changes, path planning,
and obstacle avoidance. The items that are being placed in the kit may be of varying size and shape and have various grasping and inspection
requirements.
%
\begin{figure}[htb!]
\begin{center}
\includegraphics[width=8.5cm]{images/RITAExecution.jpg}
\caption{Major components that make up the Sense--Model--Act paradigm of the kitting station.}
\label{fig:SenseModelAct}
\end{center}
\end{figure}
\label{sect:simulation}
In order to experiment with robotic systems, a researcher requires a controllable robotic platform, a control system that interfaces to the robotic system and provides behaviors for the robot to carry out, and an environment to operate in. Our kitting application relies on an open source (the game engine is free, but license restrictions do apply), freely available framework capable of fulfilling all of these requirements. This framework is the Unified System for Automation and Robot Simulation (USARSim) \cite{USARSimWeb}. It provides the robotic platform and environment.

\subsection{The USARSim Framework}

USARSim~\cite{CARPIN.LNAI.2006,WANG.WSC.2003} is a high-fidelity physics-based simulation system based on the Unreal Developers Kit (UDK)~\cite{UDKWeb} from Epic Games. USARSim was originally developed under a National Science Foundation grant to study Robot, Agent, Person Teams in Urban Search and Rescue~\cite{LEWIS.ICHC.2003}. Since that time, it has been turned into a National Institute of Standards and Technology (NIST)-led, community-supported, open source project that provides validated models of robots, sensors, and environments. Altogether, the Karma Physics engine~\cite{KarmEngine} and high-quality 3D rendering facilities of the Unreal game engine allow the creation of realistic simulation environments that provide the embodiment of a robotic system. Furthermore, USARSim comes with tools to develop objects and environments and it is possible to control the objects in the game through a Transmission Control Protocol/Internet Protocol (TCP/IP) socket with a host computer.

Through its usage of UDK, USARSim utilizes the physX physics engine~\cite{physXWeb} and high-quality 3D rendering facilities to create a realistic robotic system simulation environment. The current release of USARSim consists of various model environments, models of commercial and experimental robots, and sensor models. High fidelity at low cost is made possible by building the simulation on top of a game engine. By delegating  simulation specific tasks to a high volume commercial platform (available for free to most users) which provides superior visual rendering and physical modeling, full user effort can be devoted to the robotics-specific tasks of modeling platforms, control systems, sensors, interface tools, and environments. These tasks are in turn accelerated by the advanced editing and development tools integrated with the game engine. This leads to a virtuous spiral in which a wide range of platforms can be modeled with greater fidelity in a short period of time.

%USARSim was originally based upon simulated environments in the Urban Search and Rescue (USAR) domain. Realistic disaster scenarios as well
%as robot test methods were created (Figure 1(a)). Since then, USARSim has been used worldwide and more environments have been developed for different purposes. Other environments such as the NIST campus (Figure 1(b)) and factories (Figure 1(c)) have been used to test the performance of algorithms in different efforts [9–11]. The simulation is also widely used for the RoboCup Virtual Robot Rescue Competition [12], the IEEE Virtual Manufacturing and Automation Challenge [13], and has been applied to the DARPsA Urban Challenge (Figure 1(d)).


%\begin{figure}[t!]
%\centering
%\subfigure[Test Room.]{\label{TestRoom}\includegraphics[width=4cm]{images/Worlds/testRoom.jpg}}\qquad
%\subfigure[Factory,]{\label{3D_World-c}\includegraphics[width=4cm]{images/Worlds/factory.jpg}}
%\caption{Sample of 3D environments in USARSim.} \label{3D_World}
%\end{figure}
%\begin{figure}[t!]
%\centering
%\subfigure[Air Robot AR100B.]{\label{Fig:AirRobot}\includegraphics[width=4cm]{images/Robots/airRobot_2.jpg}}\qquad
%\subfigure[Kuka KR60,]{\label{Fig:KR60}\includegraphics[width=4cm]{images/Robots/kr60.jpg}}
%\caption{Sample of vehicles in USARSim.}
%\end{figure}
%USARSim was initially developed with a focus on differential drive wheeled robots. However, USARSim's open source framework has encouraged wide community interest and support that now allows USARSim to offer multiple robots, including humanoid robots, aerial platforms (Figure~\ref{Fig:AirRobot}), robotic arms (Figure~\ref{Fig:KR60}), and commercial vehicles. All robots in USARSim have a chassis, and may contain multiple wheels, sensors, and
%actuators. The robots are configurable (e.g. specify types of
%sensors/end effectors) through a configuration file that is read at run-time. The properties of the robots can
%also be configured, such as the battery life and the frequency of
%data transmission.

%%--------------
USARSim was originally based upon simulated environments in the (Urban Search and Rescue) USAR domain. Realistic disaster scenarios as well as robot test methods were created (Figure~\ref{TestRoom}).
Since then, USARSim has been used worldwide and more environments have been developed for different purposes. Other environments such as the NIST campus (Figure~\ref{3D_World-b}) and factories (Figure~\ref{3D_World-c}) have been used to test the performance of algorithms in different efforts~\cite{WANG.HFES.2005,BALAGUER.IROS.2008,KOOTBALLY.ITEA.2010}. The simulation is also widely used for the RoboCup Virtual Robot Rescue Competition \cite{RoboCupWeb}, the IEEE Virtual Manufacturing and Automation Challenge \cite{VMACWeb}, and has been applied to the DARPA Urban Challenge (Figure~\ref{3D_World-a}).

\begin{figure}[t!]
\centering
\subfigure[Test Room.]{\label{TestRoom}\includegraphics[width=4cm]{images/Worlds/testRoom.jpg}}\qquad
\subfigure[NIST main campus.]{\label{3D_World-b}\includegraphics[width=4cm]{images/Worlds/nist2.jpg}}\qquad
\subfigure[Factory,]{\label{3D_World-c}\includegraphics[width=4cm]{images/Worlds/factory.jpg}}\qquad%
\subfigure[Road course.]{\label{3D_World-a}\includegraphics[width=4cm]{images/Worlds/arda1.jpg}}
\caption{Sample of 3D environments in USARSim.} \label{3D_World}
\end{figure}

USARSim was initially developed with a focus on differential drive wheeled robots. However, USARSim's open source framework has encouraged wide community interest and support that now allows USARSim to offer multiple robots, including humanoid robots (Figure~\ref{Fig:Nao}), aerial platforms (Figure~\ref{Fig:AirRobot}), robotic arms (Figure~\ref{Fig:KR60}), and commercial vehicles (Figure~\ref{Fig:Kiva}). In USARSim, robots are based on physical computer aided design (CAD) models of the real
robots and are implemented by specialization of specific existing classes. This structure allows for easier development of new platforms that model custom designs.

All robots in USARSim have a chassis, and may contain multiple wheels, sensors, and
actuators. The robots are configurable (e.g., specify types of
sensors/end effectors) through a configuration file that is read at run-time. The properties of the robots can
also be configured, such as the battery life and the frequency of
data transmission.

\begin{figure}[t!]
\centering
\subfigure[Aldebaran Robotics Nao.]{\label{Fig:Nao}\includegraphics[width=4cm]{images/Robots/nao.jpg}}\qquad
\subfigure[Air Robot AR100B.]{\label{Fig:AirRobot}\includegraphics[width=4cm]{images/Robots/airRobot_2.jpg}}\qquad
\subfigure[Kuka KR60,]{\label{Fig:KR60}\includegraphics[width=4cm]{images/Robots/kr60.jpg}}\qquad
\subfigure[Kiva Robot.]{\label{Fig:Kiva}\includegraphics[width=4cm]{images/Robots/kiva.jpg}}
\caption{Sample of vehicles in USARSim.}
\end{figure}

\subsection{The Simulated Sensor System}
\label{subsection:usartruth}
Poses of objects in the virtual environment are retrieved with the USARTruth tool. USARTruth is capable of reading information about objects in USARSim by connecting as a client to TCP socket port 3989. The simulator USARTruthConnection object listens for incoming connections on port 3989 and receives queries over a socket in the form of strings formatted into key-value pairs.

The USARTruth connection accepts two different keys, ``class'' and ``name''. When USARSim receives a new string over the connection, it sends a sequence of key-value formatted strings back over the socket, one for each Unreal Engine Actor object that matches the requested class and object names. An example of the strings returned by USARSim is given below along with a description for each key.


\{Name P3AT\_0\} \{Class P3AT\} \{Time 29.97\} \{Location 0.67,2.30,1.86\} \\ \{Rotation 0.00,0.46,0.00\} where:
%\{Name P3AT\_0\} \{Class P3AT\} \{Time 29.97\} \{Location 0.67,2.30,1.86\} \{Rotation 0.00,0.46,0.00\} where:
%The return strings include the following keys:

\begin{itemize}
\item Name: The internal name of the object in USARSim.
\item Class: The name of the most specific Unreal Engine class the object belongs to.
\item Time: The number of seconds that have elapsed since the simulator started, as a floating-point value.
\item Location: The comma-separated position of the object in global coordinates.
\item Rotation: The comma-separated orientation of the object in global coordinates, in roll, pitch, yaw form.
\end{itemize}


%The virtual sensor system uses USARTruth to simulate real-world sensors by opening a new USARTruth connection and sending one request for specified Unreal Engine class names.

% each different Unreal Engine class name found in the \textsf{MySQL Database}. For simulation purposes, the Unreal Engine class name is stored as an object's ``external shape''. It is important to note that the Location and Rotation values coming out of USARTruth do not include noises and assume perfect sensor data.
%
% The system then parses the data coming in from USARTruth and updates the relative pose in the MySQL database for each object returned. Since USARTruth returns object locations in global coordinates, the relative pose for each object is updated without changing its transformation tree; that is, the RefObject for its PhysicalLocation is unchanged. The actual updated relative pose is computed according to
% \begin{equation}
% L' = LG^{-1}G'
% \end{equation}
% where $L'$ is the updated relative transformation, $L$ is the old relative transformation (read from the MySQL database), $G$ is the old global transformation (computed from the transformation tree in the database), and $G'$ is the updated global transformation (retrieved from USARTruth).

\section{6DOF Pose Estimation within the Knowledge Driven Methodology}
\label{sect:architecture}
\begin{figure}[!t!h!b!]
\centering
\includegraphics[width=10cm]{images/KnowledgeDrivenRobotics.pdf}
\caption{Knowledge Driven Design extensions -- In this figure, green shaded
  boxes with curved bottoms represent hand generated files while light blue
  shaded boxes with curved bottoms represent automatically created boxes.
  Rectangular boxes represent processes and libraries. }
\label{fig:methodology}
\end{figure}

The knowledge driven methodology presented in this section is not purposed
to act as a stand-alone system architecture. Rather it is intended to be an
extension to well developed hierarchical, deliberative architectures such
as 4D/RCS~\cite{Albus2000}. The overall knowledge driven methodology of the
system is depicted in Figure~\ref{fig:methodology}. The figure is organized
vertically by the representation that is used for the knowledge and
horizontally by the classical sense-model-act paradigm of intelligent
systems. The remainder of this section gives a brief description of each
level of the hierarchy to help the reader understand the basic concepts
implemented within the system architecture in order to captivate the main effort described in this paper. The reader may find a more
detailed description of each component and each level of the architecture
in other documented
publications~\cite{BALAKIRSKY.IROS.2012}.

\subsection{Domain Specific Information}
Knowledge begins with Domain Specific Information
(DSI) that captures operational knowledge that is necessary to be
successful in the particular domain in which the system is designed to
operate. This includes information on items ranging from what actions and
attributes are relevant, to what the necessary conditions are for an action
to occur and what the likely results of the action are. The authors have
chosen to encode this basic information in a formalism known as a state
variable representation~\cite{NAU.2004}.

\subsection{Ontology}
\label{sub:ontology}
The information encoded in the DSI is then organized into a domain
independent representation. A base ontology (\textsf{OWL/XML Kitting})
contains all of the basic information that was determined to be needed
during the evaluation of the use cases and scenarios. The knowledge is
represented in as compact a form as possible with knowledge classes
inheriting common attributes from parent classes. The \textsf{OWL/XML SOAP}
ontology describes not only aspects of actions and predicates but also the
individual actions and predicates that are necessary for the domain under
study. The instance files describe the initial and goal states for the
system through the \textsf{Kitting Init Conditions File} and the
\textsf{Kitting Goal Conditions File}, respectively. The initial state file
must contain a description of the environment that is complete enough for a
planning system to be able to create a valid sequence of actions that will
achieve the given goal state. The goal state file only needs to contain
information that is relevant to the end goal of the system. For the case of building a kit, this may simply be that a complete kit is located in a bin designed to hold completed kits.

Since both the OWL and XML implementations of the knowledge representation are file based, real time information proved to be problematic. In order to solve this problem, an automatically generated MySQL database has been introduced as part of the knowledge representation. More information on the generated MySQL database is given in the description of the Planning Language level.


\subsection{Planning Language}
Aspects of the knowledge previously described are automatically extracted and encoded in a form
that is optimized for a planning system to utilize (the Planning Language).
The planning language used in the knowledge driven system is expressed with
the Planning Domain Definition Language (PDDL)~\cite{PDDL} (version 3.0).
The PDDL input format consists of two files that specify the domain and the
problem. As shown in Figure~\ref{fig:methodology}, these files are
automatically generated from the ontology. From
these two files, a domain independent planning
system~\cite{Coles.ICAPS.2010} was used to produce a static \textsf{Plan Instance File}.

%As mentioned in Section~\ref{sub:ontology}, a MySQL is automatically generated and the tables of the database are filled based on real data from the OWL/XML Kitting Init Condition File (created at the Ontology level). 
While the knowledge representation presented in this paper provides the \lq\lq{}slots\rq\rq{} necessary for representing dynamic information, the
static file structure makes the utilization of these slots awkward. It is desirable to be able to represent the dynamic information in a dynamic database. For this reason, the authors have developed a technique for automatically generating tables for storing,  and access functions for obtaining, the data from the ontology in a MySQL database.

Reading data from and to the MySQL database instead of the ontology file offers the community easy access to a live data structure. Furthermore, it is more practical to modify the information stored in a database than if it was stored in an ontology, which in some cases, requires the deletion and re-creation of the whole file. A literature review reveals many efforts and methodologies that have been designed to produce SQL databases from ontologies. Our effort builds upon the work of Astrova et al.\cite{Astrova2007}. 

In addition to generating and filling the database tables, the authors have created tools that automatically generate a set of C{}\texttt{++} classes for reading and writing
information to the kitting MySQL database. The choice of C{}\texttt{++} was a team preference and we believe that other object-oriented languages could have been used in this project.

The interaction of the MySQL database with the Connectors to Predicate Evaluation (in the Act level) is detailed in Section \ref{sec:sensor}.

\subsection{Robot Language}
Once a plan has been formulated, the knowledge is transformed into a
representation that is optimized for use by a robotic system (the Robot
Language). The interpreter combines knowledge from the plan with knowledge
from the MySQL database to form a sequence of sequential actions that the
robot controller is able to execute. The authors devised a canonical robot
command language (CRCL) in which such lists can be written. The purpose of
the CRCL is to provide generic commands that implement the functionality of
typical industrial robots without being specific either to the language of
the planning system that makes a plan or to the language used by a robot
controller that executes a plan.

\begin{figure}[!t!h!b!]
\centering
\includegraphics[width=10cm]{images/flowchart.pdf}
\caption{Flowchart diagram for tasking the simulated sensor.}
\label{fig:sensor}
\end{figure}

As seen previously, Section~\ref{subsection:usartruth} describes how a simulated sensor system operates to retrieve 6DOF poses of objects in the kitting workcell. This section describes when the simulated sensor system is used. Figure~\ref{fig:sensor} is a flowchart that represents some of the steps used for kitting, from parsing the \textsf{Plan Instance File} to the execution of each action from this file. Since the focus of this paper is on the sensor system, the authors have limited the representation and description of Figure~\ref{fig:sensor} around the sensor system and did not include the steps prior to the \textsf{Plan Instance File} generation. The reader may find this missing information in the description of Figure~\ref{fig:methodology} in Section~\ref{section:architecture}. The different steps depicted in Figure~\ref{fig:sensor} are categorized into main components that are numbered. A description of each main component is given in the following subsections.

\subsection{Read Plan Instance File}
As described in Section~\ref{section:architecture}, the \textsf{Plan Instance File} is generated by the \textsf{Domain Independant Planning System} from the \textsf{PDDL Domain File} and the \textsf{PDDL Problem File}. An example of a plan is given in Figure~\ref{fig:Solution}. This plan describes the PDDL actions that a robot will need to execute in order to build a kit that consists of one part of type D and one part of type E. At the beginning of the plan (line 1), the end effector that is capable of grasping parts is taken from the end effector changing station and attached to the robot. Lines 2 and 4 display the actions for picking up a part of type E and D, respectively. Lines 3 and 5 display the actions for putting parts E and D in the kit, respectively. Finally, at line 6, the end effector is put back in the end effector changing station.
\begin{figure}[t!h!]
\begin{minipage}{.5\paperwidth}
\begin{list}{}{\setlength{\leftmargin}{0em}}\item\small
\begin{Verbatim}[commandchars=\\\{\},fontsize=\normalsize, numbers=left, numbersep=0pt]
(\opsmall{attach-endeffector} \constsmall{robot_1} \constsmall{part_gripper} \constsmall{part_gripper_holder} \constsmall{changing_station_1})
(\opsmall{take-part} \constsmall{robot_1} \constsmall{part_e1} \constsmall{ptr_e} \constsmall{part_gripper})
(\opsmall{put-part} \constsmall{robot_1} \constsmall{part_e1} \constsmall{kit_a2b3c3d1e1} \constsmall{work_table_1} \constsmall{ptr_e})
(\opsmall{take-part} \constsmall{robot_1} \constsmall{part_d1} \constsmall{ptr_d} \constsmall{part_gripper})
(\opsmall{put-part} \constsmall{robot_1} \constsmall{part_d1} \constsmall{kit_a2b3c3d1e1} \constsmall{work_table_1} \constsmall{ptr_d})
(\opsmall{remove-endeffector} \constsmall{robot_1} \constsmall{part_gripper} \constsmall{part_gripper_holder} \constsmall{changing_station_1})
\end{Verbatim}
\end{list}
\end{minipage}
\caption{Excerpt of the PDDL solution file for kitting.}
\label{fig:Solution}
\end{figure}


\subsection{Generate CRCL Commands}
\label{subsection:CRCL}
Each action of the plan is sequentially interpreted and then directly executed by the robot. The \textsf{Interpreter} takes as input a PDDL action from the \textsf{Plan Instance File} and outputs a set of CRCL commands for this action. To facilitate late binding, the PDDL actions within the plan  do not specify the exact locations of the parts and components that are involved. This kind of knowledge detail is maintained by sensor processing and is stored in the \textsf{MySQL Database}. As described in Section~\ref{section:architecture}, the generation of the tables in the \textsf{MySQL Database} is followed by data insertion in these tables for all the objects in the environment. However, there is no guarantee that the poses of these objects are still accurate as they may have been altered in different ways. At this point, the sensor system is tasked to retrieve information about objects of interest. Objects of interest are the ones for which the poses are needed to execute some CRCL commands. Before tasking the sensor to retrieve the poses of objects of interest via the \textsf{get} message, the external shape of each object of interest must be retrieved from the \textsf{ExternalShape} MySQL table.

An external shape is a shape defined in an external file. An external shape has a model format name, stored in the field \textsf{hasExternalShape\_ModelName} of the ontology, a model type name, stored in the field \textsf{hasExternalShape\_ModelTypeName}, and a model file name which is the name of the file containing the model, stored in the field \textsf{hasExternalShape\_ModelFileName}. Using the information retrieved from the aforementioned fields for each object of interest, the system then parses the data coming in from USARTruth and updates the relative pose in the \textsf{MySQL Database} for each object returned. This is performed via the \textsf{set} message. Since USARTruth returns object locations in global coordinates, the relative pose for each object is updated without changing its transformation tree; that is, the object of reference for its physical location is unchanged.


The actual updated relative pose is computed according to Equation~\ref{eq:one}.
 \begin{equation}
\label{eq:one}
 L' = LG^{-1}G'
 \end{equation}

where $L'$ is the updated relative transformation, $L$ is the old relative transformation (read from the \textsf{MySQL Database}), $G$ is the old global transformation (computed from the transformation tree in the \textsf{MySQL Database}), and $G'$ is the updated global transformation (retrieved from USARTruth).

Once the above process is performed, the \textsf{Interpreter} uses the new data to generate a set of CRCL commands for the current action, as depicted by \mycirc{A} in Figure~\ref{fig:sensor}. The \op{take-part} action at line 2 in Figure~\ref{fig:Solution} is interpreted as the sequence of CRCL commands displayed in Table~\ref{tab:takepart}, where the numerical data used in the \texttt{MoveTo} commands are computed with the new 6DOF poses. The reader may find more information about the whole set of CRCL commands in~\cite{Balakirsky2012-1}.

\begin{table}[h!]

\caption{A set of CRCL commands for the action \op{take-part}.}
\centering
  \begin{tabular}{l}
    \hline
    \texttt{initCannon()}\\
    \texttt{Message (``take part part\_e1'')}\\
    \texttt{MoveTo(\{\{-0.03, 1.62, -0.25\}, \{0, 0, 1\}, \{1, 0, 0\}\})}\\
    \texttt{Dwell (0.05)}\\
    \texttt{MoveTo(\{\{-0.03, 1.62, 0.1325\}, \{0, 0, 1\}, \{1, 0, 0\}\})} \\
    \texttt{CloseGripper ()} \\
    \texttt{MoveTo(\{\{-0.03, 1.62, -0.25\}, \{0, 0, 1\}, \{1, 0, 0\}\})}\\
    \texttt{Dwell (0.05)}\\
    \texttt{endCannon()}\\
    \hline
  \end{tabular}
  \label{tab:takepart}
\end{table}

Once a set of CRCL commands is generated for a PDDL action, it is sent to the \textsf{Robot Controller} to be executed by the robot. The \textsf{Predicate Evaluation} process is then called before and after each set of CRCL commands is carried out by the robot.

\subsection{Predicate Evaluation (Preconditions)}
As mentioned in Section~\ref{section:architecture}, a PDDL action consists of one precondition section and one effect section that are defined in the \textsf{OWL/XML SOAP Ontology}. Preconditions and effects consist of a set of predicates. For instance, the predicates in the precondition and effect for the action \op{take-part}(\const{robot\_1},\const{part\_e1},\\\const{ptr\_e},\const{part\_gripper}) are defined in Table~\ref{tab:takepartpddl}.


\begin{table}[h!]
\caption{The precondition and the effect for the action \op{take-part}.}
\centering
\begin{tabular}{ l|l }
  \textit{precondition} & \textit{effect} \\
  \hline
  \stvarsmall{part-location-partstray}(\constsmall{part\_e1},\constsmall{ptr\_e})
  &$\neg$\stvarsmall{part-location-partstray}(\constsmall{part\_e1},\constsmall{ptr\_e})\\
  \stvarsmall{robot-empty}(\constsmall{robot\_1})
  & $\neg$\stvarsmall{robot-empty}(\constsmall{robot\_1})\\
  \stvarsmall{endeff-location-robot}(\constsmall{part\_gripper},\constsmall{robot\_1})
  & \stvarsmall{part-location-robot}(\constsmall{part\_e1},\constsmall{robot\_1})\\
  \stvarsmall{robot-with-endeff}(\constsmall{robot\_1},\constsmall{part\_gripper})
  & \stvarsmall{robot-holds-part}(\constsmall{robot\_1},\constsmall{part\_e1})\\
  \stvarsmall{endeff-type-part}(\constsmall{part\_gripper},\const{part\_e1})
  &  \\
  \stvarsmall{partstray-not-empty}(\constsmall{ptr\_e})
  &
\end{tabular}
  \label{tab:takepartpddl}
\end{table}

\subsubsection{Spatial Relations -- }
 The evaluation of the predicates in the precondition section assures that all the requirements are met in the environment before the robot carries out the action. As such, the output of the \textsf{Predicate Evaluation} process is a Boolean value. The kitting system relies on the representation of spatial relations that are stored in the \textsf{OWL/XML SOAP Ontology} to compute the truth-value of each predicate. A brief description of each spatial relation is given below. A thorough analysis of spatial relations used for kitting is well documented in~\cite{SCHLENOFF.RAS.2013}.
\begin{itemize}
 \item Predicates: These are domain-specific states that are of interest to the current activity. The truth-value of predicates can be determined through the logical combination of intermediate state relations.
\item Intermediate state relations: These are generic, re-usable level state relations that can be inferred from the combination of Region Connected Calculus (RCC8)~\cite{Wolter2000} and cardinal direction relations.
\item RCC8 relations: RCC8 is a well-known and cited approach for representing the relationship between two regions in Euclidean space or in a topological space. RCC8 was initially developed for a two-dimensional space, but has been extended for the purpose of the kitting effort to a three-dimensions space by applying it along all three planes (x-y, x-z, y-z). Each of the intermediate state relations consists of a set of logical rules that associate these RCC8 relations to them. There are 24 RCC8 relations and 6 cardinality direction operators.
\end{itemize}

\subsubsection{Sensor System -- }
To evaluate the truth-value of any given predicate, the sensor system is tasked to retrieve the 6DOF pose estimation of the predicate's parameters. This is performed the same way it is described in Section~\ref{subsection:CRCL} and is represented by \mycirc{B} in Figure~\ref{fig:sensor}. The \textsf{Predicate Evaluation} process then proceeds as follows:

%The predicates developed for the kitting effort have at least one parameter and at most two parameters. In the case the predicate has two parameters, the external shape of each parameter is fetched from the \textsf{MySQL Database} and is passed to USARTruth to get the pose information of each object. The \textsf{Predicate Evaluation} process then proceeds as follows:
\begin{itemize}
\item[] In the \textsf{OWL/XML SOAP Ontology}:
\begin{enumerate}
 \item Identify the predicate.
 \item Identify the intermediate state relation for the predicate from step 1.
 \item Identify the set of logical rules that associate RCC8 relations to the intermediate state relation from step 2.
\end{enumerate}
\end{itemize}

\subsubsection{RCC8 Evaluation -- }
Next, the poses of the predicate's parameters are used to compute the truth-value of the identified set of logical rules of RCC8 relations for this predicate. The predicates developed for the kitting effort have at least one parameter and at most two parameters. To evaluate the set of RCC8 relations, two methods are used.


\begin{enumerate}
\item In the case the predicate has two parameters, the poses of these two parameters are used to compute the truth-value of the set of RCC8 relations. For instance, the predicate \stvar{part-location-partstray}(\const{part\_e1},\const{ptr\_e}) from Table~\ref{tab:takepartpddl} is true if and only if the part \const{part\_e1} is \textbf{Partially-In} the parts tray \const{ptr\_e}.  \textbf{Partially-In} (formula~\ref{formula:partiallyin1}) is a state relation that represents an object fully inside of a second object in two dimensions and partially in the third dimension. Please note that the state relation presented in formula~\ref{formula:partiallyin1} is used to demonstrate how the parameters of a predicate are used to compute the truth-value of a set of RCC8 relations. Descriptions of each state relation and each RCC8 relation (\texttt{Z-Plus}, \texttt{Z-NTPP}, \texttt{Z-NTPPi}, etc) are available in~\cite{SCHLENOFF.RAS.2013}.


\begin{gather}
\label{formula:partiallyin1}
\textbf{Partially-In}(\textit{part\_e1}, \textit{ptr\_e}) \rightarrow   \\
(\texttt{Z-Plus}(\textit{part\_e1}, \textit{ptr\_e}) \wedge (\texttt{Z-NTPP}(\textit{part\_e1}, \textit{ptr\_e}) \vee \texttt{Z-NTPPi}(\textit{part\_e1}, \textit{ptr\_e}) \vee  \notag\\\texttt{Z-PO}(\textit{part\_e1}, \textit{ptr\_e}) \vee \texttt{Z-TPP}(\textit{part\_e1}, \textit{ptr\_e}) \vee \texttt{Z-TPPi}(\textit{part\_e1}, \textit{ptr\_e}))) \wedge \notag\\
(\texttt{X-NTPP}(\textit{part\_e1}, \textit{ptr\_e}) \vee \texttt{X-NTPPi}(\textit{part\_e1}, \textit{ptr\_e}) \vee \texttt{X-TPP}(\textit{part\_e1}, \textit{ptr\_e}) \vee  \notag\\ \texttt{X-TPPi}(\textit{part\_e1}, \textit{ptr\_e}))\wedge
(\texttt{Y-NTPP}(\textit{part\_e1}, \textit{ptr\_e}) \vee \texttt{Y-NTPPi}(\textit{part\_e1}, \textit{ptr\_e}) \vee  \notag\\ \texttt{Y-TPP}(\textit{part\_e1}, \textit{ptr\_e}) \vee \texttt{Y-TPPi}(\textit{part\_e1}, \textit{ptr\_e})) \notag
\end{gather}

\item In the case the predicate has only one parameter, a second object is needed to compute the truth-value of the predicate between the parameter and the other object. The authors remind the reader that RCC8 relations necessarily require two objects. In this case, the sensor system is tasked to retrieve the pose for each other object in the workcell to be used as the second object. Depending on the predicate, the search space for the other object can be narrowed down. For instance, the predicate \stvar{partstray-not-empty}(\const{ptr\_e}) (Table~\ref{tab:takepartpddl}) is true if and only if at least one part of type E is in the parts tray \const{ptr\_e}. It is not necessary to extend the search for the second object among the other types of object present in the workcell. It is not relevant, for instance, to check if the parts tray contains an end effector. On the other hand, to check the truth value of the predicate \stvar{robot-empty}(\const{robot\_1}) (Table~\ref{tab:takepartpddl}), which is true if and only if the robot \const{robot\_1} is not holding anything in the end effector attached to the robot, the predicate evaluation needs to scan a wider search space than the one described for the predicate \stvar{partstray-not-empty}(\const{ptr\_e}), i.e., the search space includes all parts of each type, all types of kit trays, etc. Once the search space has been defined, the external shape for each object in the search space is retrieved from the appropriate table from the \textsf{MySQL Database}. As described previously, the external shape is used by the sensor system to retrieve the 6DOF pose for the corresponding object. The truth-value of the set of logical rules that associate RCC8 relations to the intermediate state relation for the predicate is then computed for these two objects, that is the predicate parameter and the searched object.

\end{enumerate}

If all the predicates within the precondition section are true, the \textsf{MySQL Database} is updated with the current poses of these predicates' parameters. It is important that all the predicates are true in order to update the \textsf{MySQL Database}. This ensures that a complete (stable) state of the environment is stored in the \textsf{MySQL Database}. If at least one predicate is evaluated to false, it is considered a failure and the kitting process is terminated.

\subsection{Execute Action}
When all the predicates within the precondition of an action have been evaluated to true, this action is executed by the robot. To confirm that the action was successfully accomplished, the \textsf{Predicate Evaluation} process evaluates the predicates within the effect section for this action. The effect section consists of predicates that are expected to be true after performing a PDDL action. The evaluation of the predicates within the effect section is performed to confirm that these expectations are attained.

\subsection{Predicate Evaluation (Effects)}
The methodology previously described to evaluate the predicates within the precondition section of an action is also used to evaluate the predicates within the effect section of this action. This is depicted by \mycirc{C} in Figure~\ref{fig:sensor}. As mentioned earlier, all the predicates within the effect section must be evaluated to true to define the action as successful. In the case of a successful action, the next action within the \textsf{Plan Instance File} is processed. Once all the actions within the \textsf{Plan Instance File} have been executed by the robot, the kitting process is complete.

This paper describes the approach that uses a simulated sensor system to retrieve 6DOF pose estimations of objects of interest during kit building applications. The approach is mainly used during the predicate evaluation process. The use of a simulated sensor system allows the current kitting system to move towards an agile system where the current observations on parts and components are fed into the predicate evaluation process, i.e., before and after action executions.

It is also intended to apply contingency plans once an action failure occurs. One of the contingency plans is to re-plan from a state of the environment that is stable. Information on this stable state is retrieved from the \textsf{MySQL Database} that was updated from the latest poses of objects of interest. During the re-planning process, the new initial state becomes the stable state while the goal state stays unchanged.

As mentioned in Section \ref{subsection:DSI}, pose information coming from USARTruth is assumed to be perfect. In a future effort, the authors will attempt to present a model for the different sources of noise relative to each sensor-based pose estimation step (similar to the one described in~\cite{Meeden.GPC.1998}), and use measurements of real sensor data to validate the model. Noisy sensor data can be used during the simulation of failures and the application of contingency plans.


The current kitting workcell involves objects that are originally placed on a non-movable surface and also involves a pedestal-based articulated arm. To move towards an agile and flexible manufacturing system, the authors will need to address more challenging scenarios where parts come into the workcell via conveyor belts and where the robotic arm can be of gantry type. These new settings will need to be simulated and tested for kitting applications.



% Once a failure is detected and the failure mode is identified, one of the contingency plans is to re-plan from a known state of the environment
% Moreover, updating the MySQL database with the latest information from the kitting work cell can greatly help failure recovery. Once a failure occur, the latest stable version of the kitting work cell can be used as the new initial state of the environment to generate a new plan where the goal state stays unchanged.


%
%



%
% ---- Bibliography ----
%
\bibliographystyle{plain}
\bibliography{rita2014}
\clearpage
\addtocmark[2]{Author Index} % additional numbered TOC entry
\renewcommand{\indexname}{Author Index}
\printindex
\clearpage
\addtocmark[2]{Subject Index} % additional numbered TOC entry
\markboth{Subject Index}{Subject Index}
\renewcommand{\indexname}{Subject Index}
%\input{subjidx.ind}
\end{document}
