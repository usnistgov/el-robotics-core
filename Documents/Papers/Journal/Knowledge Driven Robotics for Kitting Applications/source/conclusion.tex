This article presented work that is 
part of the Industrial Subgroup of the IEEE Robotics and
Automation Society's Ontologies for Robotics and Automation Working
Group. A new knowledge driven design methodology and detailed model were
presented for the specific area of kit building.
The National Institute of Standards and Technology's
Knowledge Driven Planning and Modeling project that made
this work possible is
scheduled to continue for an additional two years. During this
time, we hope to improve on all aspects of the knowledge
representation and standardization effort. These improvements 
include increasing the scope of the knowledge model to
include areas such as assembly, increased outreach to industry, improvement of
test methods and metrics, and improvements of our ontology
and knowledge representation that will be fed to the IEEE
working group.

While a simulated testbed exists that employs the techniques
detailed in this article, we hope to prove the utility of our
methodology by migrating the work to an actual robotic cell.
We anticipate that changes in the representation and perhaps the
methodology will need to take place to cope with real-world considerations.
In addition, we hope to increase the scope of our simulated testbed to
include assembly operations. New actions, preconditions, and effects
will need to be explored and generated to support this more
complex domain.

We have  demonstrated the ability to build several varieties 
of kits without any change in the robot's programming. However, formal
test methods and metrics need to be developed with industry input so that
this technique may be compared with other robotic planning techniques. We
hope to work closely with the IEEE working group to formalize
these new measurement techniques.

The current draft of the knowledge model that has been submitted
to the IEEE working group does not include the notion of failures.
Therefore, future work also includes the 
representation of failures and of contingency plans in the ontology and the ability 
to run these plans based on detected failures.
