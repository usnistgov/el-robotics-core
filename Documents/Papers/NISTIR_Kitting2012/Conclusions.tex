The IPMAS project is scheduled to continue for an additional two years. During this time,
we hope to improve on all aspects of the knowledge representation and standardization
effort. These improvements include increased outreach to industry, improvement on
test methods and metrics, and improvements on our ontology and knowledge
representation that will be fed to the IEEE working group.

\subsection{Kitting Viewer Development Plans}

As mentioned earlier, the kittingViewer is far from complete. It is
planned to add the following capabilities.

\begin{itemize}

\item Add drawing the kitting workstation in its current state. The initial
  state of the workstation is already available in data as soon as the XML
  data file that describes it is read in.

\item Add updating the positions of objects as the robot executes commands.
  It will be necessary to compare the position of the robot with the
  positions of objects when OpenGripper and CloseGripper commands are
  executed in order to determine if the robot is grasping them.

\item Add metrics related to the positions of objects. This might include
  (1) the number of objects that should have been moved, (2) the number of
  objects that were moved, (3) the number of objects that were moved to the
  correct place, (4) the number of objects that were moved to the wrong
  place.

\item Add metrics related to constraint violations. This might include (1)
  the number of instances of picking up an object that weighs more than the
  robot's load capacity, (2) the number of instances of the robot moving
  outside of its work volume, (3) the number of instances of using a
  gripper to move an object when the gripper is not qualified to move the
  object. It will also be necessary to decide what the simulation should do
  in these cases and implement that.

\item Add a total score metric, and implement finding the total score using
  a configuration file in which the user assigns weights to the other
  metrics.
\end{itemize} 

\subsection{Knowledge Representation Development Plans}
We have created a knowledge driven system that is capable of building
kits in a flexible and agile manor assuming perfect actions. For this system
to be practical, this restriction must be removed. 
To enable this, our current work on the development of a taxonomy of
predicates for the situational awareness necessary for kit building will
be continued and expanded. The system will also be augmented to
allow for the checking of necessary preconditions before actions are
executed, and the verification of results after an action has occurred.

To date, we have developed a knowledge representation that supports
kitting operations. In cooperation with the IEEE Working Group, this
representation will be expanded to support general assembly operations.
In addition, we will work with the IEEE Working Group, academia, and 
industry to standardize the knowledge representations, test methods,
and metrics.