In order to maintain compatibility with the IEEE working group,
the ontology has been fully defined in OWL. However, due to
several difficulties defined below, the ontology was also fully defined
in the XML schema language. 
Although the two 
models are conceptually identical, there
are some systematic differences between the models (in addition to
differences inherent in using two different languages).

\begin{itemize}
\item The complexType names (i.e. class names) in XML schema have the
  suffix ``Type'' added which is not used in OWL. This is so that the same
  names without the suffix can be used in XML schema language as element
  names without confusion.

\item All of the XML schema complexTypes have a ``Name'' element that is
  not present in OWL. It is not needed in OWL because names are assigned as
  a matter of course when instances of classes are created.

\item As shown in Figure~\ref{fig:WorkstationModel}, the XML schema model has a list of ``Object"
elements. This collects all of the movable objects. The OWL model does not
have a corresponding list. In an OWL data file, the movable objects may
appear anywhere.

\item Attribute names in OWL have a prefix, as described below. The
  prefixes are not used in XML schema.
\end{itemize}

\subsection{OWL Specifics}
The kitting workstation model was defined first in OWL because the IEEE RAS
Ontologies for Robotics and Automation Working Group has decided to use
OWL, and the authors are participating in the activities of that working
group. OWL allows the use of several different syntaxes. The
functional-style syntax (which is the most compact one) has been used to
write the OWL version of the kitting workstation model.

In addition to having the model defined in OWL, OWL data files describing
specific initial states and goal states were defined in OWL, also using the
functional-style syntax.  Software tools were built in C++ and Java to work with the
OWL model and data files conforming to the model.

The initial intent has been to use OWL files for presenting the initial and
goal conditions for planning problems, and the authors have implemented a
planning system that uses OWL files.

The primary tool used by the OWL community for building and checking OWL
models and data files is named Prot\'{e}g\'{e} \cite{Horridge.2011}.  Prot\'{e}g\'{e} was
used for checking the kitting model and data files as they were
built.  Prot\'{e}g\'{e} continues to be used for checking the model and data files
whenever they are changed. The layout of the hierarchy in
Figure~\ref{fig:ClassHierarchy} is identical to what may be seen in
 Prot\'{e}g\'{e}'s class hierarchy window when the kitting model is loaded.

Several difficulties in working with OWL were encountered. The primary
effect of these difficulties was to make it essentially impossible to
debug OWL files.

Defining a model in OWL is quite different from defining the same model in
other information modeling languages with which the authors are intimately
familiar: C++, EXPRESS \cite{EXPRESSmanual}, and XML schema. Three of the
major differences involve (1) the assignment of attributes in classes, (2)
OWL's ``open world'' assumption, and (3) the distinction between model
files and data files.\\

\subsubsection{Class Attributes}
In other languages, assigning a typed attribute to a class requires a
single line of code. For example, the X attribute may be put into a
cartesian point class in XML schema language with
\newline \sf $<$xs:element name=``X'' type=``xs:decimal''/$>$\rm
\newline or in C++ with
\newline \sf double X; \rm
\newline or in EXPRESS with
\newline \sf X : REAL; \rm \newline
In these other languages, the name of the attribute is local to the class.
Hence, an attribute with a given name can appear in more than one class, and
there will be no confusion.

In OWL, there is no simple method of declaring a class attribute. Instead,
a property must be declared along with properties of the property. The
following lines are used in the OWL model to say that all points and only
points have an X attribute which is a decimal number.
\newline
\newline \sf Declaration(DataProperty(hasPoint\_X))
\newline DataPropertyDomain(:hasPoint\_X :Point)
\newline DataPropertyRange(:hasPoint\_X xsd:decimal)
\newline EquivalentClasses(:Point ObjectIntersectionOf(
\newline \hspace*{0.2in}DataSomeValuesFrom(:hasPoint\_X xsd:decimal)
\newline \hspace*{0.2in}DataAllValuesFrom(:hasPoint\_X xsd:decimal))) \rm
\newline
\newline
The \sf hasPoint \rm prefix used in the property name is not an OWL
requirement. It is one of several naming conventions for OWL being used by
the authors. The prefix is both for the benefit of a human reader (to make
it obvious that this is a property of a Point) and to differentiate this X
attribute from an X attribute of some other class (call it \sf Foo\rm)
which would have the prefix \sf hasFoo \rm.

As described above, with OWL it is necessary to make many statements in
order to build a class in a typical object-oriented style. OWL does not
assume a typical object-oriented style. It assumes the world might be more
complex than that. Hence, many OWL statements are required to produce
effects made in a few statements in other object-oriented languages. Having
to write a lot of statements is tedious but not a roadblock. A more serious
problem is that if a statement necessary to produce an object-oriented
effect is omitted, that is not an OWL error.  Prot\'{e}g\'{e} does not have an
object-oriented mode in which it will warn the user if a required statement
is missing. There are no OWL tools that will help with finding missing
statements. This is a debugging problem.

OWL was built so that it would support automated reasoning about the
relationships among properties, classes, and individuals.  Prot\'{e}g\'{e} allows
the use of several alternate automatic reasoners. In a typical
object-oriented style, there is no use for reasoning of that sort.
Everything useful to know about the relationships among properties,
classes, and individuals is already known. Hence having an automated
reasoning capability of the sort for which OWL was built is not useful
for the kitting model.\\

\subsubsection{Open World Assumption}
OWL makes an ``open world'' assumption.  In an open world, anything might
be true that is not explicitly declared false and is not inconsistent with
what has been declared true. This makes it easy for errors to go
unrecognized as such by  Prot\'{e}g\'{e} (or any other OWL tool). For example,
suppose the line \sf DataPropertyDomain(:hasPoint\_X :Point) \rm given
above is mistyped as \sf DataPropertyDomain(:hasPoint\_x :Point)\rm. When
 Prot\'{e}g\'{e} loads the file and the reasoner is started, no errors are
detected.  Prot\'{e}g\'{e} assumes that the DataPropertyDomain for \sf hasPoint\_X
\rm is unknown (that is not an error in OWL and  Prot\'{e}g\'{e}) and that there is a new
property named \sf hasPoint\_x \rm about which the only thing known is its
DataPropertyDomain (also not an error in OWL and  Prot\'{e}g\'{e}, even though there is no
explicit DataProperty declaration for the new property). The error can be
detected by a human by studying the list provided by selecting the
DataProperties tab in  Prot\'{e}g\'{e}. Similar errors, such as mistyping the name of an individual, are
similarly accepted without error in OWL and  Prot\'{e}g\'{e}, with similar effects.
The difficulties caused by the open world assumption would not occur if
 Prot\'{e}g\'{e} had a closed world mode, but it has none.\\

\subsubsection{Model Files vs. Data Files}
While other languages have different file formats for models and
data conforming to the models, OWL does not distinguish between model files
and data files.  Prot\'{e}g\'{e} does not provide any method of specifying that a
file is a model file or a data file. The conceptual difference is simple. 
Model files describe classes and data types (and, possibly,
constraints). Data files give information about individuals (instances of
one or more classes -- often called objects). The authors have made it a
practice to distinguish OWL model files from OWL data files. An OWL data
file can inadvertently change an OWL model, a bug that is very hard to
find. That cannot happen with EXPRESS or XML schema.\\

\subsubsection{Bugs in Files}
Since humans are error-prone, and the kitting OWL files were built by
humans, the OWL files had errors of the sort mentioned above. Some of these
errors were discovered when the OWL files were processed by the tools
developed for processing them and strange results were observed. Other
errors were found when a method of generating OWL data files automatically
from XML data files was developed, as described next.

\subsection{XML Specifics}
To better explore the pros and cons of various representations,
the authors are using XML schema and XML data files in parallel with the
corresponding OWL files.\\

\subsubsection{XML Tools}
Two automated tools developed by the authors are being used: an xml schema
parser (xmlSchemaParser) and a code generator (GenXMiller).  

The xmlSchemaParser reads an XML schema file, stores it in terms of
instances of C++ classes, and reprints the schema. When the xmlSchemaParser
runs, it performs many checks on the validity of the schema that is input
to it. The xmlSchemaParser handles almost all portions of the XML schema
syntax. A few of the rarely-used elements of syntax are not implemented.

The GenXMiller reads an XML schema and writes code for reading and writing
XML data files corresponding to that schema. The code that is generated
includes C++ classes (.hh and .cc files), a parser (YACC and Lex files) and
a stand-alone parser file in C++ that uses the other files.  The executable
utility produced by compiling a stand-alone parser reads and echoes any XML
data file corresponding to the schema. The GenXMiller is still under
development and currently handles only a subset of the XML schema language. 
The GenXMiller is not a
new type of system. Several other code generators that use an XML schema
as input have been developed \cite{XMLSchema1,ApacheXML}. 
Even more XML schema parsers are
available. However, having the knowledge about XML schema and XML data
files gained by developing that software and having an intimate knowledge
of the source code for it has proved very valuable in converting XML representations
to OWL representations.

The xmlSchemaParser and the GenXMiller use the same underlying parser,
which is built in YACC and Lex \cite{LexAndYACC}.

In addition to using the xmlSchemaParser and the GenXMiller, a commercial
XML tool named XMLSpy \cite{XMLSpyManual} has been used to check all XML
schemas and XML data files.\\

\subsubsection{Handling Kitting Data Files}
There is only one conceptual kitting model, but there are several kitting
data files corresponding to it. If the kitting model is used to represent
various starting and goal configurations, there
will be many more data files. Hence, the problem of generating bug-free
data files was tackled first.

An XML schema, kitting.xsd, was written by hand modeling the same
information as the OWL kitting workstation model, kittingClasses.owl. The
GenXMiller was then used to generate C++ classes and a parser for XML
kitting data files corresponding to kitting.xsd. The C++ classes that were
generated included code for printing XML kitting data
files. That code was rewritten by hand so that it prints OWL data files
rather than XML data files. The utility produced by compiling the code is
called the owlPrinter. To produce an OWL kitting data file, one writes an
XML kitting data file and runs it through the owlPrinter.

To determine that the owlPrinter works properly, it seems sufficient to
demonstrate that OWL data files generated automatically by the owlPrinter
from XML data files conforming to kitting.xsd contain exactly the same OWL
statements as are contained in manually prepared OWL data files intended to
contain the same information and conforming to kittingClasses.owl. This
demonstration was achieved as follows.

\begin{enumerate}[ (i) ]
\item Three XML data files were written manually containing the same
  information as three OWL data files. Each of the OWL files was at least
  1,100 lines (20 pages) long. Among the three there were statements of
  almost all of the types possible under the kittingClasses.owl model. It was
  decided, therefore, that successful performance for these three files
  would be an adequate test.
\item The three XML data files were run through the owlPrinter to produce
  three OWL files.
\item Since the owlPrinter has a different approach to ordering OWL
  statements than was taken in preparing OWL files manually, and a slightly
  different method of formatting statements, two small utilities were
  written to enable file comparison. The first utility, compactOwl, reads
  an OWL file and writes an OWL file containing the same statements but
  with blank lines and comments removed, and with each statement on a
  single line. For each pair of matching OWL files (manually written and
  automatically generated), compactOwl was used to generate a corresponding
  pair of compacted OWL files. The second utility, compareOwl, reads each
  of a pair of OWL files, alphabetizes the statements from each of them on
  two saved lists, and then goes through the two lists checking that the
  n$^{th}$ line of one list is identical to the n$^{th}$ line of the other list.
  CompareOwl was used to compare each of the three sets of pairs of
  compacted files.
\item While the tests just described were being made, changes were made to
  correct errors in the manually written XML and OWL data files being
  tested and in the code for the owlPrinter. The tests revealed errors in
  all three types of files.
\end{enumerate}

After the testing just described was complete, using the owlPrinter
another OWL data file was prepared from a manually written XML data file
for which there was no manually written OWL counterpart. The automatically
generated OWL data file was checked in  Prot\'{e}g\'{e} and no errors were reported.

OWL data files
may now be prepared with much less likelihood of human error for the following
reasons.
\begin{itemize}

\item Property names and names of individuals will not be misspelled.

\item Statements will not be accidentally omitted.

\item Validity checks made in the kittingParser and XMLSpy will do a
  better job of detecting errors in XML data files. For example, required
  attributes that are missing will be detected.

\end{itemize}

\subsubsection{Handling the Kitting Model}
As described above, the equivalent model files kitting.xsd and
kittingClasses.owl were both prepared manually. If changes to the kitting
model are made, it will be necessary to change both of those files and the
code for the owlPrinter. It would be good to have kitting.xsd as the
primary source file for the model and to generate kittingClasses.owl
automatically from it. The authors believe this is possible and have
started working on it. The work is not yet complete, but no roadblocks are
anticipated. The approach being using is to modify the printer code in the
xmlSchemaParser so that it prints an OWL class file rather than an XML
schema file.

It would also be desirable to be able to modify the owlPrinter
automatically if the kitting model is changed. Doing that is a
substantially more difficult task than the other two automatic conversions,
and the authors are not planning to attempt it. The approach would be to
modify the GenXMiller so that the code it generates automatically would
read XML data files and automatically generate OWL data files.
