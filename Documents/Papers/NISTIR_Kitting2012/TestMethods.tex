Acording to the American Society for Testing and Materials (ASTM) \cite[p. vii]{ASTM99}, a test method is a
definitive procedure that produces a test result. It is the authors' desire to develop test methods that will
lead to a better understanding of what it means have an ``agile" and ``flexible" system and metrics that 
will allow for the measurement of a system's agility and flexibility. We have chosen to begin our study
with the domain of kit building since it is a greatly simplified manufacturing/assembly domain. However, 
even in the domain of kit building, there is a large amount of variance that must be accounted for. For
example, will parts be rigid or flexible? Will a vacuum gripper or a parallel jaw mechanism or a fingered 
gripper be utilized for part picking? Will parts be picked from a tray or a bin? What aspects of the
process will be stressed to demonstrate flexibility and agility?

In keeping with the idea of reduced complexity, we will strive to come up with test methods that stress
the system's agility and not the system's robotic configuration or abilities. As such, we will utilize
rigid objects that are of simple shape (rectilinear) and a vacuum effector for the handling of the parts.