Acording to the American Society for Testing and Materials (ASTM) \cite[p. vii]{ASTM99}, a test method is a
definitive procedure that produces a test result. It is the authors' desire to develop repeatable test methods that will
lead to a better understanding of what it means have an ``agile" and ``flexible" planning system and metrics that 
will allow for the measurement of a system's agility and flexibility. We have chosen to begin our study
with the domain of kit building since it is a greatly simplified manufacturing/assembly domain. However, 
even in the domain of kit building, there is a large amount of variance that must be accounted for. For
example, will parts be rigid or flexible? Will a vacuum gripper, or a parallel jaw mechanism, or a fingered 
gripper be utilized for part picking? Will parts be picked from a tray or a bin? What aspects of the
process will be stressed to demonstrate flexibility and agility?

In keeping with the ideas of reduced complexity and repeatability, we will strive to come up with test methods that stress
the system's agility and not the system's robotic configuration or abilities. As such, we make the following assumptions:
\begin{itemize}
	\item All of the contents of the kit will be rigid objects.
	\item All of the rigid objects are of simple shape (rectilinear) and have a flat surface for a top.
	\item A vacuum effector is utilized for the handling of the parts.
	\item All of the parts will be located in a well-defined parts tray. This will allow for repeatable experiments in terms of part placement.
	\item All of the part's initial and final positions will be within the reach volume of the robot.
	\item There are no obstacles located within the robot's reach volume.
\end{itemize}

\subsection{Test Requirements}
The test methods themselves are designed as a series of tests that have increasing complexity. It is assumed that the tests will be performed
in order, and that a system that is not capable of performing test {\it n} will fail at test {\it n+1} as well. For all of the tests,
the initial condition of the world and the goal kit configurations are provided as XML and/or OWL files conforming with the IEEE RAS Ontologies
for Robotics and Automations Working Group's kitting ontology. The planning system is required to produce an output that
will construct the required kit(s) from the initial condition. All of the commands must be in the form of CRCL, and the system
is allowed to submit new plans that respond to environmental changes.

The parts supply consists of one or more trays of raw materials and
the kit tray is a flat container with separators between locations for individual parts. The part supply trays may be auto-filling (e.g. 
a single location that is continuously fed with a part) or limited quantity trays. Test methods may be configured to require end-of-tool
changes for the picking of various parts.

\subsection{Basic Kit}
The first test method is the construction of a single basic kit. The kit will contain two or more different types of parts. The design of the
kit and parts supply will be known before the test begins. The actual location of the kit tray and parts supplies will not be known before
runtime. The location and orientation (yaw) of the kit and parts supplies will be varied during consecutive runs of the test method.
The planning system is required to submit a single CRCL formated plan that will construct the given kit. There are no execution errors.
Each kit construction will be evaluated by our standard metrics as described in Section \ref{sect:Metrics}. 
This test method will evaluate the following aspects of agility and flexibility:
\begin{itemize}
	\item Ability to correctly build a specific predefined kit.
	\item Agility in terms of part tray and kit placement. Will show that exact fixturing is not necessary for the construction
	of a kit.
	\item If the robot is capable of supporting tool changing, different end-of-arm tooling may be required for grasping
	the various parts.
\end{itemize}

\subsection{New Variety of Basic Kit}
This test provides the system with a never before seen kit variation. The variation will be delivered in the previously
mentioned IEEE RAS OWL/XML format. In addition to the standard kit construction metrics, the time from the receipt
of the new kit configuration to the start of first construction will be recorded. The amount of down-time for the
cell will also be noted. This test will measure the agility of the robot cell in coping with new kit varieties.

\subsection{Basic Kit, Multiple Varieties}
This test builds on the previous test by requiring the construction of two or more kit varieties in a pseudo-random ordering.
The kit configurations will be known before the start of the test. 
This test method will evaluate the following additional aspects of agility and flexibility:
\begin{itemize}
	\item Ability to correctly build several kits with varying part placements without manual intervention.
	This will demonstrate agility in terms of kit layout and contents.
	\item Ability to manipulate items of varying size and weight. This test method will require the workcell
	to move empty kit trays from storage to a construction location and then to move finished kits to
	a bin of finished kits.
	\item If the robot is capable of supporting tool changing, different end-of-arm tooling may be required for 
	kit manipulation. 
\end{itemize}

\subsection{Construction Errors}
This test will evaluate the system's ability to recover from predictable error conditions. These conditions will include
dropped parts and parts with detected defects. It is expected that the planning systems will interrupt the
execution of the kit build in order to provide updated plans to cope with the unexpected events.