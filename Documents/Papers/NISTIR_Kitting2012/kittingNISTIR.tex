%%%%%%%%%%%%%%%%%%%%%%%%%%%%%%%%%%%%%%%%%%%%%%%%%%%%%%%%%%%%
%2345678901234567890123456789012345678901234567890123456789012345678901234567890
%        1         2         3         4         5         6         7         8

\documentclass[letterpaper, 10 pt, conference]{ieeeconf}  % Comment this line out
                                                          % if you need a4paper
%\documentclass[a4paper, 10pt, conference]{ieeeconf}      % Use this line for a4
                                                          % paper

\IEEEoverridecommandlockouts                              % This command is only
                                                          % needed if you want to
                                                          % use the \thanks command
\overrideIEEEmargins
% See the \addtolength command later in the file to balance the column lengths
% on the last page of the document



% The following packages can be found on http:\\www.ctan.org
%\usepackage{graphics} % for pdf, bitmapped graphics files
%\usepackage{epsfig} % for postscript graphics files
%\usepackage{mathptmx} % assumes new font selection scheme installed
%\usepackage{times} % assumes new font selection scheme installed
%\usepackage{amsmath} % assumes amsmath package installed
%\usepackage{amssymb}  % assumes amsmath package installed
\usepackage{graphicx}
\usepackage{verbatim}
\usepackage{multirow}
\usepackage{rotating}
\usepackage{moreverb}                    % for boxedboxedverbatim
\usepackage{array}
\usepackage{fancyvrb}
\usepackage{multicol}
\usepackage{mdwlist}
\usepackage{enumerate}
%\usepackage{tikz-er2}


\newtheorem{defn}{Definition}

\newcommand{\class}[1] {\textit{#1}}
\newcommand{\const}[1] {$\mathit{#1}$}
\newcommand{\objvar}[1] {$\mathsf{#1}$}
\newcommand{\stvar}[1] {\textsf{#1}}
\newcommand{\op}[1] {\textsl{#1}}
\newcommand{\nil} {\textit{nil}\ }

\newcommand\T{\rule{0pt}{2.6ex}}
\newcommand\B{\rule[-1.2ex]{0pt}{0pt}}

%\definecolor{violetred}{cmyk}{0,0.85,0.31,0.18}
%\definecolor{darkblue}{cmyk}{1,1,0,0.45}
%\definecolor{lavenderblush4}{cmyk}{0,0.06,0.04,0.45}
%\definecolor{packergreen}{cmyk}{0.46,0,0.21,0.76}
%\definecolor{fuschia}{cmyk}{0,100,0,0}
%\definecolor{graycmyk}{cmyk}{0,0,0,0.74}
%
%\usetikzlibrary{calc,trees,positioning,arrows,chains,shapes.geometric,%
%    decorations.pathreplacing,decorations.pathmorphing,shapes,%
%    matrix,shapes.symbols,positioning,shadows}
%
%% styles for flowcharts
%
%\tikzstyle{every entity} = [top color=white, bottom color=blue!30,
%                            draw=blue!50!black!100, drop shadow]
%\tikzstyle{empty} = [top color=white, bottom color=white,
%                            draw=white]
%
%\tikzstyle{every weak entity} = [drop shadow={shadow xshift=.7ex,
%                                 shadow yshift=-.7ex}]
%\tikzstyle{every attribute} = [top color=white, bottom color=green!20,
%                               draw=green, node distance=1cm, drop shadow]
%\tikzstyle{ELLIPSE} = [draw, ellipse, top color=white, bottom color=green!20, draw=green, drop shadow]
%\tikzstyle{MainAttribute} = [draw, rectangle,top color=white, bottom color=red!20,
%                               draw=red, node distance=1cm, drop shadow]
%\tikzstyle{DATABASE} = [draw, rectangle, rounded corners,top color=white, bottom color=graycmyk!50, draw=graycmyk, inner sep=10pt, drop shadow={shadow xshift=.7ex, shadow yshift=-.7ex}]
%
%
%\tikzstyle{myarrow}=[->, >=stealth', thick, shorten <=2pt,shorten >=2pt]
%
%
%
%\tikzstyle{output} = [draw, rectangle, rounded corners,top color=white, bottom color=red!30, draw=red, inner sep=10pt, drop shadow={shadow xshift=.7ex, shadow yshift=-.7ex}]
%
%\tikzstyle{abstract}=[rectangle, draw=black, rounded corners, fill=blue!40, drop shadow,
%        text centered, anchor=north, text=white, text width=3cm]
%
%
%\newbox{\LegendOutput}
%\savebox{\LegendOutput}{
%    (\begin{tikzpicture}[]
%    \node[output] (2) {\hspace{5 mm}};
%    \end{tikzpicture}
%    )}


\newenvironment{mylisting}
{\begin{list}{}{\setlength{\leftmargin}{1em}}\item\small}
{\end{list}}

\newenvironment{mytinylisting}
{\begin{list}{}{\setlength{\leftmargin}{1em}}\item\tiny\bfseries}
{\end{list}}


\title{\LARGE \bf
Metrics and Test Methods for Industrial Kit Building
}

%\author{ \parbox{3 in}{\centering Stephen Balakirsky\\
%         Intelligent Systems Division\\
%         National Institute of Standards and Technology\\
%        Gaithersburg, MD 20899, USA\\
%         {\tt\small stephen.balakirsky@nist.gov}}
%         \hspace*{ 0.5 in}
%         \parbox{3 in}{ \centering Zeid Kootbally\\
%          Department of Mechanical Engineering \\
%         University of Maryland\\
%         College Park, MD 20742, USA\\
%         {\tt\small zeid.kootbally@nist.gov}}\\ \\
%	\parbox{2.25 in}{\centering Craig Schlonoff\\
%         Intelligent Systems Division\\
%         National Institute of Standards and Technology\\
%        Gaithersburg, MD 20899, USA\\
%         {\tt\small craig.schlenoff@nist.gov}}
%        \hspace*{0.05in}
%         \parbox{2.25 in}{ \centering Thomas Kramer\\
%          Department of Mechanical Engineering \\
%         Catholic University of America\\
%         Washington, DC 20064, USA\\
%         {\tt\small thomas.kramer@nist.gov}}
%        \hspace*{0.05in}
%         \parbox{2.25 in}{ \centering Satyandra K. Gupta\\
%          Maryland Robotics Center\\
%         University of Maryland\\
%         College Park, MD 20742, USA\\
%         {\tt\small skgupta@umd.edu}}\\
%}

\author{Stephen Balakirsky, Thomas Kramer, and Zeid Kootbally% <-this % stops a space
\thanks{S. Balakirsky is with the Intelligent Systems Division, National Institute of Standards and Technology, Gaithersburg, MD, USA (e-mail:stephen.balakirsky@nist.gov)}% <-this % stops a space
\thanks{Z. Kootbally is with the Department of Mechanical Engineering, University of Maryland, College Park, MD, USA (email: zeid.kootbally@nist.gov)}%
\thanks{T. Kramer is with the Department of Mechanical Engineering, Catholic University of America, Washington, DC, USA (email: thomas.kramer@nist.gov)}%
} 


\begin{document}
\begin {center}
{\huge{Outline}}
\end {center}

\begin{enumerate}
\item Description of kitting
\item Test methods
	\begin{enumerate}
	\item Inputt format (xml, owl)
	\item Output format (Robot canonical command language)
	\item method itself (build kits) with varying part mix and configuration, missing parts, ...
	\end{enumerate}
\item metrics
	\begin{enumerate}
	\item Plan
		\begin{enumerate}
		\item number of steps
		\item goal achievement
		\item optimality
		\item maintain constraints
		\item number of errors
		\end {enumerate}
	\item Execution
	\end {enumerate}
\item assumptions
	\begin {enumerate}
	\item Work volume totally reachable
	\end {enumerate}
\end{enumerate}


\maketitle
\thispagestyle{empty}
\pagestyle{empty}


%%%%%%%%%%%%%%%%%%%%%%%%%%%%%%%%%%%%%%%%%%%%%%%%%%%%%%%%%%%%
\begin{abstract}

The IEEE RAS Ontologies for Robotics and Automation Working Group is dedicated to developing a methodology for knowledge representation and reasoning in robotics and automation. As part of this working group, the Industrial Robots sub-group is tasked with studying industrial applications of the ontology. One of the first areas of interest for this subgroup is the area of kit building or kitting. It is anticipated that
utilization of the ontology will allow for the development of higher performing kitting systems. However, the definition of "higher performing"
has yet to be defined. This paper addresses this issue by providing the basis for performance methods and metrics that are designed to
determine the performance of a kitting system.
\end{abstract}


%%%%%%%%%%%%%%%%%%%%%%%%%%%%%%%%%%%%%%%%%%%%%%%%%%%%%%%%%%%%
\section{INTRODUCTION}
Material feeding systems are an integral part of today's assembly line operations. These systems assure that pars are available where and when 
they are needed during the assembly operations by providing either a continuous supply of parts at the station, or a set of parts (known
as a 'kit') that contains the required parts for one or more assembly operations. In continuous supply, a quantity of each part that
may be necessary for the assembly operation are stored at the assembly station. If multiple versions of a product are being assembled,
(mixed-model assembly)
a larger variety of parts than are used for an individual assembly may need to be stored. With this material feeding scheme, parts
storage and delivery systems must be duplicated at each assembly station.

An alternative approach to continuous supply is know as kitting. In kitting, parts are delivered to the assembly station in kits that contain
the exact parts necessary for the completion of one assembly object. According to Bozer and McGinnis \cite{Bozer1992} "A kit is a specific
collection of components and/or subassemblies that together
(i.e., in the same container) support one or more assembly operations for a given product or shop order". In the case of mixed-model
assembly, the contents of a kit may vary from product to product.
Kitting itself may be viewed as a specialization of the general bin-picking problem. The use of kitting allows a single delivery system to feed
multiple assembly stations.

In industrial assembly of manufactured products, kitting is often performed prior to final assembly. Manufacturers utilize kitting
due to its ability to provide cost savings \cite{Carlsson_2008} including saving manufacturing or assembly space \cite{Medbo2003}, reducing assembly workers walking and searching times \cite{Schwind1992}, and increasing line flexibility \cite{Bozer1992} and balance \cite{Jiao2000}.

Several different techniques are used to create kits. A kitting operation where a kit box is stationary until filled at a single
kitting workstation is referred to as {\it batch kitting}. In {\it zone kitting}, the kit moves while being filled and will pass through one or
more zones before it is completed. This paper focuses on batch kitting processes.

In batch kitting, the kit's component parts may be staged in containers positioned in the workstation or may arrive on a conveyor.
Component parts may be fixtured, for example placed in compartments on trays, or may be in random orientations, for example
placed in a large bin. In addition to the kit's component parts, the workstation usually contains a storage area for empty kit boxes as
well as completed kits.

Kitting has not yet been automated in many industries where automation may be feasible. Consequently, the cost of building kits is higher than it could be. We are addressing this problem by proposing performance methods and metrics that will allow for the unbiased comparison of 
various approaches to building kits  in an agile manufacturing environment. The performance methods that we propose must be simple enough to be repeatable at a variety of 
testing locations, but must also capture the complexity inherent to variants of kit building. The test methods must address concerns such as
measuring performance against variations in kit contents, kit layout, and component supply.
For our test methods, we assume that a robot performs a series of pick-and-place operations
in order to construct the kit. These operations include:
\begin{enumerate}
\item Pick empty kit and place on work table.
\item Pick multiple component parts and place in kit.
\item Pick completed kit and place in full kit storage area.
\end{enumerate}
Each of these actions may be a compound action that includes other actions such as end-of-arm tool changes, path planning,
and obstacle avoidance.

It should be noted that multiple kits may be built simultaneously. Finished kits are moved to the assembly floor where components
are picked
from the kit for use in the assembly procedure. The kits are normally designed to facilitate component picking in the correct
sequence for assembly. Component orientation may be constrained by the kit design in order to ease the pick-to-assembly process.
Empty kits are returned to the kit building area for reuse.

%Although the knowledge requirements described in the previous paragraph have been identified for the kitting domain, they are clearly applicable to many types of industrial robot applications. As such, we expect that these knowledge requirements will serve as the basis for the industrial robot ontology being developed in the IEEE  RAS Ontologies for Robotics and Automation Working Group \cite{Madhavan2011} (henceforth referred to as the IEEE WG). Throughout the process of developing the kitting ontology, the group will constantly look at the applicability of the requirements outside of kitting and move the pertinent knowledge ``up'' the ontology (whether in the portion that models the kitting sub-domain, the industrial robot domain, or the upper ontology), as appropriate.

%In keeping with the standards philosophy of the IEEE WG, we require the models being developed to be as widely applicable as possible. %Therefore, we have
%created a layered model abstraction where users may adopt as many of the layers of the abstraction as make sense for their
%specific application.  The architecture shown in Figure , though developed for the kitting ontology, can be equally applicable to the %implementation of any type of formal manufacturing knowledge representation. Said in a different way, the implementor can plug in a knowledge representation for a different domain and the architecture would still be valid. In a similar manner, different planning language abstractions could be utilized in the planning language layer of the
%abstraction and different planning/execution systems could be utilized in the robot language layer of the abstraction.



%%%%%%%%%%%%%%%%%%%%%%%%%%%%%%%%%%%%%%%%%%%%%%%%%%%%%%%%%%%%


\section{Architecture Description}
\label{sect:Architecture}


\addtolength{\textheight}{-15cm}   % This command serves to balance the column lengths
                                  % on the last page of the document manually. It shortens
                                  % the textheight of the last page by a suitable amount.
                                  % This command does not take effect until the next page
                                  % so it should come on the page before the last. Make
                                  % sure that you do not shorten the textheight too much.

%%%%%%%%%%%%%%%%%%%%%%%%%%%%%%%%%%%%%%%%%%%%%%%%%%%%%%%%%%%%%%%%%%%%%%%%%%%%%%%%

%%%%%%%%%%%%%%%%%%%%%%%%%%%%%%%%%%%%%%%%%%%%%%%%%%%%%%%%%%%%
\section{CONCLUSIONS AND FUTURE WORK}
\label{sect:Conclusions}


\bibliographystyle{plain}
\bibliography{kitting}
\end{document}
