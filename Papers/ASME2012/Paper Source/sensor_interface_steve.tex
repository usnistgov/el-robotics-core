\subsection*{Sensor Interface} 
ROS provides a rich vocabulary of sensor interface messages. The {\it RosSim} node strives to automatically match simulated sensors to the appropriate ROS topic. Currently, USARSim's inertial navigation, ground truth, and LADAR sensors are supported. These sensors automatically join the robot transform tree and publish their sensor messages at the rate that the {\it RosSim} node receives the sensor output. It is the intent of the authors to implement the full array of USARSim's sensors as time and resources permit.

The USARSim/ROS interface allows one to utilize known, published algorithms with simulated sensors and environments. However, the computational expense of the sensor processing must still be carried by the target hardware. One benefit of simulation is that one can not only simulate raw sensor output, but also the results from complex sensor processing tasks. One such example is the USARSim object recognition sensor. This sensor  is simulated in much the same manner as a laser scanner. However, instead of reporting the range that each beam travels, the sensor accumulates the number of beam hits that occur on each detected object. The number of hits, along with the percentage of the object that is visible may then be used to determine the amount of noise to add to the objects position and recognized type. This information may then be sent over standard ROS topics, without incurring the overhead burden of running the actual object and pose recognition algorithms.
