%%%%%%%%%%%%%%%%%%%%%%%%%%%%%%%%%%%%%%%%%%%%%%%%%%%%%%%%%%%%
%2345678901234567890123456789012345678901234567890123456789012345678901234567890
%        1         2         3         4         5         6         7         8

%\documentclass[letterpaper, 10 pt, conference]{ieeeconf}  % Comment this line out
                                                          % if you need a4paper
\documentclass[a4paper, 10pt, conference]{ieeeconf}      % Use this line for a4
                                                          % paper

\IEEEoverridecommandlockouts                              % This command is only
                                                          % needed if you want to
                                                          % use the \thanks command
\overrideIEEEmargins
% See the \addtolength command later in the file to balance the column lengths
% on the last page of the document



% The following packages can be found on http:\\www.ctan.org
%\usepackage{graphics} % for pdf, bitmapped graphics files
%\usepackage{epsfig} % for postscript graphics files
%\usepackage{mathptmx} % assumes new font selection scheme installed
%\usepackage{times} % assumes new font selection scheme installed
%\usepackage{amsmath} % assumes amsmath package installed
%\usepackage{amssymb}  % assumes amsmath package installed
\usepackage{graphicx}
\usepackage{verbatim}
\usepackage{multirow}
\usepackage{rotating}
\usepackage{moreverb}                    % for boxedboxedverbatim
\usepackage{array}
\usepackage{fancyvrb}
\newtheorem{defn}{Definition}

\newcommand{\class}[1] {\textit{#1}}
\newcommand{\const}[1] {$\mathit{#1}$}
\newcommand{\objvar}[1] {$\mathsf{#1}$}
\newcommand{\stvar}[1] {\textsf{#1}}
\newcommand{\op}[1] {\textsl{#1}}
\newcommand{\nil} {\textit{nil}\ }

\newcommand\T{\rule{0pt}{2.6ex}}
\newcommand\B{\rule[-1.2ex]{0pt}{0pt}}




\newenvironment{mylisting}
{\begin{list}{}{\setlength{\leftmargin}{1em}}\item\small}
{\end{list}}

\newenvironment{mytinylisting}
{\begin{list}{}{\setlength{\leftmargin}{1em}}\item\tiny\bfseries}
{\end{list}}


\title{\LARGE \bf
An  Industrial Robotic Knowledge Representation for Kit Building Applications
}

\author{ \parbox{3 in}{\centering Stephen Balakirsky\\
         Intelligent Systems Division\\
         National Institute of Standards and Technology\\
        Gaithersburg, MD 20899, USA\\
         {\tt\small stephen.balakirsky@nist.gov}}
         \hspace*{ 0.5 in}
         \parbox{3 in}{ \centering Zeid Kootbally\\
          Department of Mechanical Engineering \\
         University of Maryland\\
         College Park, MD 20742, USA\\
         {\tt\small zeid.kootbally@nist.gov}}\\ \\
	\parbox{2.25 in}{\centering Craig Schlonoff\\
         Intelligent Systems Division\\
         National Institute of Standards and Technology\\
        Gaithersburg, MD 20899, USA\\
         {\tt\small craig.schlenoff@nist.gov}}
        \hspace*{0.05in}
         \parbox{2.25 in}{ \centering Thomas Kramer\\
          Department of Mechanical Engineering \\
         Catholic University of America\\
         Washington, DC 20064, USA\\
         {\tt\small thomas.kramer@nist.gov}}
        \hspace*{0.05in}
         \parbox{2.25 in}{ \centering Satyandra K. Gupta\\
          Maryland Robotics Center\\
         University of Maryland\\
         College Park, MD 20742, USA\\
         {\tt\small skgupta@umd.edu}}\\
}

%\author{Author1 and Author2% <-this % stops a space
%\thanks{This work was not supported by any organization}% <-this % stops a space
%\thanks{H. Kwakernaak is with Faculty of Electrical Engineering, Mathematics and Computer Science,
%        University of Twente, 7500 AE Enschede, The Netherlands
%        {\tt\small h.kwakernaak@autsubmit.com}}%
%\thanks{P. Misra is with the Department of Electrical Engineering, Wright State University,
%        Dayton, OH 45435, USA
%        {\tt\small pmisra@cs.wright.edu}}%
%} 