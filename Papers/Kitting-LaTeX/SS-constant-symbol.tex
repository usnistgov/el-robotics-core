\subsection{Constants Symbols}\label{constant_symbol}

{\textit{Constant symbols}}~\cite{NAU.2004}, also called {\textit{object symbols}} are partitioned into disjoint classes corresponding to the objects of the domain. Let $D$ be the finite set of all constant symbols in the kitting planning domain. $D$ is partitioned into the following sets of constant symbols.
%An abstract version of the kitting domain can be defined by the following finite sets of constant symbols:

\begin{itemize}
\item A set of {\textit{robots}}$\lbrace${\constant{r1, r2, \ldots}}$\rbrace$: Robotic arms that can move objects in the workstation in order to build kits.
\item To build a kit, the different types of kit trays are used as illustrated in the following sequence:\\ \textit{empty kit tray}$\rightarrow$\textit{partially full kit tray}$\rightarrow$\textit{finished kit tray}.\\ The different types of kit trays are the ones described below.
\begin{itemize}
\item A set of {\textit{empty kit trays}}$\lbrace${\constant{ekt1, ekt2, \ldots}}$\rbrace$: Empty kit trays do not contain parts yet. Once an empty kit tray contains at least one part, it becomes a \textit{partially full kit tray}.
\item A set of {\textit{partially full kit trays}}$\lbrace${\constant{pfkt1, pfkt2, \ldots}}$\rbrace$: A partially full kit tray contains some parts but not all the parts that are necessary to build a kit. Once a partially full kit tray has all the parts required for a kit, the partially full kit tray becomes a \textit{finished kit tray}.
\item A set of {\textit{finished kit trays}}$\lbrace${\constant{fkt1, fkt2, \ldots}}$\rbrace$: A finished kit tray contains all the parts that constitute a kit.
\item A set of {\textit{part supplies}}$\lbrace${\constant{ps1, ps2, \ldots}}$\rbrace$: Part supplies may be a tray or box with parts inside or a box containing trays with parts. Multiple part supplies can be located in the workstation, and each part supply contains a specific type of part.
\end{itemize}
\item A set of {\textit{parts}}$\lbrace${\constant{pA-1, pA-2,\ldots, pB-1, pB-2,\ldots, pC-1, pC-2,\ldots}}$\rbrace$: A component that goes in a kit. Parts are delivered to the workstation in part supplies. A kit can have multiple parts of the same type (from the same part supply), thus to differentiate these parts from part supplies A, B, and C for example, parts from part supply A are noted {\constant{pA-1, pA-2,\ldots}}, parts from part supply B are noted {\constant{pB-1, pB-2,\ldots}}, and parts from part supply C are noted {\constant{pC-1, pC-2,\ldots}}.

\item A set of {\textit{grippers}}$\lbrace${\constant{g1, g2, \ldots}}$\rbrace$: Device at the end of the robotic arm used to grasp objects in the workstation. Two types of grippers exist for the kitting domain:
    \begin{enumerate}
    \item Grippers that hold parts.
    \item Grippers that hold empty and finished kit trays (empty and finished kit trays are the same kit trays).
    \end{enumerate}

\item A set of {\textit{boxes of finished kit trays}}$\lbrace${\constant{boxfkt1, boxfkt2, \ldots}}$\rbrace$: A box that contains only finished kit trays.

\item A set of {\textit{boxes of empty kit trays}}$\lbrace${\constant{boxekt1, boxekt2, \ldots}}$\rbrace$: A box that contains only empty kit trays.

\item A symbol {\textit{work table}}$\lbrace$\wtable$\rbrace$: a surface used to build kits. There is no need to have separate symbols for individual work tables because each work table is uniquely identified by the workstation it is in.

\item A symbol {\textit{changing station}}$\lbrace$\chstation$\rbrace$: a surface that contains grippers not in use. There is no need to have separate symbols for individual changing stations because each changing station is uniquely identified by the workstation it is in. The changing station can be placed on the work table or on a separate surface.
\end{itemize} 