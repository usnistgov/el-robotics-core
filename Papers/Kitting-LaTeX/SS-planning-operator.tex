\subsection{Planning Operators}

\begin{defn} A planning operator is a triple \textit{o=(name(o), precond(o), effects(o))}
where:
\begin{itemize}
\item \textit{name(o)} is a syntactic expression of the form $n(u_1,\dots,u_k)$, where $n$ is a symbol
called an operator symbol, $u_1,\dots,u_k$ are all of the object variable symbols that
appear anywhere in \textit{o}, and $n$ is unique (i.e., no two operators can have the
same operator symbol).
\item \textit{precond(o)} is a set of expressions on state variables and relations.
\item \textit{effects(o)} is a set of assignments of values to state variables of the form
$x(t_1,\dots,t_k)\leftarrow t_{k+1}$, where each $t_i$ is a term in the appropriate range.
\end{itemize}
\end{defn}

Considering the domain for a kitting station, the operators on this domain are defined as follows:

\begin{itemize}
\renewcommand{\labelitemi}{$\blacksquare$}
\renewcommand{\labelitemii}{$\square$}

\item \textit{takeEkt} (\robot, \ekt, \boxekt)
\begin{itemize}
\item ;;The robot \robot takes the empty kit tray \ekt from the box of empty kit trays \boxekt.
\item \textit{precond}: \textit{ektloc}(\ekt)=\boxekt, \textit{rhold}(\robot)=\nil
\item \textit{effects}: \textit{ektloc}(ekt)=\robot, \textit{rhold}(\robot)=\ekt
\end{itemize}

\item \textit{putEkt} (\robot, \ekt, \wtable)
\begin{itemize}
\item ;;The robot \robot puts the empty kit tray \ekt on the work table \wtable.
\item \textit{precond}: \textit{ektloc}(\ekt)=\robot, \textit{rhold}(\robot)=\ekt
\item \textit{effects}: \textit{ektloc}(\ekt)=\wtable, \textit{rhold}(\robot)=\nil
\end{itemize}

\item \textit{takeFkt} (\robot, \fkt, \wtable)
\begin{itemize}
\item ;;The robot \robot takes the full kit tray \fkt from the work table \wtable.
\item \textit{precond}: \textit{fktloc}(\fkt)=\wtable, \textit{rhold}(\robot)=\nil
\item \textit{effects}: \textit{fktloc}(\fkt)=\robot, \textit{rhold}(\robot)=\fkt
\end{itemize}

\item \textit{putFkt} (\robot, \fkt, \boxekt)
\begin{itemize}
\item ;;The robot \robot puts the full kit tray \fkt in the box of full kit trays \boxfkt.
\item \textit{precond}: \textit{ektloc}(\fkt)=\robot, \textit{rhold}(\robot)=\fkt
\item \textit{effects}: \textit{ektloc}(\fkt)=\boxfkt, \textit{rhold}(\robot)=\nil
\end{itemize}

\item \textit{takeP} (\robot, \parts, \partsup)
\begin{itemize}
\item ;;The robot \robot takes the part \parts from the part supply tray \partsup.
\item \textit{precond}: \textit{ploc}(\parts)=\partsup, \textit{rhold}(\robot)=\nil
\item \textit{effects}: \textit{ploc}(\parts)=\robot, \textit{rhold}(\robot)=\parts
\end{itemize}

\item \textit{putP} (\robot, \parts, \kt)
\begin{itemize}
\item ;;The robot \robot puts the part \parts in the kit tray \kt.
\item \textit{precond}: \textit{ploc}(\parts)=\robot, \textit{rhold}(\robot)=\parts
\item \textit{effects}: \textit{ploc}(\parts)=\kt, \textit{rhold}(\robot)=\nil
\end{itemize}

\item \textit{aboveP} (\parts, {\constant{pos1}}, {\constant{pos2}})
\begin{itemize}
\item ;;The robot r stations over the part \parts at the position pos2. The previous position of the robot is pos1.
\item \textit{precond}: \textit{rpos}(\robot)={\constant{pos1}}
\item \textit{effects}: \textit{rpos}(\robot)={\constant{pos2}}
\end{itemize}

\item \textit{aboveKit} (\kt, {\constant{pos1}}, {\constant{pos2}})
\begin{itemize}
\item ;;The robot \robot stations over the kit tray \kt at the position pos2. The previous position of the robot is pos1.
\item \textit{precond}: \textit{rpos}(\robot)={\constant{pos1}}
\item \textit{effects}: \textit{rpos}(\robot)={\constant{pos2}}
\end{itemize}

\item \textit{aboveBekt} (\boxekt, {\constant{pos1}}, {\constant{pos2}})
\begin{itemize}
\item ;;The robot \robot stations over the box of empty kit trays at the position pos2. The previous position of the robot is pos1.
\item \textit{precond}: \textit{rpos}(\robot)={\constant{pos1}}
\item \textit{effects}: \textit{rpos}(\robot)={\constant{pos2}}
\end{itemize}

\item \textit{aboveBfkt} (\boxfkt, {\constant{pos1}}, {\constant{pos2}})
\begin{itemize}
\item ;;The robot \robot stations over the box of full kit trays at the position {\constant{pos2}}. The previous position of the robot is {\constant{pos1}}, the final position is {\constant{pos2}}.
\item \textit{precond}: \textit{rpos}(\robot)={\constant{pos1}}
\item \textit{effects}: \textit{rpos}(\robot)={\constant{pos2}}
\end{itemize}

\item \textit{attach} (\robot, \grip, \chstation)
\begin{itemize}
\item ;;The robot \robot attaches the gripper \grip.
\item \textit{precond}: \textit{gloc}(\grip)=chstation,
\item \textit{effects}: \textit{gloc}(\grip)=\robot
\end{itemize}

\item \textit{remove} (\robot, \grip, chstation)
\begin{itemize}
\item ;;The robot \robot removes the gripper \grip.
\item \textit{precond}: \textit{gloc}(\grip)=\robot,
\item \textit{effects}: \textit{gloc}(\grip)=\chstation
\end{itemize}
\end{itemize} 