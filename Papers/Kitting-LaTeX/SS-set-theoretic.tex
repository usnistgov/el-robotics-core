\subsection{Set-Theoretic Representation}
In a set-theoretic representation, each state of the world is a set of propositions, and each action is a syntactic expression specifying which propositions belong to the state in order for the action to be applicable and which propositions the action will add or remove in order to make a new state of the world.

\subsubsection{Planning Domain}
\begin{defn}Let $\mathsf{L}=\lbrace p_1,\dots,p_n\rbrace$ be a finite set of proposition symbols. A set-theoretic
planning domain on $\mathsf{L}$ is a restricted state-transition system $\sum = (\mathsf{S},\mathsf{A},\gamma)$ such
that:
\begin{itemize}
\item $\mathsf{S}\subseteq 2^{\mathsf{L}}$, i.e., each state $s$ is a subset of $\mathsf{L}$. Intuitively, $s$ tells us which propositions currently hold. If $p\in s$, then $p$ holds in the state of the world represented by
$s$, and if $p\notin s$, then $p$ does not hold in the state of the world represented by $s$.
\item Each action $a\in \mathsf{A}$ is a triple of subsets of $\mathsf{L}$, which we will write as
$a = (precond(a), effects^{-}(a), effects ^{+}(a))$. The set $precond(a)$ is called the preconditions
of $a$, and the sets $effects^{+}(a)$ and $effects^{-}(a)$ are called the effects
of $a$. We require these two sets of effects to be disjoint, i.e., $effects^{+} (a)\bigcap
effects^{-}(a)=  \emptyset$. The action $a$ is applicable to a state $s$ if $precond(a)\subseteq s$.
\item $\mathsf{S}$ has the property that if $s\in \mathsf{S}$, then, for every action $a$ that is applicable
to $s$, the set $(s - effects^{-}(a)) \bigcup effects^{+}(a)\in \mathsf{S}$. In other words, whenever an
action is applicable to a state, it produces another state. This is useful to us
because once we know what $\mathsf{A}$ is, we can specify $\mathsf{S}$ by giving just a few of its
states.
\item The state-transition function is $\gamma(s,a)=(s - effects^{-}(a))\bigcup effects^{+}(a)$ if
$a\in \mathsf{A}$ is applicable to $s\in \mathsf{S}$, and $\gamma(s,a)$ is undefined otherwise.
\end{itemize}
\end{defn}

One possible set-theoretic representation for the domain depicted in Figure~\ref{fig:state_transition_example} is described below.
\begin{itemize}
\item $\mathsf{L} = \lbrace inkitbinA,inkitbinB,inkitbinC,inslotA,inslotB,inslotC,\\holdingA,holdingB,holdingC,endeffectorA,endeffectorB\rbrace$, where:
\begin{itemize}
\item $inkitbinA$: The kit part A is still in the kit bin A.
\item $inkitbinB$: The kit part B is still in the kit bin B.
\item $inkitbinC$: The kit part C is still in the kit bin C.
\item $inslotA$: The kit part A is in a slot A in the kit tray.
\item $inslotB$: The kit part B is in a slot B in the kit tray.
\item $inslotC$: The kit part C is in a slot C in the kit tray.
\item $holdingA$: The robotic arm is holding the kit part A.
\item $holdingB$: The robotic arm is holding the kit part B.
\item $holdingC$: The robotic arm is holding the kit part C.
\item $endeffectorA$: The robotic arm is using the end effector A.
\item $endeffectorB$: The robotic arm is using the end effector B.
\end{itemize}
\item $\mathsf{S}={s_0,\dots,s_7}$, where:
\begin{itemize}
\item $s_0=\lbrace endeffectorA,inkitbinA,inkitbinB,inkitbinC\rbrace$;
\item $s_1=\lbrace endeffectorA,holdingA,inkitbinB,inkitbinC\rbrace$;
\item $s_2=\lbrace endeffectorA,inslotA,inkitbinB,inkitbinC \rbrace$;
\item $s_3=\lbrace endeffectorA,holdingB,inslotA,inkitbinC\rbrace$;
\item $s_4=\lbrace endeffectorA,inslotA,inslotB,inkitbinC\rbrace$;
\item $s_5=\lbrace endeffectorB,inslotA,inslotB,inkitbinC\rbrace$;
\item $s_7=\lbrace endeffectorB,inslotA,inslotB,inslotC\rbrace$;
\end{itemize}
\item $\mathsf{A}={takeA,putA,takeB,putB,takeC,putC,change}$, where:
\begin{itemize}
\item $takeA=(\lbrace inkitbinA\rbrace,\lbrace inkitbinA\rbrace,\lbrace holdingA\rbrace)$;
\item $putA=(\lbrace holdingA\rbrace,\lbrace holdingA\rbrace,\lbrace inslotA\rbrace)$;
\item $takeB=(\lbrace inkitbinB\rbrace,\lbrace inkitbinB\rbrace,\lbrace holdingB\rbrace)$;
\item $putB=(\lbrace holdingB\rbrace,\lbrace holdingB\rbrace,\lbrace inslotB\rbrace)$;
\item $takeC=(\lbrace inkitbinC\rbrace,\lbrace inkitbinC\rbrace,\lbrace holdingC\rbrace)$;
\item $putC=(\lbrace holdingC\rbrace,\lbrace holdingC\rbrace,\lbrace inslotC\rbrace)$;
\item $change=(\lbrace endeffectorA\rbrace,\lbrace endeffectorA\rbrace,\lbrace endeffectorB\rbrace)$;
\end{itemize}
\end{itemize}

\subsubsection{Planning Solution}
\begin{defn} Let $\mathcal{P}=(\Sigma,s_0,g)$ be a planning problem. A plan $\pi$ is a solution for $\mathcal{P}$ if
$g\subseteq\gamma(s_0,\pi)$. A solution $\pi$ is redundant if there is a proper subsequence of $\pi$ that
is also a solution for $\mathcal{P}$; $\pi$ is minimal if no other solution plan for $\mathcal{P}$ contains fewer
actions than $\pi$.

\end{defn}