\subsection{State Variable Symbols}\label{state_variable_symbol}
A k-ary state variable~\cite{NAU.2004} is an expression of the form $x(v_1,\dots,v_k)$, where $x$ is a
state-variable symbol and each $v_i$ is either an object symbol (see Section~\ref{constant_symbol}) or an object variable (see Section~\ref{object_variable_symbol}).
A state variable denotes an element of a state-variable function,

\begin{equation*}
    x: D{_1^x}\times \dots\times D{_k^x}\times S \rightarrow D{_{k+1}^x}
\end{equation*}

where each $D{_i^x}\in D$ is the union of one or more classes.

\begin{comment}
Consider the kitting domain described in Section~\ref{kitting_domain}. We can use \textit{gloc}({\constant{g1}}) to designate the current location of gripper g1 in the workstation. According to the state-transition system for the kitting domain, in the state $s_0$ we have \textit{gloc}({\constant{g1}})=chstation; in the state $s_1$  we have \textit{gloc}({\constant{g1}})={\constant{r1}}. From the definition of a k-ary state variable, \textit{gloc} can be expressed as \textit{gloc}: {\itshape {grippers}} $\times S \rightarrow$ {\itshape {changing stations}} $\cup$ {\itshape {robots}}.\\ \\
\end{comment}
The topology of the kitting domain is denoted using the following unground state variables.

\begin{itemize}
\item \textit{gloc}: \textit{grippers} $\times S \rightarrow$ \textit{robots} $\cup$ \textit{changing station} $\cup$ $\lbrace nil\rbrace$: designates the location of a gripper in the workstation. The gripper can be in the changing station or attached to the robot ($\lbrace nil\rbrace$).


%\item \textit{rgrip}: \textit{robots} $\times S \rightarrow$ \textit{grippers} $\cup$ $\lbrace nil\rbrace$: designates the gripper attached to a robot. If %the robot is not equipped with a gripper, the result is $nil$.

\item \textit{topworktable}: \textit{work table} $\times S \rightarrow$ \textit{empty kit trays} $\cup$ \textit{partially full kit trays} $\cup$ \textit{finished kit trays} $\cup$ $\lbrace nil\rbrace$: designates the object placed on the work table. It could be an empty kit tray, a partially full kit tray, a finished kit tray, or nothing.

\item \textit{ektloc}: {\itshape {empty kit trays}} $\times S \rightarrow$ {\itshape {boxes of empty kit trays}} $\cup$ {\itshape {work table}} $\cup$ $\lbrace nil\rbrace$: designates the different possible locations of an empty kit tray in the workstation. An empty kit tray can be in the box of empty kit trays, on the work table, or being held by the robot ($\lbrace nil\rbrace$) when the robot moves the empty kit tray from the box of empty kit trays to the work table.

\item \textit{fktloc}: \textit{finished kit trays} $\times S \rightarrow$ \textit{boxes of finished kit trays} $\cup$ \textit{work table} $\cup$ $\lbrace nil\rbrace$: designates the different possible locations of a finished kit tray in the workstation. A finished kit tray can be in the box of finished kit trays, on the work table, or being held by the robot ($\lbrace nil\rbrace$) when the robot moves the finished kit tray from the work table to the box of finished kit trays.


\item \textit{ploc}: \textit{parts} $\times S \rightarrow$ \textit{partially full kit trays} $\cup$ \textit{part supplies} $\cup$ \textit{finished kit trays} $\cup$ $\lbrace nil\rbrace$: designates the different possible locations of a part in the workstation. The part can be in the part supply, in the partially full kit tray, in the finished kit tray, or being held by the robot ($\lbrace nil\rbrace$) when the robot moves the part from the part supply to the empty or partially full kit tray,

\item \textit{rhold}: \textit{robots} $\times S \rightarrow$ \textit{finished kit trays} $\cup$ \textit{empty kit trays} $\cup$ \textit{parts} $\cup$ $\lbrace nil\rbrace$: designates the object being held by the robot. It can be a finished kit tray, an empty kit tray, a part, or nothing. We assume that the robot is already equipped with the appropriate gripper.

\item \textit{boxempty}: \textit{boxes of finished kit trays} $\times$ \textit{boxes of empty kit trays} $\times$ \textit{part supplies} $\times$ $S \rightarrow$ $\lbrace 0\rbrace$ $\cup$ $\lbrace 1\rbrace$: designates the status of boxes of empty kit trays, boxes of finished kit trays, and part supplies. 1 if empty, 0 if not empty.

\item \textit{boxfull}: \textit{boxes of finished kit trays} $\times$ \textit{boxes of empty kit trays} $\times$ \textit{part supplies} $\times$ $S \rightarrow$ $\lbrace 0\rbrace$ $\cup$ $\lbrace 1\rbrace$: designates the status of boxes of empty kit trays, boxes of finished kit trays, and part supplies. 1 if full, 0 if not full.

\item \textit{topbox}: \textit{box of empty kit trays} $\times$  $S \rightarrow$ \textit{empty kit trays}: designates the empty kit tray at the top of the box of empty kit trays.

\item \textit{grippertype}: \textit{grippers} $\times$ $S \rightarrow$ \textit{empty kit trays} $\cup$ \textit{finished kit trays} $\cup$ \textit{parts}: designates the type of object the gripper can hold. One type of gripper can hold empty and finished kit trays. Another type of gripper can only hold parts. 

\begin{comment}
\item \textit{rabove}: \textit{robots} $\times$ $S \rightarrow$ \textit{grippers} $\cup$ \textit{boxes of empty kit trays} $\cup$ \textit{work table} $\cup$ \textit{changing station} $\cup$ \textit{parts} $\cup$ \textit{empty kit trays} $\cup$ \textit{partially full kit trays} $\cup$ \textit{finished kit trays} $\cup$ \textit{boxes of full kit trays}: designates the object the robot positions above. The situations in which these objects are used are described below:
\begin{itemize}
\item \textit{grippers}: The robot moves above a gripper before picking it up.
\item \textit{boxes of empty kit trays}: The robot moves above a box of empty kit trays before picking up an empty kit tray.
\item \textit{work table}: The robot moves above the work table before putting down an empty kit tray.
\item \textit{changing station}: The robot moves above the changing station before removing a gripper attached to the robot.
\item \textit{parts}: The robot moves above a part before picking it up.
\item \textit{empty kit trays}: The robot moves above an empty kit tray before putting down a part.
\item \textit{partially full kit trays}: The robot moves above a partially full kit tray before putting down a part.
\item \textit{finished kit trays}: The robot moves above a finished kit tray before picking it up.
\item \textit{boxes of full kit trays}: The robot moves above a box of full kit trays before putting down a finished kit tray.
\end{itemize}
\end{comment}
\end{itemize} 