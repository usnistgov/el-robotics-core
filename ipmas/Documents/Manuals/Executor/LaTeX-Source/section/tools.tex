\section{Components for Autotools}\label{section:components}

This section describes all the components required by autotools to build and distribute a project.

\subsection{\prog{aclocal}}
The \prog{aclocal} program creates the file \file{aclocal.m4} by combining stock
installed macros, user defined macros and the contents of
\file{acinclude.m4} to define all of the macros required by \file{configure.ac}
in a single file. \prog{aclocal} was created as a fix for some missing
functionality in \prog{Autoconf}, and as such we consider it a wart. In due
course \prog{aclocal} itself will disappear, and \prog{Autoconf} will perform the
same function unaided.
\begin{figure}[h!]
\centering
\scalebox{.7}{
\begin{tikzpicture}[node distance=1.5cm, every edge/.style={link}]

\node[dummy] (input) {\textit{User input files}};
\node[dummy] (optional) [right=3cm of input]{\textit{Optional input}};
\node[dummy] (process) [right=3cm of optional]{\textit{Process}};
\node[dummy] (output) [right=3cm of process]{\textit{Output files}};

\node[dummy] (acinclude) [below=.5cm of optional] {\file{acinclude.m4}};
\node[dummy] (configure) [below=2cm of input] {\file{configure.ac}};
\node[entity] (aclocal) [below=2cm of process] {\prog{aclocal}};
\node[dummy] (macro) [below=3cm of optional] {User macro files};
\node[dummy] (aclocalm4) [right=3cm of aclocal] {\file{aclocal.m4}};

\draw[myarrow] (configure.east) -- (aclocal);
\draw[myarrow] (macro.east)-|(aclocal.south);
\draw[myarrow] (acinclude.east)-|(aclocal.north);
\draw[myarrow] (aclocal.east) -- (aclocalm4.west);
\end{tikzpicture}
}
\caption{\prog{aclocal} process.}
\label{fig:aclocal}
\end{figure}

\subsection{\prog{autoheader}}
\prog{autoheader} runs m4 over \file{configure.ac}, but with key macros defined differently than when \prog{Autoconf} is executed, such that suitable cpp and cc definitions are output to \file{config.h.in}.

\begin{figure}[h!]
\centering
\scalebox{.7}{
\begin{tikzpicture}[node distance=1.5cm, every edge/.style={link}]

\node[dummy] (input) {\textit{User input files}};
\node[dummy] (optional) [right=3cm of input]{\textit{Optional input}};
\node[dummy] (process) [right=3cm of optional]{\textit{Process}};
\node[dummy] (output) [right=3cm of process]{\textit{Output files}};

\node[dummy] (aclocalm4) [below=.5cm of optional] {\file{acinclude.m4}};
\node[dummy] (acconfig) [below=0.1cm of aclocalm4] {\file{(acconfig.h)}};
\node[dummy] (configure) [below=3cm of input] {\file{configure.ac}};
\node[entity] (autoheader) [below=3cm of process] {\prog{autoheader}};
\node[dummy] (congifhin) [right=3cm of autoheader] {\file{config.h.in}};

\draw[myarrow] (configure.east) -- (autoheader.west);
\draw[myarrow] (aclocalm4.east)-|(autoheader);
\draw[myarrow] (acconfig.east)-|(autoheader);
\draw[myarrow] (autoheader.east) -- (congifhin.west);
\end{tikzpicture}
}
\caption{\prog{autoheader} process.}
\label{fig:aclocal}
\end{figure}

\subsection{\prog{automake} and \prog{libtoolize}}
\prog{automake} will call \prog{libtoolize} to generate some extra files if the macro \texttt{AC\_PROG\_LIBTOOL} is used in \file{configure.ac}. If it is not present then \prog{automake} will install \file{config.guess} and \file{config.sub} by itself.

\prog{libtoolize} can also be run manually if desired; \prog{automake} will only run \prog{libtoolize} automatically if \file{ltmain.sh} and \file{ltconfig} are missing.

\begin{figure}[h!]
\centering
\scalebox{.7}{
\begin{tikzpicture}[node distance=1.5cm, every edge/.style={link}]

\node[dummy] (input) {\textit{User input files}};
\node[dummy] (optional) [right=3cm of input]{\textit{Optional input}};
\node[dummy] (process) [right=3cm of optional]{\textit{Process}};
\node[dummy] (output) [right=3cm of process]{\textit{Output files}};

%input
\node[dummy] (configure) [below=5cm of input] {\file{configure.ac}};
\node[dummy] (makefile) [below=6cm of input] {\file{Makefile.am}};
%process
\node[entity] (libtoolize) [below=11cm of process] {\prog{libtoolize}};
\node[entity] (automake) [below=10cm of process] {\prog{automake}};
%output for automake
\node[dummy] (copying) [below=1cm of output] {\file{COPYING}};
\node[dummy] (install) [below=2cm of output] {\file{INSTALL}};
\node[dummy] (installsh) [below=3cm of output] {\file{install-sh}};
\node[dummy] (missing) [below=4cm of output] {\file{missing}};
\node[dummy] (mkinstalldirs) [below=5cm of output] {\file{mkinstalldirs}};
\node[dummy] (makefilein) [below=6cm of output] {\file{Makefile.in}};
\node[dummy] (stamph) [below=7cm of output] {\file{stamp-h.in}};
\node[dummy] (configguess1) [below=8cm of output] {\file{config.guess}};
\node[dummy] (configsub1) [below=9cm of output] {\file{config.sub}};

%output for libtoolize
\node[dummy] (configguess2) [below=11cm of output] {\file{config.guess}};
\node[dummy] (configsub2) [below=12cm of output] {\file{config.sub}};
\node[dummy] (ltmain) [below=13cm of output] {\file{ltmain.sh}};
\node[dummy] (ltconfig) [below=14cm of output] {\file{ltconfig}};

%Drawing arrows
\draw[myarrow] (configure.east)-- ++ (4,0) -- ++ (0,0)  |-(automake.west);
\draw[myarrow] (makefile.east)-- ++ (3,0) -- ++ (0,0)  |-(automake.west);
\draw[myarrow] (automake.east)-- ++ (1,0) -- ++ (0,0)  |-(copying.west);
\draw[myarrow] (automake.east)-- ++ (1,0) -- ++ (0,0)  |-(install.west);
\draw[myarrow] (automake.east)-- ++ (1,0) -- ++ (0,0)  |-(installsh.west);
\draw[myarrow] (automake.east)-- ++ (1,0) -- ++ (0,0)  |-(missing.west);
\draw[myarrow] (automake.east)-- ++ (1,0) -- ++ (0,0)  |-(mkinstalldirs.west);
\draw[myarrow] (automake.east)-- ++ (1,0) -- ++ (0,0)  |-(makefilein.west);
\draw[myarrow] (automake.east)-- ++ (1,0) -- ++ (0,0)  |-(stamph.west);
\draw[myarrow] (automake.east)-- ++ (1,0) -- ++ (0,0)  |-(configguess1.west);
\draw[myarrow] (automake.east)-- ++ (1,0) -- ++ (0,0)  |-(configsub1.west);

\draw[myarrow] (libtoolize.east)-- ++ (1,0) -- ++ (0,0)  |-(configguess2.west);
\draw[myarrow] (libtoolize.east)-- ++ (1,0) -- ++ (0,0)  |-(configsub2.west);
\draw[myarrow] (libtoolize.east)-- ++ (1,0) -- ++ (0,0)  |-(ltmain.west);
\draw[myarrow] (libtoolize.east)-- ++ (1,0) -- ++ (0,0)  |-(ltconfig.west);
\end{tikzpicture}
}
\caption{\prog{automake} and \prog{libtoolize} processes.}
\label{fig:aclocal}
\end{figure}

The versions of \file{config.guess} and \file{config.sub} installed differ
between releases of \prog{Automake} and \prog{Libtool}, and might be different
depending on whether \prog{libtoolize} is used to install them or not. Before
releasing your own package you should get the latest versions of these
files from \url{ftp://ftp.gnu.org/gnu/config}, in case there have been
changes since releases of the GNU \prog{Autotools}.

\subsection{\prog{autoconf}}
\prog{autoconf} expands the m4 macros in \file{configure.ac}, perhaps using macro definitions from \file{aclocal.m4}, to generate the configure script.

\begin{figure}[h!]
\centering
\scalebox{.7}{
\begin{tikzpicture}[node distance=1.5cm, every edge/.style={link}]

\node[dummy] (input) {\textit{User input files}};
\node[dummy] (optional) [right=3cm of input]{\textit{Optional input}};
\node[dummy] (process) [right=3cm of optional]{\textit{Process}};
\node[dummy] (output) [right=3cm of process]{\textit{Output files}};

\node[dummy] (aclocal) [below=.5cm of optional] {\file{aclocal.m4}};
\node[dummy] (configureac) [below=2cm of input] {\file{configure.ac}};
\node[entity] (autoconf) [below=2cm of process] {\prog{autoconf}};
\node[dummy] (configure) [right=3cm of autoconf] {\file{configure}};

\draw[myarrow] (configureac.east) -- (autoconf);
\draw[myarrow] (aclocal.east)-|(autoconf.north);
\draw[myarrow] (autoconf.east) -- (configure.west);
\end{tikzpicture}
}
\caption{\prog{autoconf} process.}
\label{fig:aclocal}
\end{figure}

\subsection{\prog{configure}}
The purpose of the preceding processes was to create the input files
necessary for configure to run correctly. You would ship your project
with the generated script and the files in columns, other input and
processes (except \file{config.cache}), but configure is designed to be run
by the person installing your package. Naturally, you will run it too
while you develop your project, but the files it produces are specific
to your development machine, and are not shipped with your package --
the person installing it later will run configure and generate output
files specific to their own machine.

Running the configure script on the build host executes the various
tests originally specified by the \file{configure.ac} file, and then
creates another script, \file{config.status}. This new script generates the
\file{config.h} header file from \file{config.h.in}, and \file{Makefile} from the
named \file{Makefile.in}. Once \file{config.status} has been created, it can be
executed by itself to regenerate files without rerunning all the
tests. Additionally, if \texttt{AC\_PROG\_LIBTOOL} was used, then \prog{ltconfig} is
used to generate a \file{libtool} script.

\begin{figure}[h!]
\centering
\scalebox{.7}{
\begin{tikzpicture}[node distance=1.5cm, every edge/.style={link}]

\node[dummy] (input) {\textit{User input files}};
\node[dummy] (optional) [right=3cm of input]{\textit{Other input}};
\node[dummy] (process) [right=3cm of optional]{\textit{Process}};
\node[dummy] (output) [right=3cm of process]{\textit{Output files}};

%other input
\node[dummy] (configsite) [below=1cm of optional] {\file{config.site}};
\node[dummy] (configcache1) [below=2cm of optional] {\file{config.cache}};
\node[dummy] (confighin) [below=4.2cm of optional] {\file{config.h.in}};
\node[dummy] (Makefilein) [below=5.2cm of optional] {\file{Makefile.in}};


%process
\node[entity] (configure) [below=1.5cm of process] {\prog{configure}};
\node[entity] (configstatus1) [below=2cm of configure] {\file{config.status}};
\node[entity] (ltconfig) [below=2cm of configstatus1] {\prog{ltconfig}};

\node[dummy] (ltmainsh) [left=3cm of ltconfig] {\file{ltmain.sh}};
%output
\node[dummy] (configcache2) [right=3cm of configure] {\file{config.cache}};
\node[dummy] (configstatus2) [below=1cm of configcache2] {\file{config.status}};
\node[dummy] (configh) [below=4.6cm of output] {\file{config.h}};
\node[dummy] (Makefile) [below=5.5cm of output] {\file{Makefile}};
\node[dummy] (stamph) [below=6.4cm of output] {\file{stamp-h}};
\node[dummy] (libtool) [right=3cm of ltconfig] {\prog{libtool}};

\draw[myarrow] (configsite.east)-- ++ (0.5,0) |-(configure.west);
\draw[myarrow] (configcache1.east)-- ++ (0.39,0) |-(configure.west);
\draw[myarrow] (confighin.east)-- ++ (0.5,0) |-(configstatus1.west);
\draw[myarrow] (Makefilein.east)-- ++ (0.5,0) |-(configstatus1.west);
\draw[myarrow] (ltmainsh.east)-- (ltconfig);

\draw[myarrow] (configure.east)-- (configcache2);
\draw[myarrow] (configstatus1.east)-- ++ (1,0) -- ++ (0,0)  |-(configstatus2.west);
\draw[myarrow] (configstatus1.east)-- ++ (1,0) -- ++ (0,0)  |-(configh.west);
\draw[myarrow] (configstatus1.east)-- ++ (1,0) -- ++ (0,0)  |-(Makefile.west);
\draw[myarrow] (configstatus1.east)-- ++ (1,0) -- ++ (0,0)  |-(stamph.west);
\draw[myarrow] (ltconfig.east)-- (libtool);
\end{tikzpicture}
}
\caption{\prog{configure} process.}
\label{fig:aclocal}
\end{figure}
