
Material feeding systems are an integral part of today's assembly line operations.
These systems assure that parts are available where and when
they are needed during the assembly operations by providing either a continuous
supply of parts at the station, or a set of parts (known
as a kit) that contains the required parts for one or more assembly operations.
In continuous supply, a quantity of each part that
may be necessary for the assembly operation is stored at the assembly station.
If multiple versions of a product are being assembled (mixed-model assembly),
a larger variety of parts than are used for an individual assembly may need
to be stored. With this material feeding scheme, parts
storage and delivery systems must be duplicated at each assembly station.

One alternative to continuous supply is known as kitting. In kitting,
parts are delivered to the assembly station in kits that contain
the exact parts necessary for the completion of one assembly object.
According to Bozer and McGinnis \cite{Bozer1992} ``A kit is a specific
collection of components and/or subassemblies that together
(i.e., in the same container) support one or more assembly
operations for a given product or shop order''. In the case of mixed-model
assembly, the contents of a kit may vary from product to product.
The use of kitting allows a single delivery system to feed
multiple assembly stations. The individual operations of the station that
builds the kits may be viewed as a specialization of the general
bin-picking problem \cite{Schyja2012}.

In industrial assembly of manufactured products, kitting is
often performed prior to final assembly. Manufacturers utilize kitting
due to its ability to provide cost savings \cite{Carlsson_2008}
including saving manufacturing or assembly space \cite{Medbo2003},
reducing assembly workers walking and searching times \cite{Schwind1992},
and increasing line flexibility \cite{Bozer1992} and balance \cite{Jiao2000}.

Several different techniques are used to create kits. A kitting
operation where a kit box is stationary until filled at a single
kitting workstation is referred to as {\it batch kitting}.
In {\it zone kitting}, the kit moves while being filled and will pass through one or
more zones before it is completed. This article focuses on batch kitting processes.

In batch kitting, the kit's component parts may be staged in
containers positioned in the workstation or may arrive on a conveyor.
Component parts may be fixtured, for example placed in compartments
on trays, or may be in random orientations, for example
placed in a large bin. In addition to the kit's component parts,
the workstation usually contains a storage area for empty kit boxes as
well as completed kits.

For our sample implementation, we assume that
a robot performs a series of pick-and-place operations
in order to construct the kit. These operations include:
\begin{enumerate}
\item Pick up an empty kit and place it on the work table.
\item Pick up multiple component parts and place them in a kit.
\item Pick up the completed kit and place it in the full kit storage area.
\end{enumerate}
Each of these may be a compound action that includes
other actions such as end-of-arm tool changes, path planning,
and obstacle avoidance.

It should be noted that multiple kits may be built simultaneously.
Finished kits are moved to the assembly floor where components
are picked from the kit for use in the assembly procedure.
The kits are normally designed to facilitate component picking in the correct
sequence for assembly. Component orientation may be constrained
by the kit design in order to ease the pick-to-assembly process.
Empty kits are returned to the kit building area for reuse.

The knowledge methodology and model presented in this article is
designed to allow for more agility and flexibility in the
kit preparation process. This includes the ability to easily
adapt to variations in kit contents, kit
layout, and component supply as well as the ability to work in
environments where components are not fixtured to precise locations.
