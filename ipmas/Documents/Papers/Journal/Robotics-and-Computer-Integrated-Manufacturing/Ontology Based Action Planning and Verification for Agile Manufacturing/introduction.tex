\section{Introduction}
Many of today's robotic arms are capable of obtaining
sub-millimeter accuracy and repeatability. Robots such as the
Fanuc LR Mate 200\textit{i}D claim $\pm$ 0.02 mm repeatability
\cite{Fanuc200iD} which has been verified in various publicly
viewable experiments \cite{RobotAccuracy} \cite{RobotAccuracy2}.
However, these
same systems lack
the sensors and processing necessary to provide a representation of
the work cell in which they reside or of the parts that they are working
with. In fact, according to the International Federation of Robotics (IFR),
over 95\% of all robots in use do not have a sensor in the outer feedback loop.
They rely on fixtures to allow them to be robust in the presence of 
uncertainty \cite{WorldRobot}. This lack of sensing in the outer feedback loop leads to
systems that are taught or programmed to provide specific patterns of motion in
structured work cells over long production runs. These systems are unable
to detect that environmental changes have occurred, and are therefore unable
to modify their behavior to provide continued correct operation. 

Just-in-time manufacturing and small
batch processing requires changes in the manufacturing process on a batch-by-batch
or item-by-item basis. This leads to a reduction in the number of cycles
that a particular pattern of motion is useful and increases the 
percentage of time necessary for robot teaching or programming over actual cell operation.
This teaching/programming time requires that the 
cell be taken off-line which greatly impacts productivity.
For small batch processors or other customers who must frequently change their line configuration, 
this frequent downtime and lack of adaptability may be unacceptable.

Research aimed at increasing a robot's knowledge and intelligence has been performed 
to address some of these issues. 
It is anticipated that proper application of this intelligence
will lead to more agile and flexible robotic systems. Both Huckaby et al.
\cite{Huckaby2012} and Pfrommer et al. \cite{Pfrommer2013} have examined the enumeration
of basic robotic skills that may be dynamically combined to achieve production goals.
The EU-funded RObot control for Skilled ExecuTion of Tasks in natural interaction with
humans (ROSETTA) \cite{Patel2012} and Skill-Based Inspection and Assembly for 
Reconfigurable Automation Systems (SIARAS) \cite{Stenmark2013} have proposed 
distributed knowledge
stores that contain representations of robotic knowledge and skills. The focus of these
programs is to simplify interaction between the user and the robotized automation system.

The IEEE Robotics and Automation Society's Ontologies for Robotics and Automation Working Group \cite{schlenoff2012} has also taken the 
first steps in creating a knowledge repository that will
allow greater intelligence to be resident on robotic platforms.
The 
Industrial Subgroup of this working group has applied 
this infrastructure to 
create a sample kit building
system.  Kit building may be viewed as a simple, but relevant
manufacturing process. 

Balakirsky et al. \cite{Balakirsky2013} describe a kitting 
system based on the IEEE knowledge framework that
allows greater flexibility and agility by utilizing a
Planning Domain Definition Language (PDDL) \cite{PDDL} planning
system to dynamically alter the system's operation in order
to adapt to variations in its anticipated work flow. The system
does not require {\it a priori} information on part locations (i.e
fixturing is not required) and is able to build new kit varieties
without altering the robot's programming. While this body of
work makes great strides in removing the need to teach/program
the robot between production runs, all of the PDDL predicates and
actions must still be programmed/taught. 

This means that any
modification to the production process that requires a new
PDDL predicate or action will still require that the production 
line be brought down for programming/teaching. The next logical step
in adding agility to robotic production is to remove this 
programming/teaching step when new actions and predicates
are required by the system. 
This body of work examines utilizing a basis set of robotic
primitive actions to enumerate new robotic skills and the
enhancement of the IEEE ontology to store those skills in a reusable
knowledge store. This removes the need to program
the robot for new predicates and actions. The sample domain of kit building is utilized
to demonstrate this work.

The organization of the remainder of this paper is as follows.
Section \ref{sect:PDDL} provides an overview of the PDDL language and
a discussion of how PDDL is integrated into the ontology. 
Section \ref{sect:operation} discusses the detailed operation of cell
and presents the system's architecture, and Section 
\ref{sect:knowledge} discusses the knowledge representation and
the ontology. Finally, Section \ref{sect:future} presents
conclusions and future work.
%
