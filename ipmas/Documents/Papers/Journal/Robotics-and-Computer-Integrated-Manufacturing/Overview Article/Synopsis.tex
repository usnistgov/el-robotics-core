\section{Paper Synopsis}
The special issue opens with two papers that examine various aspects of the IEEE Robotics 
and Automation Society's Robotics and Automation Working Group's ontology. The first paper
presents the Core Ontology for Robotics and Automation(CORA), while the second paper presents
details on the Position, Orientation, and Space (POS) ontology that is designed for sharing spatial concepts in the Robotics 
and Automation domain.

These papers are followed by a set of articles that demonstrate human-robot cell collaboration
through the use of ontologies and software frameworks. The  intent of these papers is to 
provide mechanisms that allow users to better understand the process through their industrial
cell and to ease robot programming.

Three additional papers are presented that relate to robot programming and agility. These
papers present systems that attempt to enhance robot agility by allowing for dynamic
programming of the robot through the knowledge representation. Various protocols are
discussed along with metrics and test methods.  All of the papers in this section present
the sample domain of kit building to highlight their achievements.

From here, the special issue turns to use cases that utilize the knowledge representations for 
applications ranging from orthopedic surgery to intention recognition. Specific examples
of knowledge necessary for performing hip surgery, designing rehabilitation robots, and
using ontologies to determine the intentions of humans working together with robots
are examined.
