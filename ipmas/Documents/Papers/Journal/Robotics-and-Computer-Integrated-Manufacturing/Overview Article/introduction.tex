\section{Introduction}
Today's state-of-the-art robots are capable of sub-millimeter movement accuracy. 
However, they are often programmed by an operator using crude positional controls 
from a teach pendent. Reprogramming these robots when their task is altered requires 
that the robot cell be often taken offline for a human-led teaching period. In an 
industrial setting, for small batch processors or other customers who must 
frequently change their line configuration, this frequent down time may be unacceptable. 

The robotic systems of tomorrow need to be capable, flexible, and agile. These 
systems need to perform their duties at least as well as human counterparts, be 
quickly re-tasked to other operations, and cope with a wide variety of unexpected 
environmental and operational changes. To be successful, these systems need to combine 
domain expertise, knowledge of their own skills and limitations, and both semantic and 
geometric information.

This special issue builds off of a successful special session at the 2013 Robot Intelligence Technology and Applications (RITA) Conference focusing on "Knowledge Representation for Robotics and Automation." The special issue is comprised of 11 papers, including the best papers from this special session as well as contributions from others working in this field. The papers examine various aspects of knowledge representations and planning for industrial systems as well as use cases that comprise  both assembly and medical domains.
