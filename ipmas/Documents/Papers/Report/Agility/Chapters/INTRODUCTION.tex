%%--------------------------------%
\Chapter{Introduction}
\label{chapter:INTRODUCTION}
%%--------------------------------%

In recent years, the evolutionary progress and paradigm shifts of manufacturing systems have been moving towards agile manufacturing systems. The agile manufacturing paradigm was formulated in response to the constantly changing environment and as a basis for returning to global competitiveness. Agile manufacturing can be defined as the capability of surviving and prospering in a competitive environment of continuous and unpredictable change by reacting quickly and effectively to changing markets, driven by customer-designed products and services~\cite{Cunha.2007,CHO.CIE.1996,GUNASEKARAN.1999}.

Having the ability to adapt quickly, an agile manufacturing system (AMS) can handle the variations in realistic production
environments as well as the changing market needs. Agile manufacturing requires high responsiveness at all levels of a company, but is especially challenging on the shop floor level. One of the most important requirements for the implementation of an AMS is to enable its control system to respond and adapt to changes in production variables. These variables may be caused by expected events such as the introduction of new products, changes in the product design or high product customisation, as well as by unanticipated events such as machine failures during manufacturing, equipment drop-outs, rush orders, cancellation of orders or changes in the priority of orders etc. The control of an AMS is expected to be flexible, open, scalable and re-configurable so as to tackle the more complex and uncertain information flows. The current challenge is to develop collaborative and reconfigurable manufacturing control systems that support efficiently small batches, product diversity, high quality and low costs, by introducing innovative characteristics of adaptation, agility and modularization. Information and communication technologies, and artificial intelligence techniques, have been used for more than two decades addressing this challenge.

Today, the market tends increasingly towards mass customization, with customers clients individually selecting from various product options. Customers require high responsiveness and the ability to cope with a multitude of conditions. They increasingly need agile robotic assembly systems, able to
cope with dynamic production conditions dictated by frequent changes, low volumes and many variants. A robotic assembly system is an industrial installation that receives parts and joins them in a coherent way to form a final product. It consists of a set of equipment items (manufacturing resources or modules) such as conveyors, pallets, simple robotic axes for translation and rotation as well as more sophisticated industrial robots, grippers, sensors of various types, etc.

The current state-of-the-art industrial robots are capable of sub-millimeter movement accuracy~\cite{Briot.2009}. However, they are often programmed by an operator using crude positional controls from a teach pendent. Reprogramming these robots when their task is altered requires that the robot cell be taken off-line for a human-led teaching period. For small batch processors or other
customers who must frequently change their line configuration, this frequent down time may be unacceptable. The robotic systems of tomorrow need to perform their duties at least as well as human counterparts, be quickly re-tasked to other operations, and cope with a wide variety of unexpected environmental and operational changes. Improvements in these areas are believed to: generate industrial systems that are quickly changeable, follow a Plug\&Play approach, avoid time- and work-intensive re-programming,
and maintain productivity also under perturbations. Much attention has been given to the requirement for assembly systems to be easily reconfigurable or to seamlessly integrate
new components.

In this paper, we present a literature review of agility methods, use cases, and metrics that are attempting to enhance the agility of robots in manufacturing environments.  This is in support of the Agility Performance of Robotic System project, part of the Robotics for Smart Manufacturing Program at the National Institute of Standards and Technology (NIST). The goal of this project is to deliver robot agility performance metrics, information models, test methods and protocols, validated using a combined virtual and real testing environment, that will enable manufacturers to easily and rapidly reconfigure and re-task robot systems in assembly operations.

The paper is organized as follows:
\begin{itemize}
\item In Chapter~\ref{chapter:PRICE}, we describe some approaches to enhance agility in manufacturing robotic systems. These are categorized into hardware abstraction, process adaptability, and human cooperation.
\item In Chapter~\ref{chapter:DOWNS}, we describe some approaches in the literature that have attempted to create use cases to test robot agility in manufacturing applications. Few use cases were found, but that ones that were are presented in this section.
\item In Chapter~\ref{chapter:KOOTBALLY}, we describe a set of metrics that can be applied to the measurement of flexibility in agile manufacturing systems. These can be broken down into machine flexibility, product flexibility, material handling system flexibility, operation flexibility, routing flexibility, and volume flexibility.
\item In Chapter~\ref{chapter:CONCLUSION}, we conclude the paper and describe efforts that are planned in the future to collect use cases and metrics from industry directly.
\end{itemize}
