\documentstyle[astron]{article}

\title{Citation Examples}
\author{S. J. Hogeveen}

\begin{document}
\maketitle

\section*{Introduction}

This file gives examples of citations as they can be made with the
`{\tt astron}' bibliography style.
To see how citations are made, you should look at the source text of
this example: {\tt example.tex}.
The bibliography database that goes with this example is: {\tt example.bib}.
To get everything right, you must: 1.~run LaTeX on {\tt example.tex},
2.~run BibTeX on {\tt example}, 3.~run LaTeX on {\tt example},
4.~run LaTeX on {\tt example}.

\section*{Citations}

You can make full and short citations.
A full citation is used if the author's name is not part of the running text,
like in: `Due to general relativistic effects \cite{einstein}, light of a
star will \ldots'.
A short (year only) citation is used if the name of the author {\sl is\/} part
of the running text, like in: `Einstein \cite*{einstein} has shown that
light of a star that passes the sun will \ldots'.

You can also (short) cite more references in one \verb|\cite| or \verb|\cite*|
call, {\sl and\/} add an optional note, like in: `As many authors have shown
\cite[but not necessarily in this order]{einstein,burkhardt,chandrasekhar,%
pannekoek,pringle:ibs}.
% Note that no white spaces are left between the citation labels and the commas.
% If you do leave blanks, BibTeX will complain.
Here is what happens if you cite two (or more) publications of the same author
which appeared in the same year: \cite{kuip:ds1,kuip:ds2}.
The \verb|\cite*| form of the same citations yields: Kuiper
\cite*{kuip:ds1,kuip:ds2}.

Finally, we invoke the entire database to be entered in the list of references
with a \verb|\nocite{*}| call \nocite{*}.
Note that the \verb|\nocite| command is not replaced by {\it key\/}
information.

\bibliography{mnemonic,example}
\bibliographystyle{astron}

\end{document}

