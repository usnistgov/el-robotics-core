\subsection*{1) Set up R\-O\-S with Descartes and R\-O\-S Industrial }

Copy repositories from \href{https://github.com/ros-industrial:}{\tt https\-://github.\-com/ros-\/industrial\-:} \begin{DoxyVerb}fanuc
fanuc experimental
motoman
ros industrial core
\end{DoxyVerb}


Repositories from R\-O\-S-\/\-I Consortium \href{https://github.com/ros-industrial-consortium}{\tt https\-://github.\-com/ros-\/industrial-\/consortium} github site\-: \begin{DoxyVerb}descartes
descartes_tutorials
\end{DoxyVerb}


\subsection*{2) Installing Xerces c with Ubuntu }

\href{https://www.daniweb.com/hardware-and-software/linux-and-unix/threads/409769/ubuntu-11-10-xerces-c}{\tt https\-://www.\-daniweb.\-com/hardware-\/and-\/software/linux-\/and-\/unix/threads/409769/ubuntu-\/11-\/10-\/xerces-\/c} As far as I'm aware libxerces is the same as pretty much any other library in Debian based systems. It should be available in the repositories (the exact version will depend on which version of Ubuntu you're running).

You can use apt-\/get to install the packages for the library and the dev files. Then to use them in your C/\-C++ programs you simply \#include the appropriate headers and link with the library when compiling/linking. \begin{DoxyVerb}sudo apt-get update
apt-cache search libxerces
sudo apt-get install libxerces-c3.1 libxerces-c-dev
\end{DoxyVerb}


Need include file path C\-Make\-Lists.\-txt\-: \begin{DoxyVerb}include_directories(/usr/include/xercesc)
\end{DoxyVerb}


Link library in C\-Make\-Lists.\-txt\-: \begin{DoxyVerb}link_directories(/usr/lib/x86_64-linux-gnu/)
\end{DoxyVerb}


Need to link against libxerces.\-a in C\-Make\-Lists.\-txt\-: \begin{DoxyVerb}target_link_libraries(nist_fanuc 
libxerces-c.a  
${catkin_LIBRARIES}
${Boost_LIBRARIES}
)
\end{DoxyVerb}


\subsection*{3) Installing Code\-Synthesis X\-S\-D }

\href{http://www.codesynthesis.com/products/xsd/download.xhtml}{\tt http\-://www.\-codesynthesis.\-com/products/xsd/download.\-xhtml}
\begin{DoxyEnumerate}
\item Chose the linux deb install file that matches your computer (below 64 bit amd).
\item Download xsd\-\_\-4.\-0.\-0-\/1\-\_\-amd64.\-deb and it will say open with Ubuntu Software Center
\item Click to install, authenticate and add /usr/include/xsd/cxx/xml as include path.
\end{DoxyEnumerate}

Need include file path in C\-Make\-Lists.\-txt\-: \begin{DoxyVerb}include_directories(/usr/include/xsd/cxx/xml)
\end{DoxyVerb}


If you cannot run Ubuntu software centerto install Code\-Synthesis, you can download the source and install it. You need to go to the web page\-: \href{http://www.codesynthesis.com/products/xsd/download.xhtml}{\tt http\-://www.\-codesynthesis.\-com/products/xsd/download.\-xhtml} and select\-: \begin{DoxyVerb}xsd-4.0.0-x86_64-linux-gnu.tar.bz2
\end{DoxyVerb}


It will be saved into /usr/local/downloads, but you can save it anywhere. Then cd to where you saved it, and do this\-: \begin{DoxyVerb}tar --bzip2 -xvf xsd-4.0.0-x86_64-linux-gnu.tar.bz2 (dash-dash bzip2, dash-xvf)
\end{DoxyVerb}


It will create a directory xsd-\/4.\-0.\-0-\/x86\-\_\-64-\/linux-\/gnu.

Make a symbolic link\-: \begin{DoxyVerb}ln -s <path/to/xsd-4.0.0-x86_64-linux-gnu/libxsd/xsd /usr/local/include/xsd
\end{DoxyVerb}


e.\-g., ln -\/s /usr/local/xsd-\/4.0.\-0-\/x86\-\_\-64-\/linux-\/gnu/libxsd/xsd /usr/local/include/xsd

\subsection*{4) Optionally -\/ download netbeans }

\href{https://netbeans.org/downloads/}{\tt https\-://netbeans.\-org/downloads/}

cd to the directory where you downloaded netbeans\-: \begin{DoxyVerb}./netbeans-8.1-cpp-linux-x64.sh
\end{DoxyVerb}


\subsection*{5) Setup R\-O\-S workspace }

\begin{DoxyVerb}$ source /opt/ros/indigo/setup.bash
\end{DoxyVerb}


Let's create a catkin workspace\-: \begin{DoxyVerb}$ mkdir -p /home/local/michalos/catkin_ws/src
$ cd /home/local/michalos/catkin_ws/src
$ catkin_init_workspace
\end{DoxyVerb}


Now you can add your packages to the R\-O\-S catkin workspace. For those packages you cloned from G\-I\-T\-H\-U\-B, you will need to symbolically link the package directory to the github local directory)) \begin{DoxyVerb}$ mkdir -p /usr/local/github/ros-industrial
$ cd /usr/local/github/ros-industrial
$ git clone https://github.com/ros-industrial/fanuc_experimental.git
$ cd /home/local/michalos/catkin_ws/src
$ ln -s  /usr/local/github/ros-industrial/fanuc_experimental fanuc_experimental
\end{DoxyVerb}


And do for descartes, nist\-\_\-fanuc, etc.

\subsection*{6) Compile R\-O\-S packages }

\begin{DoxyVerb}$ cd /home/local/michalos/catkin_ws
$ catkin_make -DCMAKE_BUILD_TYPE=Debug
\end{DoxyVerb}


\subsection*{7) Run R\-O\-S Simulator }

\begin{DoxyVerb}$ cd /usr/local/michalos/
$ cd nistfanuc_ws/
$ source devel/setup.bash
$roslaunch fanuc_lrmate200id_moveit_config moveit_planning_execution.launch sim:=true 
\end{DoxyVerb}


\subsection*{8) Install Java 1.\-08 from Oracle for Java C\-R\-C\-L Tool }

To build java crcl tool one needs\-: \begin{DoxyVerb}JDK 1.8+ (http://www.oracle.com/technetwork/java/javase/downloads/index.html) and
maven 3.0.5+ (https://maven.apache.org/download.cgi)
\end{DoxyVerb}


Install maven\-: \$ sudo apt-\/get install maven

Use the command\-: \begin{DoxyVerb}mvn package
\end{DoxyVerb}


If you see this message at the beginning, bummer\-: \begin{DoxyVerb}Warning: JAVA_HOME environment variable is not set. 
\end{DoxyVerb}


You can check /usr/lib/jvm to see if a 1.\-8 Java Virtual Machine has been installed. If so, skip the installation step.

So you do not have Java installed. These are instructions for the less than sudo installers. Note, you need the Oracle Java J\-D\-K 1.\-8 version, not the 1.\-7 version of Ubuntu!!!

Download, unzip and copy to /usr/local/jdk1.8.\-0\-\_\-77 or whatever is the latest 1.\-8 version.

Change you will need to change .bashrc to set the P\-A\-T\-H to know where the jdk is installed\-: \begin{DoxyVerb}for dir in /usr/local/jdk1.8.0_77/bin /usr/lib/jvm/java-[6,7,8]-*/bin /usr/local/jdk*/bin /usr/local/jdk*/bin ; do
  if [ -x $dir/java ] ; then
    javadir=$dir
  fi
done
if [ x"$javadir" = x ] ; then javadir=/usr/bin ; fi

# platform-specific environment vars

THISPLAT=`uname -s` ; export THISPLAT

case $THISPLAT in
    Linux)
    PATH=$javadir:
\end{DoxyVerb}


And make sure you source the .bashrc before you attempt run.\-sh in java crcl. The voodoo worked for me.

\subsection*{9) Install doxygen for documentation generation }

\begin{DoxyVerb}$ sudo apt-get install doxygen\end{DoxyVerb}
 